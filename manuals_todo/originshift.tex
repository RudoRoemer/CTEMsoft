
\documentclass[12pt]{amsart}
\usepackage{amsmath}
\usepackage{bm}

\title{Origin Shifts and Dynamical Scattering}
\author{M. De Graef}
\date{\today} % delete this line to display the current date

\textheight=9.0in
\textwidth=6.5in
\oddsidemargin=0.0in
\evensidemargin=0.0in
\topmargin=-0.5in

%%% BEGIN DOCUMENT
\begin{document}

\maketitle

\section{Problem Statement}

In this document, we analyze the following problem: for space groups with multiple 
origin settings, it appears that the Bloch wave approach for dynamical scattering 
produces a result (in the case of EBSD master pattern simulations) that is different
from that produced by the scattering matrix approach.  It is not clear why this is 
the case, and we analyze the equations in what follows.

\section{Basic Scattering Theory}

\subsection{Scattering matrix approach}
The Darwin-Howie-Whelan (DHW) equations can be derived from the Schr\"odinger equation by means of the following ansatz:
\begin{equation}
	\Psi(\mathbf{r}) = \sum_{\mathbf{g}} \psi_{\mathbf{g}}(\mathbf{r}) 
	\mathrm{e}^{2\pi\mathrm{i}(\mathbf{k}_0+\mathbf{g})\cdot\mathbf{r}};\label{eq:planewaves}
\end{equation}
this expression implicitly assumes that the scattered electron can only travel along the directions predicted by the 
Bragg equation, $\mathbf{k}'=\mathbf{k}_0+\mathbf{g}$.  After substitution in the perfect crystal Schr\"odinger equation,
application of the high energy approximation, and the substitution:
\begin{equation}
	\psi_\mathbf{g}(\mathbf{r}) = S_{\mathbf{g}}(\mathbf{r}) \mathrm{e}^{\mathrm{i}\theta_{\mathbf{g}}},\label{eq:psig}
\end{equation}
where $\theta_{\mathbf{g}}$ is the phase of the Fourier coefficient $U_{\mathbf{g}}$, we obtain the following system of 
coupled first order differential equations:
\begin{equation}
    \frac{\mathrm{d} S_{\mathbf{g}}(z)}{\mathrm{d}z} =
    2\pi\mathrm{i}s_{\mathbf{g}}S_{\mathbf{g}}(z) + \mathrm{i}\pi {\sum_{\mathbf{g}'}}
    \frac{1}
    {q_{\mathbf{g}-\mathbf{g}'}}S_{\mathbf{g}'}(z),\label{eq:defectequation}
\end{equation}
where
\begin{equation}
	\frac{1}{q_{\mathbf{g}}} \equiv \frac{1}{\xi_{\mathbf{g}}} + \mathrm{i}
	\frac{\mathrm{e}^{\mathrm{i} (\theta^{\prime}_{\mathbf{g}}-\theta_{\mathbf{g}})}}{\xi^{\prime}_{\mathbf{g}}};
	\label{eq:defineq}
\end{equation}
the extinction distance $\xi_{\mathbf{g}}$ and anomalous absorption length $\xi'_{\mathbf{g}}$ are defined as:
\begin{equation}
	\frac{1}{\xi_{\mathbf{g}}}\equiv \frac{\vert U_{\mathbf{g}}\vert}{\vert\mathbf{k}_0+\mathbf{g}\vert\cos\alpha};\qquad
	\frac{1}{\xi'_{\mathbf{g}}}\equiv \frac{\vert U'_{\mathbf{g}}\vert}{\vert\mathbf{k}_0+\mathbf{g}\vert\cos\alpha};
\end{equation}
$U_{\mathbf{g}} = (2me/h^2) V_{\mathbf{g}} $, $\alpha$ is the angle between the beam direction and $\mathbf{k}_0+\mathbf{g}$, and $\mathbf{k}_0$ is 
the incident wave vector corrected for refraction.  The absorption potential Fourier coefficients are represented by $U'_{\mathbf{g}} = (2me/h^2) V'_{\mathbf{g}}$.

In the case of an origin shift of the unit cell, as found, for instance, in space groups with multiple origin settings, all the atom coordinates in the 
unit cell are shifted by the same fractional vector $\bm{\delta}$; in space group 227, we have $\bm{\delta}=\left(\frac{1}{8},\frac{1}{8},\frac{1}{8}\right)$.
This origin shift is represented by a phase shift of the Fourier coefficients of the lattice potential, so that we have:
\begin{equation}
	U_{\mathbf{g}} \rightarrow \mathrm{e}^{2\pi\mathrm{i}\mathbf{g}\cdot\bm{\delta}} U_{\mathbf{g}} = \mathrm{e}^{\mathrm{i}(\theta_{\mathbf{g}}+2\pi\mathbf{g}\cdot\bm{\delta})} \vert U_{\mathbf{g}}\vert.
\end{equation}
Depending on the reciprocal lattice vector $\mathbf{g}$, some Fourier coefficients will be phase shifted and others will remain unchanged.  It should be noted
that the Fourier coefficients of the absorptive potential are modifed by the same phase factor.

When the origin shift is carried through the DHW derivation, the substitution relation~\ref{eq:psig} becomes:
\begin{equation}
	\psi_\mathbf{g}(\mathbf{r}) = S_{\mathbf{g}}(\mathbf{r}) \mathrm{e}^{\mathrm{i}(\theta_{\mathbf{g}}+2\pi\mathbf{g}\cdot\bm{\delta})},\label{eq:psig2}
\end{equation}
and the phase factor in the definition of the factors $q_{\mathbf{g}}$ in equation~\ref{eq:defineq} becomes:
\begin{equation}
	\frac{1}{q_{\mathbf{g}}} \equiv \frac{1}{\xi_{\mathbf{g}}} + \mathrm{i}
	\frac{\mathrm{e}^{\mathrm{i} (\theta^{\prime}_{\mathbf{g}}+2\pi\mathbf{g}\cdot\bm{\delta} -\theta_{\mathbf{g}}-2\pi\mathbf{g}\cdot\bm{\delta})}}{\xi^{\prime}_{\mathbf{g}}},
	\label{eq:defineq2}
\end{equation}
and the two origin phase shifts cancel out.  This means that the DHW approach, as formulated above, is invariant under an origin shift; this is easily verified numerically.
Therefore, any simulation that relies on the scattering matrix approach, such as the EBSD master pattern, will also be invariant with respect to a unit cell origin shift.

\newcommand{\Cgj}[2]{C_{\mathbf{#1}}^{(#2)}}

\subsection{Bloch wave approach}
For the Bloch wave approach, we write the electron wave function as follows:
\begin{equation}
	\Psi(\mathbf{r})=\sum_j\alpha^{(j)}C^{(j)}(\mathbf{r})\mathrm{e}^{2\pi \mathrm{i}
        \mathbf{k}^{(j)}\cdot\mathbf{r}}=\sum_j\alpha^{(j)}\sum_{\mathbf{g}}\Cgj{g}{j} 
	\mathrm{e}^{2\pi \mathrm{i}(\mathbf{k}^{(j)} +\mathbf{g})\cdot\mathbf{r}}.
	\label{eq:blochexpansion2}
\end{equation}
The coefficients $\Cgj{g}{j}$ are known as the \textit{Bloch wave coefficients}, 
while the coefficients $\alpha^{(j)}$ are the \textit{excitation amplitudes}.
Each of the functions $C^{(j)}(\mathbf{r})$ has the periodicity of the lattice and is
known as a Bloch wave.

Substitution in the Schr\"odinger equation results in the following system of equations:
\begin{equation}
	\left[k_0^2-(\mathbf{k}+\mathbf{g})^2\right]C_{\mathbf{g}} + 
	\sum_{\mathbf{h}\neq 
	\mathbf{g}}U_{\mathbf{g}-\mathbf{h}}C_{\mathbf{h}}=0.
	\label{eq:bloch}
\end{equation}
After an origin shift, this equation becomes:
\begin{equation}
	\left[k_0^2-(\mathbf{k}+\mathbf{g})^2\right]C_{\mathbf{g}} + 
	\sum_{\mathbf{h}\neq 
	\mathbf{g}} \mathrm{e}^{2\pi\mathrm{i}(\mathbf{g}-\mathbf{h})\cdot\bm{\delta}}U_{\mathbf{g}-\mathbf{h}}C_{\mathbf{h}}=0.
	\label{eq:blochshifted}
\end{equation}
In other words, the Bloch wave dynamical matrix off-diagonal elements are phase shifted with respect to the original unit cell origin choice.
This makes sense, since Bloch waves travel at specific locations through the crystal lattice. However, the phase shift does not affect the 
diffracted intensities for a regular scattering experiment (e.g., zone axis diffraction pattern, or CBED); the phases of the individual 
Bloch waves \textit{do} change with an origin shift.  Note that the diagonal entries of the dynamical matrix do not change when the unit cell origin is changed.
The main question to be answered next is whether or not this origin shift results in differences in the EBSD master pattern.

\subsection{EBSD master patterns}
The probability of a BSE being scattered in the direction $\mathbf{k}_0$ can be expressed as:
\begin{equation}
	\mathcal{P}(\mathbf{k}_0) = \sum_{\mathbf{g}} 
    \sum_{\mathbf{h}} S_{\mathbf{g}\mathbf{h}}L^B_{\mathbf{g}\mathbf{h}},
    \label{eq:prob}
\end{equation}
where
\begin{subequations}
\begin{align}
    S_{\mathbf{g}\mathbf{h}} &\equiv \sum_{n}\sum_{i\in\mathcal{S}_n} Z^2_n\,e^{-M^{(n)}_{\mathbf{h}-\mathbf{g}}}\,e^{2\pi\mathrm{i} 
    (\mathbf{h}-\mathbf{g})\cdot\mathbf{r}_{i}};\label{eq:defa}\\
    L^B_{\mathbf{g}\mathbf{h}} &\equiv \sum_{j}\sum_{k} 
    C^{(j)\ast}_{\mathbf{g}}\alpha^{(j)\ast}\mathcal{I}_{jk}\alpha^{(k)}
    C^{(k)}_{\mathbf{h}}.\label{eq:defb}
\end{align}
\end{subequations}
For an origin shift of $\bm{\delta}$, it is easy to see that the $S_{\mathbf{g}\mathbf{h}}$ matrix is changed to:
\begin{equation}
	S_{\mathbf{g}\mathbf{h}} \equiv \sum_{n}\sum_{i\in\mathcal{S}_n} Z^2_n\,e^{-M^{(n)}_{\mathbf{h}-\mathbf{g}}}\,e^{2\pi\mathrm{i} 
    (\mathbf{h}-\mathbf{g})\cdot(\mathbf{r}_{i}+\bm{\delta})}.\label{eq:defa2}
\end{equation}


























\end{document}