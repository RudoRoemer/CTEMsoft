
\documentclass[DIV=calc, paper=letter, fontsize=11pt]{scrartcl}	 % A4 paper and 11pt font size

\usepackage[body={6.5in,9.0in},
  top=1.0in, left=1.0in]{geometry}
  
\usepackage[english]{babel} % English language/hyphenation
\usepackage[protrusion=true,expansion=true]{microtype} % Better typography
\usepackage{amsmath,amsfonts,amsthm} % Math packages
\usepackage[svgnames]{xcolor} % Enabling colors by their 'svgnames'
\usepackage[hang, small,labelfont=bf,up,textfont=it,up]{caption} % Custom captions under/above floats in tables or figures
\usepackage{booktabs} % Horizontal rules in tables
\usepackage{fix-cm}	 % Custom font sizes - used for the initial letter in the document
\usepackage{epsfig}
\usepackage{sectsty} % Enables custom section titles
\allsectionsfont{\usefont{OT1}{phv}{b}{n}} % Change the font of all section commands

\usepackage{fancyhdr} % Needed to define custom headers/footers
\pagestyle{fancy} % Enables the custom headers/footers
\usepackage{lastpage} % Used to determine the number of pages in the document (for "Page X of Total")
\usepackage{color}

\usepackage{fancyvrb}% used to include files verbatim
%\usepackage{chemsym}

\usepackage{hyperref}
\usepackage[squaren]{mySIunits}

\usepackage[backend=bibtex,style=authoryear,maxnames=15]{biblatex}
\renewbibmacro{in:}{}

% Count total number of entries in each refsection
\AtDataInput{%
  \csnumgdef{entrycount:\therefsection}{%
    \csuse{entrycount:\therefsection}+1}}

% Print the labelnumber as the total number of entries in the
% current refsection, minus the actual labelnumber, plus one
\DeclareFieldFormat{labelnumber}{\mkbibdesc{#1}}    
\newrobustcmd*{\mkbibdesc}[1]{%
  \number\numexpr\csuse{entrycount:\therefsection}+1-#1\relax}


\addbibresource[label=papers]{citations.bib}
%\addbibresource[label=books]{mypubs.bib}
%\addbibresource[label=edited]{mypubs.bib}
%\addbibresource[label=chapters]{mypubs.bib}


% Headers - all currently empty
\lhead{}
\chead{}
\rhead{}

% Footers
\lfoot{\textsf{CTEMECCI} manual, v1.0, \today}
\cfoot{}
\rfoot{\footnotesize Page \thepage\ of \pageref{LastPage}} % "Page 1 of 2"

\renewcommand{\headrulewidth}{0.0pt} % No header rule
\renewcommand{\footrulewidth}{0.4pt} % Thin footer rule

\usepackage{lettrine} % Package to accentuate the first letter of the text
\newcommand{\initial}[1]{ % Defines the command and style for the first letter
\lettrine[lines=3,lhang=0.3,nindent=0em]{
\color{DarkGoldenrod}
{\textsf{#1}}}{}}

\usepackage{titling} % Allows custom title configuration

\newcommand{\HorRule}{\color{DarkGoldenrod} \rule{\linewidth}{1pt}} % Defines the gold horizontal rule around the title

\pretitle{\vspace{-1.5in} \begin{center} \HorRule \fontsize{25}{25} \usefont{OT1}{phv}{b}{n} \color{DarkRed} \selectfont} % Horizontal rule before the title

\title{Electron Channeling Contrast\\ Image (ECCI) Simulations} % Your article title

\posttitle{\par\end{center}\vskip 0.5em} % Whitespace under the title

\preauthor{\begin{center}\large \lineskip 0.5em \usefont{OT1}{phv}{b}{sl} \color{DarkRed}} % Author font configuration

\author{\vspace*{-0.7in}} % Your name

\postauthor{\footnotesize \usefont{OT1}{phv}{m}{sl} \color{Black} % Configuration for the institution name

\par\end{center}\HorRule} % Horizontal rule after the title
\date{Program Manual, v1.0, \today}

\newcommand{\ctp}{\textsf{CTEMsoft-2013}}
%
\newcommand{\upg}[1]{\mathrm{i}U_{\mathbf{#1}}^{\prime}}
\newcommand{\combo}[1]{U_{\mathbf{#1}}+\upg{#1}}
\newcommand{\upgcombo}[2]{2k_{0}s_{\mathbf{#1}}+\upg{#2}}
\newcommand{\ugh}[2]{U_{\mathbf{#1}-\mathbf{#2}}}
\newcommand{\ughp}[2]{U_{\mathbf{#1}'-\mathbf{#2}}}
\newcommand{\ughpp}[2]{U_{\mathbf{#1}-\mathbf{#2}'}}
\newcommand{\kkg}[1]{k_{0}^{2}-(\mathbf{k}+\mathbf{#1})^{2}}
\newcommand{\Cg}[1]{C_{\mathbf{#1}}}
\newcommand{\Cgj}[2]{C_{\mathbf{#1}}^{(#2)}}
\newcommand{\Cgjp}[2]{C_{\mathbf{#1}'}^{(#2)}}
\newcommand{\Cgja}[2]{C_{\mathbf{#1}}^{(#2)\ast}}
\newcommand{\button}[1]{\colorbox{green}{\textsf{#1}} button}


\begin{document}
\maketitle

\begin{figure*}[h]
\leavevmode\centering
\epsffile{figs/SEMlogo}
\end{figure*}

\renewcommand{\contentsname}{Table of Contents}
{\small\tableofcontents}

\newpage
\section{Introduction}
This manual describes a set of two programs, 
one written in Fortran-90,\footnote{f90 is a much richer language than the original fortran-f77, and is
used for all programs in the \ctp\ package.} 
the other in IDL,\footnote{The \textit{Interactive Data Language} is an interpreted scripting language with extensive graphics capabilities.} 
that can be used for the simulation of near zone axis electron channeling contrast images (ECCIs).  Such images can be
observed in a scanning electron microscope (SEM) when the incident beam is tilted to a diffraction condition, 
and the back-scatter electron (BSE) signal is displayed as a function of the beam position.  The main f90 program
is called \textsf{CTEMECCI} (note that all programs in the \ctp\ package start with the letters ``CTEM'').
The output generated by this program can then be visualized by the IDL routine \textsf{ECCIDisplay.pro}.  

On the following pages we will try to accomplish four tasks:
\begin{enumerate}
	\item Explain briefly the underlying image formation theory and the numerical approach followed by the f90 program (section~\ref{sec:theory});
	\item Document the input files for the f90 program (section~\ref{sec:f90ecci});
	\item Document the IDL interface (section~\ref{sec:idl});
	\item Explain the use of these programs by means of a few basic examples (section~\ref{sec:examples}).
\end{enumerate}

At the time of writing of this manual, these programs have been successfully compiled on the Mac OS X platform using the commercial \textit{Absoft} compiler,
as well as the public domain gfortran compiler.  There is no basic reason why this code should not work on Windows and Linux systems, and efforts are underway
to make sure that the code will run on all platforms. It would be interesting to see some of this code
implemented in a super computer setting, since many of the routines should be quite parallellizable.  Where possible, some of the computations
are carried out using the OpenMP directives, so that multiple cores can be used.  This aspect of the code is still under further development,
and we anticipate that GPUs will become useful as well, using the OpenCL approach.

The \ctp\ package is entirely written in f90 and does not use any of the newer commands available in the 95 and 2013 versions.
The source code is extensively commented, using regular comment lines, but also using DOxygen documentation generation commands.  Hence
there exists an extensive on-line documentation of all variables, variable types, modules, subroutines, functions, etc. for the latest version 
of the code.  For selected programs, more extensive manual pages are available.  If interested, please contact the author for further information.

The visualization part of the code consists of a series of  IDL routines that are available as source code or in the form of a Virtual Machine application. 
If you have an IDL license, then you can compile and run the IDL source code; alternatively, if you do not have a license,
then you can use one of the VM apps to perform the same task.  Note that in the VM environment, you will not be able to alter/compile the 
source code.

\newpage
\section{Electron channeling contrast images: the underlying theory\label{sec:theory}}
The computation of an SEM-generated ECCI can be carried out by means of the scattering matrix approach, and is rather similar to 
what is needed for zone axis or systematic row (S)TEM simage simulations.  In fact, the \textsf{CTEMECCI} program is directly 
derived from the \textsf{CTEMZAdefect} program, and we refer to the manual for this program for detailed information on the 
image formation theory in the presence of defects.  There are some differences in the ECCI case, notably that the electron wave length is a bit longer due 
to the lower accelerating voltage.  Since the ECCI signal is caused by backscattered electrons (BSEs), a first order approach to 
ECCI image simulations involves computation of the wave function as a function of depth inside the sample (a computation that is similar 
to that used for (S)TEM images) and combine that with the depth integration approach used for the \textsf{CTEMECP} program.

In the presence of a defect, we need to modify the expression for the probability of scattering from each subset $\mathcal{S}$ of atomic sites within the 
unit cell for a given incident beam direction $\mathbf{k}_0$:
\begin{equation}
	\mathcal{P}(\mathbf{k}_0) = \sum_{\mathbf{g}} 
    \sum_{\mathbf{h}} S_{\mathbf{g}\mathbf{h}}L_{\mathbf{g}\mathbf{h}},
    \label{eq:prob}
\end{equation}
with
\begin{subequations}
\begin{align}
    S_{\mathbf{g}\mathbf{h}} &\equiv \sum_{n}\sum_{i\in\mathcal{S}_n} Z^2_n\,\mathrm{e}^{-M^{(n)}_{\mathbf{h}-\mathbf{g}}}\,e^{2\pi\mathrm{i} 
    (\mathbf{h}-\mathbf{g})\cdot\mathbf{r}_{i}};\label{eq:defa}\\
    L_{\mathbf{g}\mathbf{h}} &\equiv \frac{1}{z_{0}}\sum_{j}\sum_{k}\int\limits_{0}^{z_{0}} \mathrm{d}z\, 
    C^{(j)\ast}_{\mathbf{g}}(z)\alpha^{(j)\ast}(z) \mathrm{e}^{-2\pi(\alpha_{jk}(z)+\mathrm{i}\beta_{jk}(z))z} \alpha^{(k)}(z)
    C^{(k)}_{\mathbf{h}}(z).\label{eq:defb}
\end{align}
\end{subequations}
Note that all the Bloch wave parameters now have acquired a $z$-dependence that is caused by the fact that the defects in the 
material will cause variations of the wave function along the incident beam direction.  This complicates matters, since now
we have to carry out a Bloch wave computation at each depth (we'll discretize the integral and repeat the computation at regular
depth increments).  

Alternatively, we should be able to rewrite this expression in terms of the wave function expansion that is used for the 
scattering matrix approach, which would then allow us to re-use the (S)TEM defect source code.  Either way should provide 
the same result.  In the scattering matrix approach, the total electron wave function is expressed as a linear combination of
plane waves, traveling in the directions allowed by the Bragg equation:
\begin{equation}
	\Psi(\mathbf{r}) = \sum_{\mathbf{g}} \psi_{\mathbf{g}}(\mathbf{r}) \mathrm{e}^{2\pi\mathrm{i}(\mathbf{k}_0+\mathbf{g})\cdot\mathbf{r}}
\end{equation}
The modulus squared of this expression provides the probability of finding an electron at position $\mathbf{r}$:
\begin{equation}
	\vert\Psi(\mathbf{r})\vert^2 = \sum_{\mathbf{g}} \sum_{\mathbf{h}} \psi^{\ast}_{\mathbf{g}}(\mathbf{r}) \psi_{\mathbf{h}}(\mathbf{r})
	 \mathrm{e}^{2\pi\mathrm{i}(\mathbf{h}-\mathbf{g})\cdot\mathbf{r}}.
\end{equation}
Using the standard substitution
\[
	\psi_\mathbf{g}(\mathbf{r}) = S_{\mathbf{g}}(\mathbf{r}) \mathrm{e}^{\mathrm{i}\theta_{\mathbf{g}}}
\]
we obtain:
\begin{equation}
	\vert\Psi(\mathbf{r})\vert^2 = \sum_{\mathbf{g}} \sum_{\mathbf{h}} S^{\ast}_{\mathbf{g}}(\mathbf{r}) S_{\mathbf{h}}(\mathbf{r})
	\mathrm{e}^{\mathrm{i}(\theta_{\mathbf{h}}-\theta_{\mathbf{g}})}
	 \mathrm{e}^{2\pi\mathrm{i}(\mathbf{h}-\mathbf{g})\cdot\mathbf{r}}.
\end{equation}
Surpressing the position dependence of the integration column, and keeping only the depth dependence, we obtain
for the thickness-integrated probability of BSE scattering for a given incident beam direction $\mathbf{k}_0$:
\begin{equation}
	\mathcal{P}(\mathbf{k}_0) = \frac{1}{z_0}\sum_{n}\sum_{i\in\mathcal{S}_n}\sum_{\mathbf{g}}\sum_{\mathbf{h}} \int\limits_{0}^{z_{0}} \mathrm{d}z\,
	 Z^2_n\,\mathrm{e}^{-M^{(n)}_{\mathbf{h}-\mathbf{g}}}\,\mathrm{e}^{2\pi\mathrm{i} (\mathbf{h}-\mathbf{g})\cdot\mathbf{r}_{i}} 
%	\mathrm{e}^{\mathrm{i}(\theta_{\mathbf{h}}-\theta_{\mathbf{g}})} 
	S^{\ast}_{\mathbf{g}}(z) S_{\mathbf{h}}(z)
    \label{eq:prob2}
\end{equation}
[Note that one of the phase factors has disappeared, due to the fact that it is explicitly taken into account in the 
dynamical matrix computation; keeping it in the equation above would result in double-counting of the phase factor.]
This can be rewritten in the same form as for the ECP computation:
\begin{equation}
	\mathcal{P}(\mathbf{k}_0) = \sum_{\mathbf{g}}\sum_{\mathbf{h}} S_{\mathbf{g}\mathbf{h}} L_{\mathbf{g}\mathbf{h}},
    \label{eq:prob2}
\end{equation}
where the matrices are redefined as
\begin{subequations}
\begin{align}
    S_{\mathbf{g}\mathbf{h}} &\equiv \sum_{n}\sum_{i\in\mathcal{S}_n} Z^2_n\,\mathrm{e}^{-M^{(n)}_{\mathbf{h}-\mathbf{g}}}\,\mathrm{e}^{2\pi\mathrm{i} 
    (\mathbf{h}-\mathbf{g})\cdot\mathbf{r}_{i}};\label{eq:defnewa}\\
    L_{\mathbf{g}\mathbf{h}} &\equiv \frac{1}{z_{0}}\int\limits_{0}^{z_{0}} \mathrm{d}z\,  S^{\ast}_{\mathbf{g}}(z) S_{\mathbf{h}}(z).\label{eq:defnewb}
\end{align}
\end{subequations}
These equations are equivalent to the original Bloch wave equations in the presence of defects; they are implemented in the \textsf{CTEMECCI} program.
The first matrix can be computed ahead of time for a given set of diffracted beams, whereas the second matrix must be recomputed for each image column.
If we use Bethe potentials (not yet implemented in the current version of the program), then both matrices must be recomputed for each image pixel/beam
orientation. The wave amplitudes $S_{\mathbf{g}}(z)$ follow from the scattering matrix approach in a way that is identical to that used in the (S)TEM simulations.

In practical terms, the integration must be replaced by a summation of the form:
\begin{equation}
	L_{\mathbf{g}\mathbf{h}} \approx \frac{1}{N}\sum\limits_{i=1}^{N} S^{\ast}_{\mathbf{g}}(z_i) S_{\mathbf{h}}(z_i),
\end{equation}
which requires the computation of all scattered amplitudes for all values of $z_i$, followed by the summation.  The value of 
$N$ is determined by the integration stepsize $\Delta z$, i.e., $N=z_0/\Delta z$.  The propagation of the electron wave
inside the crystal is then incorporated by the scattering matrix at the correct depth, via the recursion:
\[
	\mathbf{S}(z_i) = \mathcal{S}_i\mathbf{S}(z_{i-1}).
\]
The overall integration depth should probably be determined from Monte Carlo simulations; for now, we can simply determine the appropriate 
$z_0$ value by trial-and-error.

In terms of the \textsf{CTEMECCI} program itself, the major program portions are:
\begin{itemize}
	\item read crystal structure and general parameter files;
	\item determine list of contributing reflections and incident wave vector range;
	\item determine the defect configuration and associated phase shift factors;
	\item for each incident beam direction, compute $\mathcal{P}(\mathbf{k}_0)$, integrating over the thickness.
	\item store all computed patterns in a file that can be read by the IDL \textsf{ECCIDisplay} program.
\end{itemize}

\section{The \protect\textsf{CTEMECCI.f90} program\label{sec:f90ecci}}

\subsection{Program overview\label{sec:f90overviewecci}}
The \textsf{CTEMECCI} program is a relatively basic program and is in its structure rather similar to the \textsf{CTEMZAdefect} program.
The only major difference lies in the computation of the thickness-integrated intensity rather than individual transmitted and 
diffracted beam intensities.  The program input uses the namelist file approach described in the next section, and the output is a 
binary file, for now with a special format.\footnote{At some point in the future, all output files from the \ctp\ programs will use
the public domain HDF5 file format.}  

The program reads the input and crystal structure files,  a foil descriptor file, and one or more defect descriptor files; then an ECCI image is computed for 
each of a range of incident beam directions.  The beam directions can be defined as a cone or as a linear trace, corresponding to the program modes
``array'' and ``trace'', respectively.  The resulting ECCI images are stored in an output file that can be visualized using the \textsf{ECCIDisplay} IDL interface.

\subsection{Namelist input files\label{sec:f90inputecp}}
The main input file for this program, \textsf{CTEMECCI.nml}, is formatted as follows:
\fvset{frame=lines,formatcom=\color{blue},fontsize=\footnotesize}
\VerbatimInput{../templatefolder/CTEMECCI.template}
Most of the parameters are straightforward.  The crystal structure file can be created with 
the \textsf{CTEMmkxtal} program.  The most important parameter is the \textsf{progmode} or program mode.  This
parameter can take on one of two values ``array'' or ``trace''.  In ``array'' mode, the program will generate a series of 
ECCI images for incident beam directions that are organized on a square grid in the ECP, distributed symmetrically with
respect to the point defined by the parameter \textsf{lauec} or Laue Center.  If \textsf{lauec} is set to $(0,0)$, then the array
of ECCI images will be symmetric with respect to the zone axis orientation, otherwise the pattern will be symmetric with 
respect to the point identified by the \textsf{lauec} parameter.  The incident beam directions are distributed symmetrically inside the cone,
the radius of which is defined by the parameter \textsf{ktmax} in units of $\vert\mathbf{g}_a\vert$ (refer to the manual for the 
ECP program for more information).  The parameter \textsf{dkt} defines the stepsize of the incident beam grid in the same
units; hence, the number of samples along the horizontal direction will be the integer nearest to \textsf{ktmax}$/$\textsf{dkt}.

The second program mode, ``trace'', requires two coordinate pairs corresponding to the start and end points of a straight line,
along with the number of steps along the line; the corresponding parameters are \textsf{lauec2} for the end point and \textsf{nktstep}
for the number of steps.  

The program also needs a description file for the foil (although the sample is really a bulk sample in this case); for a description 
of this file, as well as all of the defect files, please refer to the manual for the \textsf{CTEMZAdefect} program.  The \textsf{CTEMECCI}
program takes one additional type of defect descriptor file, namely for dislocations that penetrate the top surface.  Such dislocations
are described by a displacement field first derived by E. Yoffe, hence all file names related to these defects have the letter Y in them.
The input format for Yoffe dislocations is somewhat similar to the regular format, with a few important distinctions. An example namelist
file is shown here:
\fvset{frame=lines,formatcom=\color{blue},fontsize=\footnotesize}
\VerbatimInput{../templatefolder/Ydislocation.template}
Note that the Yoffe-Shaibani-Hazzledine model for the displacement field of a surface penetrating dislocation is only valid 
for materials that are close to elastically isotropic.  Hence the displacement field only depends on the Poisson ratio, which is
one of the input parameters of the dislocation file (in the future, this parameter will be moved to the foil description file).
The position of the dislocation is the point where the dislocation penetrates the sample surface, not the location at the foil
center plane as is the case for the (S)TEM image simulation programs.  Note also that the \textsf{CTEMECCI} program will accept 
regular dislocation files; these are useful for dislocations that are parallel to the surface, for instance misfit dislocations 
at some distance from the top surface.  Stacking faults can also be defined, as described below.

Experience has shown that the ECCI images are not very sensitive to reflections from the higher order Laue zones, so the minimum
$d$-spacing parameter \textsf{dmin} can be set to a significantly larger value than in the \textsf{CTEMECP} program.  A value of 
about $0.1$ is reasonable; in the current implementation, which does not use Bethe potentials, a smaller value will significantly 
increase the computation times.

The size of the ECCI images is determined by the value of \textsf{DF\_npix} and \textsf{DF\_npiy}, and individual image pixels have a size 
of \textsf{DF\_L}$\times$\textsf{DF\_L} \nano\squaren\meter.  So a change in \textsf{DF\_L} is equivalent to zooming in or out of the image.
However, make sure that you understand the way defects are defined, as described in the manual for the \textsf{CTEMZAdefect} program.

The \textsf{summode} parameter has two possible values: ``diag'' and ``full''.  When set to ``diag'', the program uses only the diagonal 
entries of the $L$ and $S$ matrices in the summation; for ``full'', all entries are used.  The ``diag' option causes a slightly faster
program execution.  One can experiment with these options, but likely the ``full'' summation is more correct than only using the diagonal 
entries.  For large integration depths, it might be sufficient to only use the diagonal entries, but this needs to verified further.

Finally, for visualization purposes, the parameter \textsf{ECPname} must be present; it must point to an existing file generated by the 
\textsf{CTEMECP} program for the same zone axis and accelerating voltage as used in the ECCI run.  The value of the \textsf{ktmax}
parameter in the ECP run must be larger than that used for the ECCI run, so that the incident beam directions for the ECCI program 
can be displayed on the EC pattern.

\subsection{Stacking fault namelist files}
For the ECCI program, stacking fault descriptor files are slightly different from the ones for the \textsf{CTEMZAdefect} program.
A stacking fault is defined by means of a position parameter \textsf{(id,jd)} (fractional coordinates) as before, except that the 
coordinates are located in the top surface of the foil instead of in the center plane.  The combination of the SF plane and the coordinates defines the trace of the 
fault on the surface; the separation parameter then represents the distance between the points where the partials intersect the 
surface, and the partials will be located symmetrically with respect to the point \textsf{(id,jd)}.  The syntax of the SF input file
remains unchanged from that used by other programs, which means that stacking fault locations in the image will depend on which 
program is used to compute the images (TEM vs.\ ECCI).  Note that stacking fault files must also have the \textsf{poisson} variable in
them.  All partial dislocations are assumed to be Yoffe-type dislocations, hence they make use of the Poisson ratio only for the computation
of the displacement field.


\subsection{Bethe parameters namelist file\label{sec:f90BetheParameters}}
The Bethe potential approximation used in this program can be adjusted by the
user via the \textsf{BetheParameters.nml} namelist file, which must be present 
in the folder in which the program is executed; if this file is not present, then the
program will assume default values for all three parameters.  The values shown below
are the current program defaults.
\fvset{frame=lines,formatcom=\color{blue},fontsize=\footnotesize}
\VerbatimInput{../templatefolder/BetheParameters.template}
Note that the default value  is much too large for SEM-type runs and will result in very long program execution times.
For ECCI images, \textsf{cutoff} should be set to $20$--$30$ or so; note that this is not the case for ECPs, so care 
must be taken when executing these programs.

\section{The \protect\textsf{ECCIDisplay.pro} program\label{sec:idl}}

\subsection{Program overview\label{sec:idloverview}}
The data file generated by the \textsf{CTEMECCI} program consists of geometrical information and an array of ECCI images.
This file can be visualized using the IDL program \textsf{ECCIDisplay.pro}, described in this section.

The \textsf{ECCIDisplay.pro} routine requires an IDL program license or can be executed using the Virtual Machine, without the need for a license. 
If you do have a license, then make sure that the folder containing the routines is part of your IDL pathname,
and that IDL is properly installed for the UNIX shell that you will be using (csh, bsh, etc...).
To execute the program, first start an IDL session in a terminal window (using the /Appplications/Utilities/Terminal program), 
start IDL, compile the \textsf{ECCIDdisplay.pro} routine by typing:
\begin{verbatim}
	IDL> .r ECCIDisplay <return>
\end{verbatim}
at the IDL prompt and hitting return, followed by typing 
\begin{verbatim}
	IDL> ECCIDisplay <return>
\end{verbatim}
to start the display program.  The main program widget window will appear as well as a file selector window.  
Details of all the windows are described in the following subsections.

In addition to requiring an IDL license, the program expects the X-windows environment to be installed.  On the Mac, this corresponds
to the \textsf{XQuartz} program which can be downloaded from the open source site \textsf{http://xquartz.macosforge.org/}.  This program
must be installed in the system /Applications/Utilities folder and requires OS X 10.6 or later.

Useful thing to know: if the \textsf{ECCIDisplay.pro}  program hangs for some reason, you can reset your IDL session by typing
\begin{verbatim}
	IDL> .reset <return>
\end{verbatim}
This will destroy all program widgets and reset IDL to its original state.  If that does not work, then you may have to ``force quit'' the Terminal program
using the standard Esc-Option-Command key stroke.

There is also a Virtual Machine version of this program, which can be executed without an IDL license.

\subsection{Main window\label{sec:idlmain}}
This display program will load an ECCI data file as well as an ECP pattern, and several geometrical quantities.
The main window contains no user adjustable parameters, and only displays a few parameters from the computation.
The main display widget is shown in Fig.~\ref{fig:widget1}.  

\begin{figure}[h]
\leavevmode\centering
\epsffile{figs/ECCIwidget1}
\caption{\label{fig:widget1}Main \textsf{ECCIDisplay.pro} program widget.}
\end{figure}

The only user options are listed at the bottom of the widget: the \button{ECCI File} and the \button{Quit}, the latter's function being 
rather clear.  When the \button{ECCI File} is selected, a file selection interface appears, and the user can select an ECCI data file. 
Several parameters will be read from the file, and their values displayed in the main widget.  Note that  all widgets 
can be repositioned anywhere on the screen, and that the window locations will be stored in the preferences file described in the next section.  

Once a file has been loaded, the text fields will be updated, and a new widget appears, the ECP widget.  Depending on the setting of the 
\textsf{progmode} and \textsf{lauec} parameters, his window will have a slightly different appearance; there are three different cases, 
shown in Fig.~\ref{fig:widget2}.  From left to right, ``array'' mode with \textsf{lauec}$=(0,0)$, ``array'' mode with \textsf{lauec}$=(5,7)$,
and ``trace'' mode with \textsf{lauec}$=(11,2)$ and \textsf{lauec2}$=(6.5,6.5)$.  Each green cross indicates an incident beam direction for
which an ECCI image has been computed.  Clicking near a cross will cause a new widget to pop up (see Fig.~\ref{fig:widget3}), showing the corresponding ECCI image.

The graphics window on the left will display the ECCI image corresponding to the selected beam direction.  The intensity can be scaled in two different
ways: ``global'', which means that the minimum and maximum intensity of the entire computed set will be taken as black and white, respectively, and 
``by image'', which scales the intensity to maximize the contrast in an individual image.

When one clicks at any point inside the ECCI image, the corresponding ECP pattern will be displayed in the right window.  Note that this is 
nearly always a low resolution image, with the same number of pixels as there are ECCI images in the data set.  The \button{Save} can be used to
save either image to a file.  Below each image, the minimum and maximum intensity are displayed; these are on an arbitrary absolute scale.

The averaging radius can be set to a non-zero value and then the image on the right will be the average ECCI image, averaged over a circle with 
the set radius (in units of the spacing between the green crosses in the ECP pattern window).  The circle will appear in purple on the ECP pattern.

\begin{figure}[h]
\leavevmode\centering
\epsfxsize=5in\epsffile{figs/ECCIwidget2}
\caption{\label{fig:widget2}ECP pattern display widget for three different program parameter settings.}
\end{figure}

\begin{figure}[h]
\leavevmode\centering
\epsfxsize=3in\epsffile{figs/ECCIwidget3}
\caption{\label{fig:widget3}ECCI image display widget.}
\end{figure}

In ``array'' mode, the third line in the ECCI Image Widget will show \textsf{Dimension of ECCI Mosaic} followed by a number.  This number
represents the size (in pixels) of the mosaic image that can be created by combining all ECCI images in the data set into a single image,
with each tile positioned at the correct location relative to the incident beam directions.  The number that is originally displayed in 
this text field is the default size (no image scaling performed), but one can change this to any other desired number of pixels (note that
the image scaling is performed by regridding the array of pixels; no smoothing is performed, and, obviously, no extra data is introduced 
by interpolation).  In ``trace'' mode, the third line reads \textsf{Number of images in row}; the set of images along the trace will be 
displayed in a mosaic with that many images in a row, and as many rows as needed.


\subsection{Preferences file\label{sec:idlpref}}
Upon the first execution of the \textsf{ECCIDisplay.pro} routine, a preferences file will be created in the user's home folder.  
The file is called \textsf{.ECCIgui.prefs}; the starting period
means that the file will not show up in a Finder window or on the UNIX command line when a simple \textsf{ls} command is issued.  This is a 
regular editable text file consisting of name::value pairs, although there is no real
need for the user to ever edit this file.  The first line shows the number of entries in the file, and then each entry is listed on a 
separate line.  A commented version of the preferences file is as follows:
\fvset{frame=lines,formatcom=\color{blue},fontsize=\footnotesize,numbers=left}
\VerbatimInput{ECCIgui-commented.prefs}
Note that the comments are not part of the actual preferences file.  All lines must be present in the file or the program will exit with an error message; the order 
of the lines is not important, but the file will always be written in the same order.
The values shown above are not default values, but represent a random snapshot of the program status after it has been used for a while.

When the program starts, it will first internally initialize all variables to default values, and then read the preferences
file, if it exists.  Then the widgets will be created using the preference values.  When the program is ended normally (by
pressing the \button{QUIT}), all current values, including the widget locations, are written to the preferences
file.


\newcommand{\dislo}[6]{$\frac{1}{6}[#1\,#2\,#3]\vert [#4\,#5\,#6]$}
\newcommand{\dislop}[3]{$\frac{1}{2}[#1\,#2\,#3]$}
\newcommand{\bo}{\bar{1}}
\newcommand{\bt}{\bar{2}}

\newpage
\section{A few worked examples\label{sec:examples}}
In this final section, we show two simple examples to illustrate how to use the combination of f90 and IDL programs 
to obtain simulated ECCI images.  The examples consist of misfit and other dislocations in GaP, and a stacking fault pyramid in GaP.
Note that these examples assume that the \textsf{CTEMECP} program has already been used to produce an ECP pattern file, using the 
following input file: (see \textsf{SEMECP} manual for more information)
\fvset{frame=lines,formatcom=\color{blue},fontsize=\scriptsize,numbers=left}
\VerbatimInput{../examples/ECCISimulations/ECP5.nml}

We will use the GaP Zinc-Blende structure for this example.  This structure has space group $\mathbf{F\bar{4}3m}$ with lattice parameter
$a=0.54505$ \nano\meter; Ga is located in the origin $(0,0,0)$, and P at $(1/4,1/4,1/4)$; both atoms have full occupancy $1.0$
and Debye-Waller factor can be set to the default value of $0.004$ \nano\meter$^{-2}$.  This value produces decent results, but 
if more accurate Debye-Waller factors for Ga and P are known, then those should be used instead.  The crystallographic parameters
can be entered into a data file using the \textsf{CTEMmkxtal} program; it would make sense to name this file \textsf{GaP.xtal}.

\subsection{Various dislocations in GaP}
Consider a region of interest with dimensions $1\times 1$ $\mu$m$^2$, in which there are four isolated $60^{\circ}$ dislocations,
four stacking faults (dissociated $60^{\circ}$ dislocations), and a 2D grid of $8$ $60^{\circ}$ misfit dislocations at a depth of $75$ nm,
parallel to the surface.  All the defect descriptor files can be found in the \textsf{Examples/ECCISimulations} folder. 

An ECCI image simulation is carried out using the general parameters described in \textsf{Ex5.1.nml}:
\fvset{frame=lines,formatcom=\color{blue},fontsize=\scriptsize,numbers=left}
\VerbatimInput{../examples/ECCISimulations/Ex5.1.nml}
and in the foil description file \textsf{FOIL5.1.nml}:
\fvset{frame=lines,formatcom=\color{blue},fontsize=\footnotesize,numbers=left}
\VerbatimInput{../examples/ECCISimulations/FOIL5.1.nml}

Note that these input files result in a very long execution time, around $14$ hours on a $2\times 3.06$ GHz 6-core Intel Xeon Mac Pro,
using $12$ threads.  This is due to the fact that the current implementation of the algorithm does not yet make use of the Bethe potential
approximation; the program determines which $\mathbf{g}$-vectors to take into account for the exact zone axis orientation, and then uses
those vectors for all other orientations as well, regardless of how strongly excited each reflection is.  In the next version of the program,
this will be modified, so that for each incident beam orientation, only the relevant strong beams will be taken into account.
For now, best thing to do is to start this program at the end of the day and the data file should be ready the next morning.  Remember to
make sure that your computer does not go to sleep during that time!  Obviously, you can change the parameters in the input file if you
want a shorter run time; for instance, you can put \textsf{dkt}$=1.0$ instead of $0.5$, which should reduce the execution time by a factor 
of $4$ or so.  It also helps to select the Laue center to be away from the zone axis; this typically reduces the number of contributing beams.

Next, we can execute the program by entering
\begin{verbatim}
	../exe/CTEMECCI Ex5.1.nml
\end{verbatim}
where we assume that the program is being run from a folder at the same level as the exe folder.  


Once the data file has been created, then it is a simple matter to use the IDL Virtual Machine application to 
visualize the result.   The IDL app is located in the VMapps folder of the distribution; simply double clicking the \textsf{ECCIDisplay}
application will start up the Virtual Machine; click on the \textsf{Click to Continue} option in the IDL start up screen to 
initialize the app; then follow the guidelines from section~\ref{sec:idl}.

\begin{figure}[h]
\leavevmode\centering
\epsfxsize=6in\epsffile{figs/ECCIdislo}
\caption{\label{fig:ECCIdislo}ECCI images of a set of dislocations near the $[001]$ zone axis orientation of GaP, $25$ kV, field of view is 
$1$ $\mu$m on the side.  These images were taken for incident beam directions a) $(0,2)$, b) $(4,-6)$, and c) $(6,-4)$.}
\end{figure}

\newcommand{\disloc}[6]{$\frac{1}{2}[#1\,#2\,#3]\vert [#4\,#5\,#6]$}

Fig.~\ref{fig:ECCIdislo} shows example images acquired at several grid positions.  The individual defects labeled in a) have the following
characteristics (dislocations are represented by the symbol $\mathbf{b}\vert\mathbf{u}$, with $\mathbf{b}$ the burgers vector and $\mathbf{u}$
the line direction):
{\small \begin{enumerate}
	\item \disloc{0}{1}{\bo}{1}{1}{0}
	\item \disloc{0}{1}{1}{1}{1}{0}
	\item \disloc{1}{0}{\bo}{1}{1}{0}
	\item \disloc{1}{0}{1}{1}{1}{0}
	\item \disloc{\bo}{0}{\bo}{\bo}{1}{0}
	\item \disloc{\bo}{0}{1}{\bo}{1}{0}
	\item \disloc{0}{1}{\bo}{\bo}{1}{0}
	\item \disloc{0}{1}{1}{\bo}{1}{0}

	\item \disloc{1}{0}{\bo}{0}{1}{\bo}
	\item \disloc{0}{1}{\bo}{\bo}{0}{\bo}
	\item \disloc{1}{0}{\bo}{0}{\bo}{\bo}
	\item \disloc{0}{1}{\bo}{1}{0}{\bo}
	
	\item \dislo{1}{\bt}{\bo}{\bo}{0}{\bo}, \dislo{\bo}{\bo}{\bt}{0}{\bo}{\bo} on $(11\bo)$
	\item \dislo{\bt}{\bo}{1}{0}{\bo}{\bo}, \dislo{\bo}{1}{2}{1}{0}{\bo} on $(\bo 1\bo)$
	\item \dislo{\bo}{2}{\bo}{1}{0}{\bo}, \dislo{1}{1}{\bt}{0}{1}{\bo} on $(\bo\bo\bo)$
	\item \dislo{2}{1}{1}{0}{1}{\bo}, \dislo{1}{\bo}{2}{\bo}{0}{\bo} on $(1\bo\bo)$
	\end{enumerate}}
Defects $1$--$8$ are misfit dislocations parallel to the sample surface (we are ignoring the cutoff geometry for the time being) at a 
depth of $75$ \nano\meter. Defects $9$--$12$ penetrate the surface, as do the stacking faults $13$--$16$, which have surface penetrating 
partials; in the list above, the leading partial is listed first.  Note that the line direction for surface penetrating dislocations 
must be oriented into the sample.  Note how for some beam directions, the misfit dislocations show up as double contrast lines.  Images (b) and (c) 
are taken at opposite sides of the $(1\bo 0)$ Kikuchi band, so they display a contrast reversal.


\subsection{Stacking fault pyramid in GaP}

We begin by defining a stacking fault pyramid that has its apex at the center of the image, and is characterized by the 
parameters shown in the table below.  The stacking fault pyramid consists of four stacking faults on $\{111\}$ planes, oriented such
that they form a pyramid with apex inside the foil, and the faults intersecting the surface in a square
shape.  The image simulations have the $[100]$ direction oriented horizontally, and $[010]$ vertically.
The stacking faults are numbered $1$ through $4$ in counter-clockwise fashion, starting in the upper 
right quadrant of the image.  The table below shows the essential parameters of each stacking fault, 
including the slip plane, the leading and trailing partial dislocation burgers vectors and line 
directions in the notation $\mathbf{b}\vert\mathbf{u}$, and the Lomer-Cottrell stair-rod dislocations
that form the line common to two consecutive stacking faults.  In the present case,  
the trailing dislocation of stacking fault $i$ merges with the leading dislocation
of fault $i+1$ to form the stair-rod edge dislocation. Other combinations of burgers vectors are possible.

\renewcommand{\arraystretch}{1.5}
\begin{table}[h]
\centering\leavevmode
\begin{tabular}{|c|c|l|l|l|l|}
\hline
SF\# & Plane & Perfect &Leading partial & Trailing partial & Lomer-Cottrell\\
\hline
\hline
$1$ & $(1\,1\,\bar{1})$ & \dislop{0}{\bo}{\bo}              & \dislo{\bo}{\bo}{\bt}{1}{0}{1} & \dislo{1}{\bt}{\bo}{0}{1}{1} & \dislo{0}{\bo}{1}{0}{1}{1}\\
$2$ & $(\bar{1}\,1\,\bar{1})$ & \dislop{\bo}{0}{1}          & \dislo{\bo}{1}{2}{0}{1}{1} & \dislo{\bt}{\bo}{1}{\bo}{0}{1} & \dislo{\bo}{0}{\bo}{\bo}{0}{1}\\
$3$ & $(\bar{1}\,\bar{1}\,\bar{1})$ & \dislop{0}{1}{\bo}    & \dislo{1}{1}{\bt}{\bo}{0}{1} & \dislo{\bo}{2}{\bo}{0}{\bo}{1} & \dislo{0}{1}{1}{0}{\bo}{1}\\
$4$ & $(1\,\bar{1}\,\bar{1})$ & \dislop{1}{0}{1}            & \dislo{1}{\bo}{2}{0}{\bo}{1} & \dislo{2}{1}{1}{1}{0}{1} & \dislo{1}{0}{\bo}{1}{0}{1}\\
\hline
\end{tabular}
\end{table}

The stacking faults are defined by namelist files similar to the following file:
\fvset{frame=lines,formatcom=\color{blue},fontsize=\footnotesize,numbers=left}
\VerbatimInput{GaPSF1.nml}
The position of the fault and the separation of the partials takes some experimenting to get right.  In the file above, the stacking fault mid points
are located such that the apex of the pyramid is $100$ \nano\meter\ below the surface, resulting in a pyramid base of edge length $141.42$ \nano\meter,
and mid point locations of $(\pm 50,\pm 50)$ \nano\meter.  Converting those to fractional cordinates for an image of edge length $256\times 256$ \nano\meter\
leads to the values $(\pm 0.390625,\pm 0.390625)$.

The input parameters for the ECCI computation are shown here (look for the files in the \textsf{Examples/ECCISimulations} folder):
\fvset{frame=lines,formatcom=\color{blue},fontsize=\footnotesize,numbers=left}
\VerbatimInput{../examples/ECCISimulations/Ex5.2.nml}
Note that in this run, we compute the images across one of the Kikuchi bands, centered at location $(5,7)$.
This run does not take as long as the one from the previous example; in fact, it only took about four minutes on the same platform.
The resulting image mosaic is shown in Fig.~\ref{fig:ECCISF}.  Note how the image contrast changes drastically normal to the 
Kikuchi band, but stays nearly constant along lines parallel to the band.  The contrast is also maximal, with maximum fringe spacing, 
near the edge of the band.
%
\begin{figure}[h]
\leavevmode\centering
\epsfxsize=6in\epsffile{figs/ECCISF}
\caption{\label{fig:ECCISF}ECCI image mosaic of stacking fault pyramid; images are $256\times 256$ \nano\squaren\meter, centered on the 
position $(5,7)$, on a grid with step size $0.5$.}
\end{figure}

%
%\newpage
%\begin{refsection}[papers]
%\renewbibmacro{in:}{}
%\nocite{degraef2003b,humphreys1979a,winkelmann2003a,winkelmann2008a,weickenmeier1991a,reimer1985a}%
%  \printbibliography[title={References}]
%\end{refsection}



\end{document}



