
\documentclass[DIV=calc, paper=letter, fontsize=11pt]{scrartcl}	 % A4 paper and 11pt font size

\usepackage[body={6.5in,9.0in},
  top=1.0in, left=1.0in]{geometry}
  
\usepackage[english]{babel} % English language/hyphenation
\usepackage[protrusion=true,expansion=true]{microtype} % Better typography
\usepackage{amsmath,amsfonts,amsthm} % Math packages
\usepackage[svgnames]{xcolor} % Enabling colors by their 'svgnames'
\usepackage[hang, small,labelfont=bf,up,textfont=it,up]{caption} % Custom captions under/above floats in tables or figures
\usepackage{booktabs} % Horizontal rules in tables
\usepackage{fix-cm}	 % Custom font sizes - used for the initial letter in the document
\usepackage{epsfig}
\usepackage{sectsty} % Enables custom section titles
\allsectionsfont{\usefont{OT1}{phv}{b}{n}} % Change the font of all section commands

\usepackage{fancyhdr} % Needed to define custom headers/footers
\pagestyle{fancy} % Enables the custom headers/footers
\usepackage{lastpage} % Used to determine the number of pages in the document (for "Page X of Total")
\usepackage{color}

\usepackage{fancyvrb}% used to include files verbatim
%\usepackage{chemsym}

\usepackage{hyperref}

\usepackage[backend=bibtex,style=numeric,sorting=ydnt,maxnames=15]{biblatex}
\renewbibmacro{in:}{}

% Count total number of entries in each refsection
\AtDataInput{%
  \csnumgdef{entrycount:\therefsection}{%
    \csuse{entrycount:\therefsection}+1}}

% Print the labelnumber as the total number of entries in the
% current refsection, minus the actual labelnumber, plus one
\DeclareFieldFormat{labelnumber}{\mkbibdesc{#1}}    
\newrobustcmd*{\mkbibdesc}[1]{%
  \number\numexpr\csuse{entrycount:\therefsection}+1-#1\relax}


%\addbibresource[label=papers]{mypubs.bib}
%\addbibresource[label=books]{mypubs.bib}
%\addbibresource[label=edited]{mypubs.bib}
%\addbibresource[label=chapters]{mypubs.bib}


% Headers - all currently empty
\lhead{}
\chead{}
\rhead{}

% Footers
\lfoot{\textsf{CTEMmbcbed}/\textsf{CTEMlacbed} manual, v1.0, \today}
\cfoot{}
\rfoot{\footnotesize Page \thepage\ of \pageref{LastPage}} % "Page 1 of 2"

\renewcommand{\headrulewidth}{0.0pt} % No header rule
\renewcommand{\footrulewidth}{0.4pt} % Thin footer rule

\usepackage{lettrine} % Package to accentuate the first letter of the text
\newcommand{\initial}[1]{ % Defines the command and style for the first letter
\lettrine[lines=3,lhang=0.3,nindent=0em]{
\color{DarkGoldenrod}
{\textsf{#1}}}{}}

\usepackage{titling} % Allows custom title configuration

\newcommand{\HorRule}{\color{DarkGoldenrod} \rule{\linewidth}{1pt}} % Defines the gold horizontal rule around the title

\pretitle{\vspace{-1.5in} \begin{center} \HorRule \fontsize{25}{25} \usefont{OT1}{phv}{b}{n} \color{DarkRed} \selectfont} % Horizontal rule before the title

\title{Convergent Beam Electron\\ Diffraction Simulations} % Your article title

\posttitle{\par\end{center}\vskip 0.5em} % Whitespace under the title

\preauthor{\begin{center}\large \lineskip 0.5em \usefont{OT1}{phv}{b}{sl} \color{DarkRed}} % Author font configuration

\author{\vspace*{-0.7in}} % Your name

\postauthor{\footnotesize \usefont{OT1}{phv}{m}{sl} \color{Black} % Configuration for the institution name

\par\end{center}\HorRule} % Horizontal rule after the title
\date{Program Manual, v1.0, \today}

\newcommand{\ctp}{\textsf{CTEMsoft-2013}}
%
\newcommand{\upg}[1]{\mathrm{i}U_{\mathbf{#1}}^{\prime}}
\newcommand{\combo}[1]{U_{\mathbf{#1}}+\upg{#1}}
\newcommand{\upgcombo}[2]{2k_{0}s_{\mathbf{#1}}+\upg{#2}}
\newcommand{\ugh}[2]{U_{\mathbf{#1}-\mathbf{#2}}}
\newcommand{\ughp}[2]{U_{\mathbf{#1}'-\mathbf{#2}}}
\newcommand{\ughpp}[2]{U_{\mathbf{#1}-\mathbf{#2}'}}
\newcommand{\kkg}[1]{k_{0}^{2}-(\mathbf{k}+\mathbf{#1})^{2}}
\newcommand{\Cg}[1]{C_{\mathbf{#1}}}
\newcommand{\Cgj}[2]{C_{\mathbf{#1}}^{(#2)}}
\newcommand{\Cgjp}[2]{C_{\mathbf{#1}'}^{(#2)}}
\newcommand{\Cgja}[2]{C_{\mathbf{#1}}^{(#2)\ast}}
\newcommand{\button}[1]{\colorbox{green}{\textsf{#1}} button}


\begin{document}
\maketitle

\begin{figure*}[h]
\leavevmode\centering
\epsffile{figs/CTEMlogo}
\end{figure*}

\renewcommand{\contentsname}{Table of Contents}
{\small\tableofcontents}

\newpage
\section{Introduction}
This manual describes a set of three programs, 
two written in Fortran-90,\footnote{f90 is a much richer language than the original fortran-f77, and is
used for all programs in the \ctp\ package.} 
the other in IDL,\footnote{The \textit{Interactive Data Language} is an interpreted scripting language with extensive graphics capabilities.} 
that can be used for the simulation of regular zone axis convergent beam electron diffraction patterns (CBED) as 
well as large angle CBED (LACBED) patterns.  The main f90 programs
are named \textsf{CTEMmbcbed} and \textsf{CTEMlacbed} (note that all programs in the \ctp\ package start with the letters ``CTEM'').
The output generated by either program can then be visualized by the IDL routine \textsf{CBEDDisplay.pro}.  

On the following pages we will try to accomplish four tasks:
\begin{enumerate}
	\item Explain briefly the underlying image formation theory and the numerical approach followed by the f90 programs (section~\ref{sec:theory});
	\item Document the various input files for the f90 programs (sections~\ref{sec:f90lacbed} and \ref{sec:f90}));
	\item Document the IDL interface (section~\ref{sec:idl});
	\item Explain the use of these programs by means of a few basic examples (section~\ref{sec:examples}).
\end{enumerate}

At the time of writing of this manual, these programs have been successfully compiled on the Mac OS X platform using the commercial \textit{Absoft} compiler,
as well as the public domain gfortran compiler.  There is no basic reason why this code should not work on Windows and Linux systems, and efforts are underway
to make sure that the code will run on all platforms. It would be interesting to see some of this code
implemented in a super computer setting, since many of the routines should be quite parallellizable.  Where possible, some of the computations
are carried out using the OpenMP directives, so that multiple cores can be used.  This aspect of the code is still under further development,
and we anticipate that GPUs will become useful as well, using the OpenCL approach.

The \ctp\ package is entirely written in f90 and does not use any of the newer commands available in the 95 and 2013 versions.
The source code is extensively commented, using regular comment lines, but also using DOxygen documentation generation commands.  Hence
there exists an extensive on-line documentation of all variables, variable types, modules, subroutines, functions, etc. for the latest version 
of the code.  For selected programs, more extensive manual pages are available.  If interested, please contact the author for further information.

The visualization part of the code consists of a series of  IDL routines that are available as source code or in the form of a Virtual Machine application. 
If you have an IDL license, then you can compile and run the IDL source code; alternatively, if you do not have a license,
then you can use one of the VM apps to perform the same task.  Note that in the VM environment, you will not be able to alter/compile the 
source code.

\newpage
\section{Convergent beam electron diffraction: the underlying theory\label{sec:theory}}
The programs described in this manual implement a dynamical model for electron diffraction using the 
Bloch wave approach.  In this section, we describe briefly the Bloch wave approach as well as the 
use of Bethe potentials to reduce the size of the dynamical matrix.  The combination of these 
two approaches is used in both of the f90 programs described in this manual.

\subsection{The Bloch wave model}
We start from the basic Schr{\"o}dinger equation for high energy electrons:
\begin{equation}
	\Delta\Psi+4\pi^2k_0^2\Psi=
	-4\pi^2\left[U+\mathrm{i}U^{\prime}\right]\Psi,
	\label{eq:start}
\end{equation}
where the first term represents the kinetic energy, the second term the total energy (corrected for 
refraction), and the right hand side represents the interaction with the crystal lattice; $k_0$ represents
the refraction corrected relativistic wave number, $U$ is the position-dependent electrostatic lattice 
potential (scaled by $2me/h^2$), and $U'$ is the corresponding absorptive potential; the combination of the 
two is often refered to as the ``optical potential.''  For all computations in the \ctp\ package, we make use 
of the Weickenmeier \& Kohl atomic scattering factors and absorptive form factors.\footnote{A.~Weickenmeier and H.~Kohl,
Computation of {A}bsorptive {F}orm {F}actors for {H}igh-{E}nergy {E}lectron {D}iffraction,{\em Acta Crystall.\ A}, 47:590--597, 1991.}

The Bloch wave approach\footnote{For a basic introduction, see C.J. Humphreys,
The {S}cattering of {F}ast {E}lectrons by {C}rystals, {\em Rep.\ Prog.\ Phys.}, 42:1825--1887, 1979.}
starts from the assumption that the wave function inside the crystal
is a superposition of states that have the periodicity of the lattice, but does not make any
assumption about the allowed wave vectors inside the crystal.  The wave function is then 
assumed to be of the form 
\begin{equation}
	\Psi(\mathbf{r})=\sum_j\alpha^{(j)}\sum_{\mathbf{g}}\Cgj{g}{j} 
	\mathrm{e}^{2\pi \mathrm{i}(\mathbf{k}^{(j)} +\mathbf{g})\cdot\mathbf{r}}=
        \sum_j\alpha^{(j)}C^{(j)}(\mathbf{r})\mathrm{e}^{2\pi \mathrm{i}
        \mathbf{k}^{(j)}\cdot\mathbf{r}}
	\label{eq:blochexpansion2}
\end{equation}
The coefficients
$\Cgj{g}{j}$ are known as the \textit{Bloch wave coefficients}, 
while the coefficients $\alpha^{(j)}$ are the \textit{excitation amplitudes}.
Each of the functions $C^{(j)}(\mathbf{r})$ has the periodicity of the lattice.
Using the high energy approximation one can show that the number of coefficients is
equal to the number of scattered beams.  

When absorption is taken into account, one can show that the problem is reduced
to the determination of the eigenvalues and eigenvectors of the dynamical matrix below:
\begin{equation}
	\left(\begin{matrix}
	\upg{0} & \combo{-g} & \ldots & \combo{-h}\\
	\combo{g} & \upgcombo{g}{0} & \ldots & \combo{g-h}\\
	\vdots  &  \vdots &  \ddots  &  \vdots\\
	\combo{h} & \combo{h-g} & \ldots & \upgcombo{h}{0}
	\end{matrix}\right)
	\left(\begin{matrix}
	\Cgj{0}{j}\\ \Cgj{g}{j}\\ \vdots\\ \Cgj{h}{j}
	\end{matrix}\right)=2k_{n}\Gamma^{(j)}\left(\begin{matrix}
	\Cgj{0}{j}\\ \Cgj{g}{j}\\ \vdots\\ \Cgj{h}{j}
	\end{matrix}\right).\label{eq:blochmatrixabs}
\end{equation}
From a computational point of view, this is about the most difficult eigenvalue problem to solve, since 
the matrix is non-symmetric and complex.  In the \ctp\ package, we make use of a standard routine (\textsf{zgeev}) from the LAPACK
library to solve this problem.

Once the eigenvectors and eigenvalues are known, we need to apply the boundary condition at the 
entrance plane (typically that all amplitude is in the incident beam and none in the scattered beams at $z=0$).
One can show that the Bloch wave excitation amplitudes $\alpha^{(j)}$ are given by the first column of the 
inverse of the eigenvector matrix; we use the \textsf{zgetrf} and \textsf{zgetri} LAPACK routines to compute 
this inverse matrix.

Finally, the diffracted intensities are computed by substituting the eigenvectors, eigenvalues and excitation
amplitudes in equation~(\ref{eq:blochexpansion2}) and taking the modulus squared for each of the diffracted beams.
This computation is then repeated for each of the incident beam directions, and the results are collected in 
diffraction disks and displayed as a CBED or LACBED pattern.

So, the basic computation is not all that complicated.  What makes the \ctp\ programs complicated is the fact 
that they have been written for general crystal symmetry, and also the fact that the basic computational step
must also function for the computation of several other patterns (electron channeling pattern, electron Kossel pattern, EBSD, $\ldots$).
Furthermore, the programs also allow for a tilted incident beam, and a host of other experimental conditions.
One important complicating factor is the implementation of Bethe potentials (see next section), which  
necessitates a significant amount of bookkeeping.

\subsection{Bethe potentials to reduce the dynamical matrix size}
The dynamical matrix can become quite large, in particular when the incident beam is close to a zone
axis orientation.  The computation of eigenvalues and eigen vectors goes as $N^3$, where $N$ is the number
of beams taken into account, so even a small reduction of $N$ can have a significant effect on the 
computation time.  The standard approach is to split the scattered beams into two groups: strong
beams, for which the computation is done explicitly, and weak beams, which are treated as a 
perturbation on the strong beam dynamical matrix.

If we represent the optical potential by $U$, then one can show that substitution of the 
Bloch wave expression into the Schr{\"o}dinger equation, and application 
of the high energy approximation, results in the following system of equations:  
\[
	2k_{0}s_{\mathbf{g}}\Cgj{g}{j} + \sum_{\mathbf{h}\neq
	\mathbf{g}}\ugh{g}{h}\Cgj{h}{j}=2k_{n}\Gamma^{(j)}\Cgj{g}{j}.
\]
where $s_{\mathbf{g}}$ are the excitation errors, and $k_n$ is the normal 
component of the incident wave vector.
If the prefactor of the first term $(k_{0}\vert s_{\mathbf{g}}\vert)$
is significantly larger than the prefactor of the second term
$(\vert\ugh{g}{h}\vert)$, then the coupling of $\mathbf{g}$ to
$\mathbf{h}$ will be weak.  Let us rewrite this eigenvalue equation as
follows:
\[
\eta^{(j)}_{\mathbf{g}}\Cgj{g}{j} + \sum_{\mathbf{h}\neq
	\mathbf{g}}\ugh{g}{h}\Cgj{h}{j}= 0,
\text{ with }
\eta^{(j)}_{\mathbf{g}}\equiv 2[k_{0}s_{\mathbf{g}} - k_{n}\Gamma^{(j)}].
\]
We will denote weak beams by primed reciprocal lattice vectors, e.g., 
$\mathbf{g}'$. We will also assume that weak beams do not interact 
with each other, only with strong beams.  After a bit of mathematics,
one can show that to first order, the above system of equations can
be replaced by a new, smaller system where the sum only covers the strong beams:
\[
\bar{\eta}^{(j)}_{\mathbf{g}}\Cgj{g}{j} + \sum_{\mathbf{h}\neq
	\mathbf{g}}\bar{U}_{\mathbf{g}-\mathbf{h}}\Cgj{h}{j}= 0,
\]
The perturbed excitation errors and potential 
coefficients can be written as:
\begin{align*}
    2k_0\bar{s}_{\mathbf{g}} &\equiv 2k_0s_{\mathbf{g}}- 
    \sum_{\mathbf{h}'}\frac{\vert\ughpp{g}{h}\vert^{2}} 
    {2k_0s_{\mathbf{h}'}};\\
    \bar{U}_{\mathbf{g}-\mathbf{h}} &\equiv \ugh{g}{h}
    -\sum_{\mathbf{h}'}\frac{\ughpp{g}{h}\ughp{h}{h}}
    {2k_0s_{\mathbf{h}'}},
\end{align*}
We have replaced the excitation errors of the strong beams by effective
excitation errors; the Fourier coefficients of the electrostatic
lattice potential have also been modified to take the weaker beams
into account.  The coefficients $\bar{U}_{\mathbf{g}-\mathbf{h}}$ are
known as \textit{Bethe potentials} or \textit{dynamical potentials}. 
This approximation was first introduced by Bethe in 1928 using first order
perturbation theory.

It is not unusual for the Bethe potential approximation to allow for a 
reduction of the number of beams by a factor of $2$, thus speeding up
the computation of the eigenvalues and eigenvectors by a factor of $8$.
Once the Bloch wave coefficients of all the strong beams are known, then 
the following equation can be used to determine the Bloch wave coefficients of the weaker 
beams (and hence also their intensity):
\begin{equation}
    \Cgjp{g}{j} = -\frac{1}{\eta^{(j)}_{\mathbf{g}'}}
    \sum_{\mathbf{h}}\ughp{g}{h}\Cgj{h}{j}.\label{eq:weakCgj}
\end{equation}
While the Bethe potential approach is conceptually not that difficult, it is clear that efficient implementation 
of this approach in a framework that functions properly for arbitrary crystal symmetry
requires a bit of sophistication in the algorithm.

The Bethe potential criteria can be implemented in a number of different ways, but 
require the definition of one or more threshold levels.  Currently,
the program user has some control over the threshold between strong and weak beams. 
As will be described in more detail in section~\ref{sec:f90BetheParameters}, there are two user-adjustable 
threshold levels, called ``\textsf{cutoff}'' and ``\textsf{weakcutoff}''; if we represent
them by $c_1$ and $c_2$, respectively, then the following relations hold for a beam 
with potential coefficient $\vert U_{\mathbf{g}}\vert$ and excitation error $s_{\mathbf{g}}$:
\begin{align*}
	0 & \le  \vert s_{\mathbf{g}}\vert \le c_2\lambda\vert U_{\mathbf{g}}\vert\rightarrow \text{ $\mathbf{g}$ is a strong beam};\\
	c_2\lambda\vert U_{\mathbf{g}}\vert &\le \vert s_{\mathbf{g}}\vert \le c_1\lambda\vert U_{\mathbf{g}}\vert\rightarrow \text{ $\mathbf{g}$ is a weak beam};\\
	c_1\lambda\vert U_{\mathbf{g}}\vert &\le \vert s_{\mathbf{g}}\vert \rightarrow\text{ $\mathbf{g}$ can be ignored}.
\end{align*}
This is a simplistic criterion for distinguishing between weak and strong beams.  While the results of computations with this criterion
are quite good (see section~\ref{sec:examples}), one could argue that a better criterion (but a bit more difficult to implement) 
would be the following:  above it was stated that the coupling between two beams $\mathbf{g}$ and $\mathbf{h}$ is considered 
to be weak, if $k_0\vert s_{\mathbf{g}}\vert \gg \vert U_{\mathbf{g}-\mathbf{h}}\vert$.  So, in order for a beam to be weak, this relation
has to be satisfied for all beams $\mathbf{g}$.  Therefore, for each potentially weak beam, we need to compare the corresponding 
potential coefficients for all the other beams to their respective excitation errors.  This alternative thresholding approach will undergo
some testing before implementation (if it is deemed to produce better results in terms of computation time).  

As a final comment in this section we might state that it is possible to extend the Bethe potential approximation to the next perturbation
order; preliminary numerical tests have shown that this results in an additional computational gain, at the expense of more complicated
code.  In the current version of the source code, the first order Bethe potential approximation is used.

\subsection{Using crystal and diffraction point group symmetry}
As described in the classic article by Buxton, Eades, Steeds, and Rackham (BESR),\footnote{
``The symmetry of electron diffraction zone axis patterns,'' Phil.\ Trans.\ R. Soc, vol.\ 281, pp.\
171-194 (1976).} the CBED diffraction process follows a rather strict series of symmetry rules.
In the \ctp\ package, we have attempted to implement those rules as much as possible.  Effectively,
this means that the CBED programs ``know'' both crystal and diffraction point group symmetry, and
attempt to use it as much as possible.  For instance, the output of the \textsf{CTEMlacbed} program is a 
large file with multiple diffraction disks in it, one disk for each of the independent families 
of disks.  This is not as simple as it appears to be at first sight;  let us consider an example, 
the $[112]$ zone axis pattern for copper.

All the diffraction programs in the \ctp\ package orient the crystal lattice in a default orientation 
for all diffraction patterns.  This orientation is determined from the zone axis direction by first 
determining the shortest two reciprocal lattice vectors normal to the zone axis (regardless of whether or not they have a non-zero structure factor), and then selecting
the shortest one of those to be along the horizontal direction in the diffraction pattern.  For the 
$[112]$ zone axis, the shortest reciprocal lattice vectors are $\mathbf{g}_{1\bar{1}0}$ and 
$\mathbf{g}_{11\bar{1}}$.  According to the diffraction group tables in BESR, the $[112]$ zone axis 
in the point group $\mathbf{m\bar{3}m}$ has $\mathbf{2}_R\mathbf{mm}_R$ as its diffraction group (Table 4 in BESR, reproduced in
section~\ref{sec:examples}).  As a 
consequence, the Bright Field symmetry is $\mathbf{m}$, the Whole Pattern symmetry is $\mathbf{m}$, the special
Dark Field symmetry is $\mathbf{m}$, and the general Dark Field symmetry is $\mathbf{1}$; note that these are
all 2D point group symmetries, not 3D!  

There are two important observations to be made for this example:  
\begin{enumerate}
	\item The standard orientation of the 2D crystallographic point group $\mathbf{m}$ has the mirror plane in the horizontal 
	orientation when drawn on a stereographic projection (see Table 10.2.1 in the International Tables for Crystallography, Volume A).
	This is the same orientation as the shortest reciprocal lattice vector $\mathbf{g}_{1\bar{1}0}$.  However, we know that this 
	reciprocal lattice vector must be normal to a mirror plane, not parallel to it, so that we need to rotate the 2D point group
	$\mathbf{m}$ by $90^{\circ}$ to bring it in the correct orientation relative to the projected zone axis pattern.  Such issues
	arise for the 2D point groups $\mathbf{m}$, $\mathbf{31m}$ and $\mathbf{3m1}$; the CBED programs properly take 
	these special cases into account.

	\item Since the Whole Pattern (WP) symmetry for the $[112]$ zone axis in copper is $\mathbf{m}$, this means that,
	from a diffraction group point of view, the reflections $\mathbf{g}_{11\bar{1}}$ and $\mathbf{g}_{\bar{1}\bar{1}1}$ are
	no longer equivalent, since they do not belong to the same family with respect to the 2D symmetry group $\mathbf{m}$.
	Hence, when the CBED programs determine families of equivalent reflections, this operation is performed with respect
	to the 2D symmetry groups, not the 3D crystallographic point groups.  This should be kept in mind when analyzing the lists 
	of reflections for each of the zero and higher order Laue zones.
\end{enumerate}

While the implementation of the diffraction group symmetry was extensively tested on a number of crystal structures and 
zone axis directions, it is rather complicated from a programming point of view, so it is possible that errors remain; if an error is suspected, please contact the 
author and report the crystal structure, accelerating voltage, zone axis, and beam convergence angle, as well as a brief description of the 
error.  Make sure that you properly understand the diffraction groups and the use of the symmetry tables before complaining 
about a potential error$\ldots$.



\newpage
\section{The \protect\textsf{CTEMlacbed.f90} program\label{sec:f90lacbed}}

\subsection{Program overview\label{sec:f90overviewlacbed}}
This is a general program for large angle CBED computations.  The main output of 
this program is a file containing the complete diffraction disks for all relevant
reflections, out to some large beam convergence angle.  The diffraction disks are stored
individually, along with relevant information for each reflection.  This allows one to
then interactively compute low resolution regular CBED patterns for a range of 
microscope and sample tilt conditions (using the IDL visualization program).

The program makes use of symmetry (diffraction groups) when appropriate to speed 
things up a bit.  The main computation proceeds as follows:
\begin{itemize}
	\item read input parameters;
	\item determine, for the given zone axis direction and beam convergence angle, 
	which reflections could potentially contribute to the scattering process;
	\item for those reflections, pre-compute all the potential Fourier coefficients,
	as well as an array that contains one unique set of Miller indices (unique with respect to the 
	Whole Pattern symmetry) for each 
	contributing family of $\mathbf{g}_{hkl}$;
	\item loop over all beam directions in the incident cone:
	\begin{itemize}
		\item determine the dynamical matrix, using the Bethe potential method (i.e.,
		for all participating beams, split them into strong and weak beams);
		\item solve the eigenvalue problem to obtain all intensities (for a range of 
		sample thicknesses);
		\item extract from the intensity array those values that correspond to the 
		pre-selected set of unique planes $(hkl)$;
	\end{itemize}
	\item store all diffraction disks and associated data in a binary file for visualization in IDL.
\end{itemize}
This program can take a long time to run, depending on the complexity of the crystal structure,
as well as on the beam convergence angle and requested disk size (see next section for parameter 
definitions).  It would make sense to speed things up a bit by using OpenMP calls, but that has not yet
been implemented (as of \today).

The program will automatically determine which independent families of reflections need to be considered
in the output.  This process (known internally as ``pruning the reflection list'') can be a bit time-consuming
since it involves running through the list of potential reflections for each of the incident beam directions.
The time it takes will increase with decreasing \textsf{dmin}, increasing convergence angle $\theta_c$, and increasing \textsf{npix} 
(see section~\ref{sec:f90inputlacbed} for the definition of these input parameters).
It is probably not easy to speed this up significantly, and it is unlikely that the pruning algorithm is a 
candidate for parallel implementation without rewriting significant portions of the algorithm.\footnote{In other words,
be patient, until I have a chance to come up with a better algorithm.}

As part of the screen output, the program lists the diffraction group symmetries that are relevant 
for the current crystals structure and zone axis orientation.  This is done by means of a complete 
implementation of the relevant tables in BESR.  The relevant symmetries are:  Bright Field, Whole Pattern, and Special and General
Dark Field.  The computed diffraction disks should display these symmetries (Whole Pattern only for the
\textsf{CTEMmbcbed} program in zone axis orientation, not for \textsf{CTEMlacbed} which display the proper Bright Field
and Special or General Dark Field symmetries for individual diffraction disks).

It should be noted that the output file for this program can become very large.  For an \textsf{npix} parameter 
of $512$ pixels (resulting in an image of size $1025\times 1025$), $10$ sample thickness values,
and $162$ independent reflection families (it is quite easy to get this many, even for a small unit cell),
the output file has a size of about $6.8$ Gb.  One can restrict the file size a little by specifying how  
many HOLZ layers are used in the output; in its current implementation, the IDL display program will only 
use up to three HOLZ layers in addition to the ZOLZ layer.  One can also reduce the input parameter \textsf{dmin} (see next section),
which controls the smallest $d$-spacing taking into account in the dynamical matrix, but this does affect the accuracy 
of the computation and may give rise to missing HOLZ lines.  Finally, one can vary the threshold parameters for the 
Bethe potential approximation to some extent without significantly sacrificing the accuracy of the result.

\subsection{Namelist input files\label{sec:f90inputlacbed}}
The input file for this program, \textsf{CTEMlacbed.nml}, is formatted as follows:
\fvset{frame=lines,formatcom=\color{blue},fontsize=\footnotesize}
\VerbatimInput{../templatefolder/CTEMlacbed.template}
Most of these entries are self-explanatory.  The program will not run
without a valid crystal structure file name (xtalname).  The output
file can become very large; its size is proportional to the product 
of (2npix+1)$^2$, numthick, and the total number of independent families
that contribute to the diffraction process. This latter number depends 
on the crystal structure (lower symmetry and/or larger unit cell will produce 
more families for a given value of dmin), and also on the beam 
convergence angle (the larger the angle, the larger the region of reciprocal 
space that can produce scattered intensity).  It is not unusual for the 
program to produce files that are several Gb large.  For representative 
examples of the input file, see the examples in section~\ref{sec:examples}.

\subsection{Bethe parameters namelist file\label{sec:f90BetheParameters}}
The Bethe potential approximation used in this program can be adjusted by the
user via the \textsf{BetheParameters.nml} namelist file, which must be present 
in the folder in which the program is executed; if this file is not present, then the
program will assume default values for all three parameters.  The values shown below
are the current program defaults.
\fvset{frame=lines,formatcom=\color{blue},fontsize=\footnotesize}
\VerbatimInput{../templatefolder/BetheParameters.template}
The last parameter sets the excitation error cutoff level for double diffraction reflections.  The issue with
such reflections is that they have a zero structure factor (and hence $U_{\mathbf{g}} = 0$), so that 
the standard Bethe potential threshold approach can not be used at all.  The current implementation
allows the user to set a threshold for the excitation error (in nm$^{-1}$) above which the double diffraction reflection
will not contribute at all.  Recall that double diffraction reflections can only occur in so-called non-symmorphic 
space groups, i.e., in space groups that have at least one screw axis or glide plane symbol in their space 
group symbol; in other words, in the space group \textbf{Fm$\bar{\mathbf{3}}$m}, there are no glides or screws in
the symbol, so double diffraction reflections are not possible for \textit{any} zone axis orientation.  For the space group \textbf{P6$_{\mathbf{3}}$/mmm},
on the other hand, there is a $6_3$ screw axis in the space group symbol, therefore it is possible for certain
zone axes to display double diffraction reflections.  The CBED programs automatically take these reflections into account, but the user must
set the excitation error threshold value, or accept the program default.

It is also important to note that the value of the \textsf{weakcutoff} parameter should always be less that
that of the \textsf{cutoff} parameter if one wants to make use of the Bethe approximation.  Testing has shown
that a \textsf{weakcutoff} value slightly larger than half the \textsf{cutoff} value produces reasonable 
results.  To avoid using Bethe potentials, the two parameters should be put equal to each other.

\section{The \protect\textsf{CTEMmbcbed.f90} program\label{sec:f90}}

\subsection{Program overview\label{sec:f90overview}}
This program uses much of the same code as the LACBED program, but produces CBED patterns 
instead of individual diffraction disk images.  The main thinking here is that one would 
use the LACBED program to compute a much larger range of convergence angles than needed 
for CBED, and then one can experiment in the IDL visualization program to determine what the best 
microscope parameters would be to get a good CBED pattern (which will be a low resolution
pattern based on the LACBED data).  Once the parameters are known, one can then run the 
MBCBED program to obtain a higher resolution pattern.  For zone axis orientations, the 
program will make full use of the diffraction symmetry to reduce the number of beam directions
for which the computation has to be carried out.

\subsection{CTEMmbcbed.nml\label{sec:f90input1}}
The \textsf{CTEMmbcbed.nml} input file has the following entries:
\fvset{frame=lines,formatcom=\color{blue},fontsize=\footnotesize}
\VerbatimInput{../templatefolder/CTEMmbcbed.template}
Note that the camera length parameter \textsf{camlen} is simply a scale parameter
for the CBED pattern, and does not reflect a real camera length.  The computation is
always carried out so that for each pattern pixel inside the Bright Field disk, there 
will be one single incident beam direction.  Hence, increasing the camera length will
magnify the pattern, but will also increase the computation time.  Note that when 
the camera length is reduced, the HOLZ lines will become visible in the pattern, and the 
computation time will generally be relatively short.  The IDL visualization program can be used to generate 
an input namelist file for the \textsf{CTEMmbcbed} program; see section~\ref{sec:examples} for simulation
examples.

\section{The \protect\textsf{CBEDDisplay.pro} program\label{sec:idl}}


\subsection{Program overview\label{sec:idloverview}}
The data files generated by the \textsf{CTEMmbcbed} and \textsf{CTEMlacbed} programs consist of geometrical information and arrays of diffraction 
disks or CBED patterns.\footnote{In a later version of the \textsf{CTEMlacbed} program, it will be possible to store, for each incident beam direction, 
the excitation amplitudes, Bloch wave coefficients, and eigenvalues of each participating reflection; 
that data file will become quite large in size, easily multiple gigabytes.
The main reason for storing this data is that it then becomes possible to interactively play with a series of detector parameters,
in particular the convergence angle and the LACBED disk selection.  In the current version of the program, the data file contains
pre-computed diffraction disks, and only limited parameter modifications are possible.}  These files can be visualized using the IDL program
\textsf{CBEDDisplay.pro}, described in this section.

The \textsf{CBEDDisplay.pro} routine requires an IDL program license or can be executed using the Virtual Machine, without the need for a license. 
If you do have a license, then make sure that the folder containing the routines is part of your IDL pathname,
and that IDL is properly installed for the UNIX shell that you will be using (csh, bsh, etc...).
To execute the program, first start an IDL session in a terminal window (using the /Appplications/Utilities/Terminal program), 
start IDL, compile the \textsf{CBEDdisplay.pro} routine by typing:
\begin{verbatim}
	IDL> .r CBEDDisplay <return>
\end{verbatim}
at the IDL prompt and hitting return, followed by typing 
\begin{verbatim}
	IDL> CBEDDisplay <return>
\end{verbatim}
to start the display program.  The main program widget window will appear as well as a file selector window.  
Details of all the windows are described in the following subsections.

In addition to requiring an IDL license, the program expects the X-windows environment to be installed.  On the Mac, this corresponds
to the \textsf{XQuartz} program which can be downloaded from the open source site \textsf{http://xquartz.macosforge.org/}.  This program
must be installed in the system /Applications/Utilities folder and requires OS X 10.6 or later.

Useful thing to know: if the \textsf{CBEDDisplay.pro}  program hangs for some reason, you can reset your IDL session by typing
\begin{verbatim}
	IDL> .reset <return>
\end{verbatim}
This will destroy all program widgets and reset IDL to its original state.  If that does not work, then you may have to ``force quit'' the Terminal program
using the standard Esc-Option-Command key stroke.


\subsection{Main window\label{sec:idlmain}}
The main window contains most of the parameters that can be set by the user.  When the program first starts, a program preferences file 
will be read (see section~\ref{sec:idlpref}), and several internal parameters will be initialized to the values they had the last time the program
was used.  The main display widget when the program is started is shown in Fig.~\ref{fig:widget1}.

\begin{figure}[h]
\leavevmode\centering
\epsffile{figs/CBEDwidget1}
\caption{\label{fig:widget1}Main \textsf{CBEDDisplay.pro} program widget.}
\end{figure}

The main window consists of a number of information blocks; starting from the top we have:
\begin{itemize}
	\item block 1: contains information about the data file that is currently loaded in memory: the file name and size, and the dimensions of the 
	individual diffraction disks or CBED patterns in that file.  The black stripe is a file loading progress bar, since some files can take a while to load.
	\item block 2: various pieces of information generated by the \textsf{CTEMlacbed} or \textsf{CTEMmbcbed} programs are displayed here: number 
	of independent families of reflections in the data file (\textsf{CTEMlacbed} only); the beam convergence angle [mrad]; the electron wave length [pm];
	the maximum HOLZ layer contained in the file (\textsf{CTEMlacbed} only); the foil normal indices; the number of $\mathbf{k}$ vectors considered 
	in the incident illumination cone; the crystal structure input file; the zone axis indices; the Miller indices of the reciprocal lattice vector that is horizontal
	and pointing towards the right in the CBED patterns; and the intensity cutoff value (\textsf{CTEMlacbed} only; diffraction disks that have a maximum
	intensity below this cutoff value are not included in the data file).
	\item block 3: this block contains all the symmetry information for the current data file: the crystal point group and corresponding Laue point group;
	the zone axis diffraction group and corresponding projection diffraction group; the Whole Pattern (WP) and Bright Field (BF) 2D symmetry groups;
	and the Special and General Dark Field 2D point groups.
	\item block 4: below the trio of buttons described below, one finds the program information window, where various pieces of information will be 
	displayed when the program options are being used.  One can save all this information to a file by setting the LogFile button at the bottom right
	to \textsf{On}; this will generate a log file that contains the same information that appears in the information window.  The LogFile must be 
	turned on \textit{before} starting any activities in the program, and can be turned off at any moment.  The log file will have the 
	date and time as part of its file name, as in \textsf{CBEDDisplaySunOct20\_2013\_13:09:19.log}, and will be placed in the folder that the program is called from.
\end{itemize}
At the base of the widget there are three buttons:
\begin{itemize}
	\item \button{Quit}: this is obviously the main program termination button; clicking on this button will cause the program to write its
	current preferences file, close all other open widgets, and finally close itself.
	\item \button{Load LACBED File}: this option will bring up a file load dialogue window.  Use the navigation panels to locate a data file (generated
	by the \textsf{CTEMlacbed} program) that 	you wish to open and then click the \button{Open} button.  Once the file has been loaded, all the 
	fields in blocks $1$--$3$ will be filled with the proper values, and the \button{LACBED Window} and \button{CBED Window}  will be 
	activated (between blocks 3 and 4).
	\item \button{Load MBCBED File}: this option will bring up a file load dialogue window.  Use the navigation panels to locate a data file (generated
	by the \textsf{CTEMmbcbed} program) that you wish to open and then click the \button{Open}.  Once the file has been loaded, all of the relevant
	fields in blocks $1$--$3$ will be filled with the proper values, and the \button{MBCBED Window}  will be activated (between blocks 3 and 4).
\end{itemize}




\subsection{LACBED window\label{sec:idllacbed}}
This widget can be used to display individual diffraction disks from the LACBED input file.  In this and the following three sections, the screen shots
will be taken from the second example in section~\ref{sec:example2}.  After an input file has been loaded, both the \button{LACBED Window} and \button{CBED Window}
will be highlighted.  Starting with the \button{LACBED Window}, clicking it will produce two new widgets, shown in Fig.~\ref{fig:widget2}, 
which may be repositioned as needed. The LACBED Pattern
Widget has two graphics windows with a size of $513\times 513$ pixels each.  Below each window are fields indicating the minimum and maximum intensity
in each window; the Dark Field/Eades window has an additional field with the Miller indices of the reflection/family currently selected.  There are 
no editable fields in this widget.  Note that scroll bars will appear in each of the draw areas if the size of the disk image is larger 
than $513\times 513$.  


\begin{figure}[t]
\leavevmode\centering
\epsffile{figs/CBEDwidget2}
\caption{\label{fig:widget2}\textsf{LACBED} widget and the corrsponding draw widget.}
\end{figure}

The LACBED Widget contains a number of user adjustable fields and buttons; they are, from top to bottom:
\begin{itemize}
	\item \textsf{Dark Field Display Mode}: there are three options here: 
	\begin{itemize}
		\item \textsf{Single DF}: this option will produce a dark field disk from the selected reflection;
		\item \textsf{Symmetrized DF}: if the selected reflection has a multiplicity larger than $1$, then 
		the dark field window will display the superposition of all the family members, applying the Whole
		Pattern symmetry to the Single DF image.
		\item \textsf{Eades}: this mode produces a so called Eades pattern, in which all symmetrized DF images
		within a given angular range (see below) are added together.
	\end{itemize}
	\item \textsf{Foil Thickness [nm]}: clicking on this button will produce a list of all available foil thicknesses; selecting 
	one of them will then display that value and use if for visualizations.
	\item \text{ZOLZ and HOLZ lists}: the list of ZOLZ reflections is shown first, with for each reflection the Miller 
	indices of the representative family member, the scattering angle $2\theta$ in mrad, and the multiplicity of this 
	reflection with respect to the 2D Whole Pattern symmetry group.  In the HOLZ boxes, similar lists are shown for the first 
	and higher order Laue zones.  The current program limit is set to a maximum of $3$ HOLZ lists, but if the data file contains 
	fewer (via the \textsf{maxHOLZ} entry) then only the ones present will be shown.  If a HOLZ layer does not have any
	reflections in it, the list will still appear but will be empty.
	\item \textsf{Eades angular range [mrad]}: There are two editable text fields in which the user can enter the inner and 
	outer radii of the HAADF detector; note that the outer radius should be larger than the inner radius, so it should be set first.
	\item \textsf{Disk rotation angle [CW, deg]}: if necessary, each of the disk patterns can be rotated by a clock-wise angle in
	degrees; this can be useful when comparing simulated patterns with experimental ones.
	\item \textsf{Intensity Scaling}: because to the large dynamic range of intensities in CBED patterns, it makes sense to allow the user
	to set the intensity scaling mode to linear or logarithmic (base-10).  The text field at the end of this line contains an offset
	factor that is added to each of the patterns before computation of the logarithm (to avoid taking the logarithm of zero). Note that 
	the intensity scaling only affects the DF/Eades window, not the BF window.
	\item \textsf{File Format}: patterns can be stored in one of three formats: JPEG, TIFF, or BMP.
	\item \button{Go}: click this button to display a pair of BF/DF patterns for the values entered in the fields of this widget.
	\item \button{Surprise Me}: clicking this button will display a color pattern created by randomly selecting a number of 
	symmetrized DF patterns and adding them together (this may include the BF pattern).  Each time the button is clicked a different color pattern will appear;
	sometimes such a pattern is really strikingly beautiful, in which case the \button{Save} button can be used to save it.
	\item \button{Save}: clicking this button will bring up a file save dialogue box in which the user can enter the filename
	without extension.  The image that will be saved is the one currently displayed, either a pair of BF/DF images stored as a 
	single image, or a colorized pattern.
	\item \button{Close}: close the two widgets.  The various parameters set in the widget will be retained in memory and 
	will be the starting values the next time the LACBED Widget is started.
\end{itemize}


\subsection{CBED window\label{sec:idlcbed}}
This widget allows the user to create a conventional CBED pattern based on the data in the LACBED data file.  If the \button{CBED Window} is clicked
while the LACBED Window is still open, the latter will close and the new widget will appear.  Once again there are two windows, one a simple 
graphics window with no adjustable fields, the other a control window; both are shown in Fig.~\ref{fig:widget3}.

To understand the guidelines below, it is useful to recall the definition of the Laue center: \textit{the Laue Center is the projection
of the center of the Ewald sphere onto the Zero Order Laue Zone (ZOLZ).}  It provides an easy way to define a beam tilt, since all
that is needed are two coordinates in the ZOLZ; from those two coordinates, and knowledge of the length of the incident wave vector (which
is the inverse of the electron wave length), one can reconstruct the complete wave vector.  In the \ctp\ package, the Laue Center coordinates
are determined in units of what is internally called the $\mathbf{g}_a$ vector: this is the shortest reciprocal lattice vector that is normal
to the zone axis.  It is always oriented along the horizontal axis of a diffraction pattern (regular or convergent beam), and may have a zero structure
factor.

In the CBED Widget, the following parameters can be set by the user:
\begin{itemize}
	\item \textsf{Camera Length [mm]}: sets the camera length used for display.  Note that the patterns are computed for a camera length
	of $1000$ mm, and may appear pixelated for any other value in this field.  This widget is meant to be used to determine what the best
	parameters would be to record a given CBED pattern. 
	\item \textsf{Convergence Angle [mrad]}: beam convergence angle to be used for the CBED pattern (must be smaller than the LACBED beam convergence 
	angle.  When a new value is entered in this field, the drawing in the smaller graphics window is updated (see below); we will refer to this
	convergence angle as the \textit{local} convergence angle. 
	\item \textsf{Laue Center x and y}: The displayed CBED pattern does not need to have the incident beam along the original zone axis orientation;
	the beam can be tilted, which is equivalent to a sample tilt in the opposite direction.  
	This position is updated on the drawing whenever the Laue Center coordinates are changed.  The coordinates are measured
	in units of the length of the ``Horizontal g'' vector listed in the main program widget.
	\item \textsf{Intensity Scaling}: this is identical to the version described in the previous section.
	\item \textsf{Foil Thickness}: this is identical to the version described in the previous section.
	\item \textsf{Move Mode}: When a pattern is display, and the user changes the Laue Center coordinates, then the program will immediately
	display the new pattern (when \textsf{jump} is selected), or the pattern will gradually change (mimicking the sample being tilted) form the old
	to the new location.  The number of steps to be taken can be set with the selector below the \textsf{track} button.

\begin{figure}[t]
\leavevmode\centering
\epsfxsize=6.0in\epsffile{figs/CBEDwidget3}
\caption{\label{fig:widget3}\textsf{CBED} widget and the corresponding draw widget (note the different scales, the draw widget has a $1025\times 1025$ drawing area).}
\end{figure}

	\item \textsf{graphics window}: this window displays five pieces of information:
	\begin{itemize}
		\item the largest circle (displayed in green) represents the diffraction disk size for the LACBED convergence angle; the size of this circle is 
		constant and everything else is properly scaled with respect to it.
		\item the second largest circle (in red) represents the range within which the Laue center can be varied. The radius of this circle changes
		whenever the local convergence angle is changed, and is equal to the LACBED convergence angle minus the local convergence angle.  Any
		attempts to place the Laue center outside the red circle will result in the Laue center being placed on the circle.  The reason for
		this limit is that the LACBED data set does not have sufficient information to compute the complete CBED pattern for Laue center positions 
		outside of the red circle; if reflections appear to have been truncated, then that is likely due to the fact that the Laue center was
		located outside the red circle (can happen when the local convergence angle is changed)
		\item the set of blue circles represents the individual diffraction disks in the LACBED ZOLZ data set, drawn in the correct relative 
		positions and with radii scaled to the local convergence angle.
		\item the white cross represents the position of the ``Horizontal g'' reflection that is used as a measuring unit for the Laue Center 
		coordinates.  Note that it is possible for this location to not have a blue circle associated with it, if that particular reflection is 
		forbidden by space group symmetry; whether it is present or not, the program uses the length of that vector as the internal measuring unit.
		\item the small red cross and associated red circle represents the current location of the Laue Center;  this position can be changed by clicking anywhere 
		inside the red circle (in which case the CBED pattern will be updated automatically) or by manually entering Laue center coordinates
		and clicking the \button{Go}. 
	\end{itemize}
	\item \button{Go}: click this button to display a computed CBED pattern for the values entered in the fields of this widget.
	\item \button{Create .nml file}: when the user clicks on this button, a file dialogue box will pop up, asking for 
	a file name for a \textsf{CTEMmbcbed.nml}-type input file for the \textsf{CTEMmbcbed} program; the values in the file will be
	those for the currently displayed CBED pattern.  When the program is executed using this namelist file, then the result will be 
	a high resolution CBED pattern than can be viewed with the \textsf{MBCBED} widget described in the next section.  
	\item \button{Close}: close the two widgets.  The various parameters set in the widget will be retained in memory and 
	will be the starting values the next time the CBED Widget is started.
\end{itemize}

Note an important distinction between this CBED widget, and the MBCBED widget described in the next section: all patterns shown in the 
CBED widget are extracted from the LACBED data set and displayed at the user defined camera length.  Since this camera length may 
be larger than the value of $1000$ mm used to compute the disk positions in the \textsf{CTEMlacbed} program, the resulting pattern
may appear to be strongly pixelated.  To compute a high resolution version of the corresponding pattern, one should use the \textsf{CTEMmbcbed} 
program for pattern computation and the MBCBED widget for pattern display.


\subsection{MBCBED window\label{sec:idlcbed}}
When the \button{Load MBCBED File} is selected in the main widget, a file dialogue widget will open and prompt the user for a
file name; when that file has been loaded, the \button{MBCBED Window} will become active.  Clicking on it will display
two windows, one a control widget, the other a graphics window sized appropriately for the CBED patterns.  The main controls
in the \textsf{MBCBED} widget have all been explained before, so there is no need to repeat them here.  Basically, the MBCBED
widget can be used to display a single CBED pattern for a particular thickness, adjust the image contrast, and save the 
resulting pattern to a jpeg, tiff, or bmp file.


\subsection{Preferences file\label{sec:idlpref}}
Upon the first execution of the \textsf{CBEDDisplay.pro} routine, a preferences file will be created in the user's home folder.  The file is called \textsf{.CBEDgui.prefs}; the starting period
means that the file will not show up in a Finder window or on the UNIX command line when a simple \textsf{ls} command is issued.  This is a 
regular editable text file consisting of name::value pairs.  The first line shows the number of entries in the file, and then each entry is listed on a 
separate line.  A commented version of the preferences file is as follows:
\fvset{frame=lines,formatcom=\color{blue},fontsize=\footnotesize,numbers=left}
\VerbatimInput{CBEDgui-commented.prefs}
Note that the comments are not part of the actual preferences file.  All lines must be present in the file or the program will exit with an error message; the order 
of the lines is not important, but the file will always be written in the same order.
The values shown above are not default values, but represent a random snapshot of the program status after it has been used for a while.

When the program starts, it will first internally initialize all variables to default values, and then read the preferences
file, if it exists.  Then the widgets will be created using the preference values.  When the program is ended normally (by
pressing the \button{QUIT}), all current values, including the widget locations, are written to the preferences
file.



\newpage
\section{A few worked examples\label{sec:examples}}
In this final section, we show a few simple worked examples that illustrate how to use the combination of f90 and IDL programs 
to obtain simulated CBED images.  For all the examples, the input parameters in the \textsf{BetheParameters.nml} file 
should be set to \textsf{weakcutoff}$= 100$, \textsf{cutoff}$= 200$, and \textsf{sgcutoff}$=0.05$ nm$^{-1}$; note that this 
file must be located in the folder from which the programs are being executed (see comments below).

\subsection{A thickness series of $\mathbf{[112]}$ Cu CBED patterns\label{sec:example1}}
First of all, we create the crystal structure input file using the \textsf{CTEMmkxtal} program, or by directly editing a 
text file, as described in the main \ctp\ package manual.  Then we edit the \textsf{CTEMmbcbed.nml} namelist file as follows (omitting the comment lines):
\fvset{frame=lines,formatcom=\color{blue},fontsize=\footnotesize,numbers=left}
\VerbatimInput{CTEMmbcbed-example1.template}
It is a good idea to keep a copy of each input namelist file along with the generated data file;  for instance, you could create 
a folder called \textsf{CTEMruntime} in your home directory, and inside it, an \textsf{nml} folder to keep all the namelist files
together.  In addition, you could keep an \textsf{xtal} folder for crystal structure input files.  If the namelist file above is called
\textsf{MBCBEDCu112.nml}, and is located in the \textsf{nml} folder, and you run the program from the main \textsf{CTEMruntime}
folder, then the program call should be as follows:
\begin{verbatim}
	prompt> path/CTEMmbcbed nml/MBCBEDCu112.nml
\end{verbatim}
where ``prompt'' is the UNIX command line prompt, and ``path'' is the pathname to the \ctp\ \textsf{exe} executable folder.\footnote{If you
add this folder to your general shell path variable, then the pathname can be omitted from the program call.}  It is also a good idea to 
start the namelist file name with the characters that identify the program for which the file is intended.

Note that the \textsf{outname} entry in the file above points to a file that may be on a different drive than your home directory.  It is
good practice to keep the computation results in a separate location, one that is not included in your regular file back up.  Since the 
output files can always be recreated by rerunning the program, there is no need to keep large data files on a file system that is included in
the regular file system backup.

Scanning through the above input file, we see that the computation will be carried out for the Cu crystal structure; program output will
be sent to the terminal window; all planes with $d$-spacings larger than $0.02$ nm will be considered; the accelerating voltage 
is $200$ kV; the camera length is $1000$ mm; beam direction and foil normal are parallel to $[112]$; the Laue center coordinates are
set to zero, meaning that the computation will be carried out for the exact zone axis orientation; the output image will be $500\times 500$ pixels 
large; the beam convergence angle is $5.0$ mrad; and there are $10$ output thicknesses, starting at $20$ nm with $20$ nm increments.
Note that the magnitude of the camera length is meaningless; it is only the relative magnitude that counts, so if you perform two program runs,
one with a camera length of $1000$ mm, the other with $600$ mm, the resulting patterns will be scaled properly with respect to each other.\footnote{An 
older version of this program produced output that was properly scaled for real micrographs, on photographic negatives; in that case, the 
magnitude of the camera length was properly calibrated with respect to the negative dimensions.  Nowadays, with CCD cameras and photo editing
software, there is no longer a need for an absolute camera length scale.}

When the program is executed with this input file, it will read the namelist file, followed by reading the crystal structure file, all the while
printing program output to the terminal window.  The first part of the program analyzes the zone axis orientation in terms of diffraction 
symmetry, and determines which reflections should be taken into account.  All relevant HOLZ layers will be considered automatically,
only limited by the value of the \textsf{dmin} parameter.  For the given camera length, there are $10,913$ pixels inside the diffraction disk,
which corresponds to the number of incident wave vectors to be considered.  The program also found $14,859$ potential reciprocal lattice points 
contributing to the diffraction pattern; this is based on the value of \textsf{dmin}, which results in a total of $(2n+1)^3$ reciprocal lattice 
points, which, for $n=19$ (derived from the lattice parameters and \textsf{dmin}) results in an initial total of  $59,319$ reciprocal lattice points.
Of these points roughly one quarter are eliminated due to the systematic extinction rule for a face centered Bravais lattice, leaving a total
of $14,859$ potential points.  For each incident beam direction the program then determines which of these contribute to the CBED pattern.

After computation of all the necessary potential coefficients $U_{\mathbf{g}}$, the
program will consider all incident beam directions, one at a time,\footnote{A later program version will employ OpenMP multi core capabilities.}
and store the scattered intensities in the proper locations in the diffraction pattern array.  Since a given set of Bloch wave eigenvalues
and eigenvectors can be used for a range of thicknesses, all requested thickness values are computed in one step.  The output of this
program is a data file that can be read by the IDL \textsf{CBEDDisplay.pro} routine.  While this introduces an extra step in the process,
it does have the advantage that the pattern intensities can be scaled interactively, on a logarithmic scale, to make sure that all relevant 
CBED features are visible.  If the program were to produce tiff image files, for instance, then the user would need to specify ahead of time 
how the intensities should be scaled into the range $[0\ldots 255]$; the program would need to be executed again if a different intensity 
scaling were desired.  The data file stores the intensities as real numbers, so that the intensity scaling can be performed separately and
interactively.  At the end of the run, the program displays the computation time as well as the average number of strong beams (about $50$ in this
case) and weak beams (about $25$) for the complete run.  

After the program run has completed (on a Mac Pro with $3.06$ GHz Intel Xeon processor using a single core, 
the total program execution time, including writing the data file, was about five minutes for the 
present input file), the output file can be opened in the IDL visualization program.  Use the \button{Load MBCBED File} to display 
the file selection widget, and select the file you just created.  The \button{MBCBED Window}  will be activated, and all relevant data fields
will be filled in.  Click the \button{MBCBED Window}  to create a new widget, with a series of display parameters at the top, and a graphics
window that will display the CBED pattern.  Select the thickness you wish to display, as well as the intensity scaling mode (linear or 
logarithmic). If the logarithmic mode is chosen, then you can set the offset parameter to adjust the contrast;  this parameter will be added 
to the CBED pattern before the base-10 logarithm is computed and displayed.  Finally, you can save each individual pattern by clicking
on the \button{Save} and entering a filename; three image file formats are available (jpeg, tiff, and bmp).
The resulting output images can then be combined/edited with your favorite image processing program (Photoshop, GIMP, etc), to produce
figures for reports or papers.  Fig.~\ref{fig:ex1} shows a combination of six CBED patterns for thicknesses $40$ through $140$ nm, with
logarithmic scaling and an offset parameter of $10^{-4}$.

\begin{figure}[h]
\leavevmode\centering
\epsffile{figs/ex1}
\caption{\label{fig:ex1}Cu $[112]$ CBED patterns for the parameters stated in the text, and the indicated thickness values (in nm).  Intensities are 
scaled logarithmically (base-10, with an offset of $10^{-4}$).}
\end{figure}



\subsection{A $\mathbf{[230]}$ Si LACBED pattern computation\label{sec:example2}}
In this example we use the \textsf{CTEMlacbed} program to compute large angle CBED disks
for the $[230]$ zone axis orientation of silicon, an orientation that is often used to look at 
internal strains.  To run the program, we first need to create the namelist input file, which we 
will name \textsf{LACBEDSi230.nml}:
\fvset{frame=lines,formatcom=\color{blue},fontsize=\footnotesize,numbers=left}
\VerbatimInput{CTEMlacbed-example.template}
Most of the parameters here are self-explanatory. The \textsf{maxHOLZ} parameter defines the maximum number of 
HOLZ layers that will be present in the output file;  it does not mean that the computation itself is limited to $2$ HOLZ layers.
The number of contributing HOLZ layers is determined from a combination of \textsf{dmin}, the zone axis orientation, and
the Bethe potential parameters.  The beam convergence angle is typically set to a value that is significantly larger
than that used for conventional CBED patterns, because the program keeps all diffraction disks
separate (no disk overlap).  A value of $40$ to $60$ mrad has been found to produce reasonable results.  The \textsf{npix} parameter
defines the number of pixels along the radius of the diffraction disk, not counting the center; hence the complete disk will be
inscribed in a square of edge length L=2\textsf{npix}+1.  A value of \textsf{npix}$=256$ is a reasonable compromise between having
sufficient resolution in the patterns, and keeping the computation time down.  The total number of incident beam directions for which
the Bloch wave eigenvalue problem has to be solved is equal to the number of pixels inside the diffraction disk, which is just over 
$205000$ for \textsf{npix}$=256$; program execution time increases as \textsf{npix}$^2$. 

The program is then executed by the following command line:
\begin{verbatim}
	prompt> path/CTEMlacbed nml/LACBEDSi230.nml
\end{verbatim}
On the platform used for these examples, the total computation time amounted to about $27$ minutes.  This is an average time for small unit
cells;  for larger cells, the computation time will go up quite a bit.  The output file in this case has a size of abbout $1.16$ Gb, and contains
$116\times 10$ individual diffraction disks (the first number is the total number of independent families of planes in the file, the 
second one the number of thickness values).

To load the data file, we use the \button{Load LACBED File} in the IDL \textsf{CBEDDisplay.pro} program; after loading the file, 
both the \button{LACBED Window} and \button{CBED Window} will be activated.  We begin with the LACBED visualization, so click on
the \button{LACBED Window}, which will create two new windows, one a control widget, the other a graphics widget with two image areas,
one for the BF disk, the other for the DF disk or the Eades pattern.  Make sure the Dark Field Display mode
is set to Single DF, set the thickness to $120$ nm, select the $(00\bar{4})$ reflection family
in the ZOLZ menu, use normal intensity scaling, and click on the \button{Go}.  This will display both the BF disk on the left,
and the DF $(00\bar{4})$ disk on the right, as shown in Fig.~\ref{fig:ex2}.

\begin{figure}[h]
\leavevmode\centering
\epsffile{figs/ex2}
\caption{\label{fig:ex2}Si $[230]$ BF disk (left) and $(00\bar{4})$ DF disk (middle) for a thickness of $120$ nm.  On the right, symmetrized DF
pattern is shown, after application of the Whole Pattern symmetry $\mathbf{m}$.}
\end{figure}

From the main window we see that the Whole Pattern symmetry is $\mathbf{m}$, and the mirror plane is clearly oriented vertically,
based on the BF pattern.  Setting the Dark Field Display Mode to Symmetrized, and hitting the \button{Go}, applies the WP symmetry to
the $(00\bar{4})$ dark field disk and generates the pattern shown on the right in Fig.~\ref{fig:ex2}.

Next, set the Eades parameters to $50$ and $100$ mrad, respectively, select logarithmic intensity scaling with an offset of $10^{-4}$, 
and hit the \button{Go}.  The DF image, shown in Fig.~\ref{fig:ex3} for a detector range $[50,100]$ mrad (left) and $[25,50]$ mrad (right),
now displays the superposition of all symmetrized DF disks from all HOLZ layers (including ZOLZ) for which 
the $2\theta$ angle lies inside the Eades parameter range.  In principle, one should be able to record this pattern experimentally 
using a HAADF detector in combination with a conical scanning mode.  The nice thing about this DF image is that it clearly shows high
contrast HOLZ lines, without the underlying dynamical diffraction fringes of the typical BF pattern.  Whether or not such patterns can
be used for strain analysis remains to be seen.

\begin{figure}[h]
\leavevmode\centering
\epsffile{figs/ex3}
\caption{\label{fig:ex3}Si $[230]$ DF Eades patterns for detector ranges of $[50,100]$ mrad (left) and $[25,50]$ mrad (right).}
\end{figure}

Next, click to the \button{CBED Window} in the main widget, which will replace the LACBED widgets with two new windows, one with
controls, the other a graphics window.  Set the camera length to $2000$ mm, the convergence angle to $7.5$ mrad, the Laue Center to $(0,0)$,
use normal intensity scaling, select a thickness of $120$ nm and click the \button{Go}.  The resulting pattern will only show a BF disk
with a few HOLZ lines across it.  Then set the intensity scaling more to logarithmic with an offset of $10^{-3}$, and additional 
disks will appear, but now the HOLZ lines become more difficult to observe.


\begin{figure}[h]
\leavevmode\centering
\epsffile{figs/ex4}
\caption{\label{fig:ex4}Si $[230]$ zone axis pattern (a), and tilted pattern with the Laue center at position $(2,0)$, so that the 
$(00\bar{4})$ reflection is in Bragg orientation; notice the symmetric fringe pattern across the disk.}
\end{figure}

Then set the camera length to $400$ mm and click the \button{Go}; now you can see a few reflections in two HOLZ rings, as shown in 
Fig.~\ref{fig:ex4}(a).\footnote{Note that there is no pattern save button in this widget, because the patterns are typically displayed
at a low angular resolution, resulting in pixelated disks.  To create a high resolution CBED pattern, see the following example.}
Note how the white cross in the navigation window is located at one fourth the distance between the center disk and the center of the first disk along the horizontal
direction.  We know that to bring $\mathbf{g}$ into Bragg orientation, we need to position the Laue center at $\mathbf{g}/2$, so
to bring the $(00\bar{4})$ reflection into Bragg orientation, we need to set the Laue x-coordinate to $2$;  note how the 
red cross in the navigation window is now exactly in the middle between the two disks, as required.  Hitting the \button{Go} then
produces the pattern shown in Fig.~\ref{fig:ex4}(b).  When the Move Mode is set to Jump, then the new pattern will be shown
immediately; when the Track mode is selected, along with a number from the pull down list (say $30$), then the pattern will gradually 
change to the new beam orientation, as if the sample were being tilted.  In fact, you can click anywhere inside the navigation window, 
and the Laue center will be changed to the corresponding location; clicking outside the large red circle is allowed, but the Laue 
center will not move outside that circle, since the corresponding diffraction data is not available to the program.  Feel free to play
around with this artificial sample tilt option.



\begin{figure}[h]
\leavevmode\centering
\epsffile{figs/ex5.pdf}
\caption{\label{fig:ex5}Si $[230]$ tilted pattern with the Laue center at position $(0,4)$; interpolated LACBED pattern in (a), full 
resolution pattern in (b).}
\end{figure}

\subsection{Using a LACBED simulation to initialize a regular CBED computation\label{sec:example3}}
As a final example we discuss how one can use the \textsf{LACBED} mode to initialize an MBCBED pattern computation.
For the Si $[230]$ LACBED data set discussed before, set the camera length to $400$ mm, the convergence angle to $7.5$ mrad,
the Laue center coordinates to $(0,4)$, and then hit both the \button{Go} and the \button{Create .nml file} (one after the other, obviously).
The latter button will create a file called \textsf{CTEMmbcbed.nml} in the folder that the \textsf{LACBED} data set was loaded form.
You should edit this file, which is an input file for the \textsf{CTEMmbcbed} program, with the correct parameters initialized to compute 
a high resolution version of the diffraction pattern that is currently displayed in the CBED Pattern window; the only parameter that 
will need to be modified in the output file name at the end of the namelist file, say to \textsf{Si230-output.data}.  Then execute the \textsf{CTEMmbcbed} program with 
this namelist file as input; this should only take a few minutes, and then you can open this file with the \button{Load MBCBED File} on the main
program widget.  Select the same thickness and intensity scaling (log, offset $10^{-3}$) and display the pattern, which should be a higher
resolution version of the one interpolated from the LACBED data before; both patterns are shown in Fig.~\ref{fig:ex5}.
This example shows that one can play around with the LACBED interface to set up the correct imaging parameters for the CBED pattern,
and then compute a high resolution version of that pattern using the MBCBED program.

{\color{blue}Note that the patterns in Fig.~\ref{fig:ex5} are not completely identical in terms of the HOLZ reflections; whether this 
is a problem in the IDL CBEDDisplay.pro routine or in the CTEMmbcbed program remains to be sorted out.  Figure will be updated
once the problem has been resolved.  }





\end{document}



