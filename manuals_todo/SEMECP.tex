
\documentclass[DIV=calc, paper=letter, fontsize=11pt]{scrartcl}	 % A4 paper and 11pt font size

\usepackage[body={6.5in,9.0in},
  top=1.0in, left=1.0in]{geometry}
  
\usepackage[english]{babel} % English language/hyphenation
\usepackage[protrusion=true,expansion=true]{microtype} % Better typography
\usepackage{amsmath,amsfonts,amsthm} % Math packages
\usepackage[svgnames]{xcolor} % Enabling colors by their 'svgnames'
\usepackage[hang, small,labelfont=bf,up,textfont=it,up]{caption} % Custom captions under/above floats in tables or figures
\usepackage{booktabs} % Horizontal rules in tables
\usepackage{fix-cm}	 % Custom font sizes - used for the initial letter in the document
\usepackage{epsfig}
\usepackage{sectsty} % Enables custom section titles
\allsectionsfont{\usefont{OT1}{phv}{b}{n}} % Change the font of all section commands

\usepackage{fancyhdr} % Needed to define custom headers/footers
\pagestyle{fancy} % Enables the custom headers/footers
\usepackage{lastpage} % Used to determine the number of pages in the document (for "Page X of Total")
\usepackage{color}

\usepackage{fancyvrb}% used to include files verbatim
%\usepackage{chemsym}

\usepackage{hyperref}
\usepackage[squaren]{SIunits}

\usepackage[backend=bibtex,style=authoryear,maxnames=15]{biblatex}
\renewbibmacro{in:}{}

% Count total number of entries in each refsection
\AtDataInput{%
  \csnumgdef{entrycount:\therefsection}{%
    \csuse{entrycount:\therefsection}+1}}

% Print the labelnumber as the total number of entries in the
% current refsection, minus the actual labelnumber, plus one
\DeclareFieldFormat{labelnumber}{\mkbibdesc{#1}}    
\newrobustcmd*{\mkbibdesc}[1]{%
  \number\numexpr\csuse{entrycount:\therefsection}+1-#1\relax}


\addbibresource[label=papers]{citations.bib}
%\addbibresource[label=books]{mypubs.bib}
%\addbibresource[label=edited]{mypubs.bib}
%\addbibresource[label=chapters]{mypubs.bib}


% Headers - all currently empty
\lhead{}
\chead{}
\rhead{}

% Footers
\lfoot{\textsf{EMECP} manual, v3.0, \today}
\cfoot{}
\rfoot{\footnotesize Page \thepage\ of \pageref{LastPage}} % "Page 1 of 2"

\renewcommand{\headrulewidth}{0.0pt} % No header rule
\renewcommand{\footrulewidth}{0.4pt} % Thin footer rule

\usepackage{lettrine} % Package to accentuate the first letter of the text
\newcommand{\initial}[1]{ % Defines the command and style for the first letter
\lettrine[lines=3,lhang=0.3,nindent=0em]{
\color{DarkGoldenrod}
{\textsf{#1}}}{}}

\usepackage{titling} % Allows custom title configuration

\newcommand{\HorRule}{\color{DarkGoldenrod} \rule{\linewidth}{1pt}} % Defines the gold horizontal rule around the title

\pretitle{\vspace{-1.5in} \begin{center} \HorRule \fontsize{25}{25} \usefont{OT1}{phv}{b}{n} \color{DarkRed} \selectfont} % Horizontal rule before the title

\title{Electron Channeling\\ Pattern Simulations} % Your article title

\posttitle{\par\end{center}\vskip 0.5em} % Whitespace under the title

\preauthor{\begin{center}\large \lineskip 0.5em \usefont{OT1}{phv}{b}{sl} \color{DarkRed}} % Author font configuration

\author{\vspace*{-0.7in}} % Your name

\postauthor{\footnotesize \usefont{OT1}{phv}{m}{sl} \color{Black} % Configuration for the institution name

\par\end{center}\HorRule} % Horizontal rule after the title
\date{Program Manual, v3.0, \today}

\newcommand{\ctp}{\textsf{EMsoft}}
%
\newcommand{\upg}[1]{\mathrm{i}U_{\mathbf{#1}}^{\prime}}
\newcommand{\combo}[1]{U_{\mathbf{#1}}+\upg{#1}}
\newcommand{\upgcombo}[2]{2k_{0}s_{\mathbf{#1}}+\upg{#2}}
\newcommand{\ugh}[2]{U_{\mathbf{#1}-\mathbf{#2}}}
\newcommand{\ughp}[2]{U_{\mathbf{#1}'-\mathbf{#2}}}
\newcommand{\ughpp}[2]{U_{\mathbf{#1}-\mathbf{#2}'}}
\newcommand{\kkg}[1]{k_{0}^{2}-(\mathbf{k}+\mathbf{#1})^{2}}
\newcommand{\Cg}[1]{C_{\mathbf{#1}}}
\newcommand{\Cgj}[2]{C_{\mathbf{#1}}^{(#2)}}
\newcommand{\Cgjp}[2]{C_{\mathbf{#1}'}^{(#2)}}
\newcommand{\Cgja}[2]{C_{\mathbf{#1}}^{(#2)\ast}}
\newcommand{\button}[1]{\colorbox{green}{\textsf{#1}} button}


\begin{document}
\maketitle

\begin{figure*}[h]
\leavevmode\centering
\epsffile{figs/SEMlogo}
\end{figure*}

\renewcommand{\contentsname}{Table of Contents}
{\small\tableofcontents}

\newpage
\section{Introduction}
This manual describes a set of three programs to compute realistic electron channeling patterns (ECPs).  The sequence of programs is 
very similar to that of the \textsf{EMEBSD} suite of programs; in fact, the first program, \textsf{EMMCOpenCL} is identical for both
types of patterns.  The ECPs can be 


one written in Fortran-90,\footnote{f90 is a much richer language than the original fortran-f77, and is
used for all programs in the \ctp\ package.} 
the other in IDL,\footnote{The \textit{Interactive Data Language} is an interpreted scripting language with extensive graphics capabilities.} 
that can be used for the simulation of zone axis electron channeling patterns (ECPs).  Such ECPs can be
observed in a scanning electron microscope (SEM) when the incident beam is rocked about a single point in the sample surface and 
the back-scatter electron (BSE) signal is displayed as a function of the beam rocking angle.  The main f90 program
is called \textsf{EMECP} (note that all programs in the \ctp\ package start with the letters ``EM'').
The output generated by this program can then be visualized by the IDL routine \textsf{ECPDisplay.pro}.  

On the following pages we will try to accomplish four tasks:
\begin{enumerate}
	\item Explain briefly the underlying image formation theory and the numerical approach followed by the f90 program (section~\ref{sec:theory});
	\item Document the input files for the f90 program (section~\ref{sec:f90ecp});
	\item Document the IDL interface (section~\ref{sec:idl});
	\item Explain the use of these programs by means of a few basic examples (section~\ref{sec:examples}).
\end{enumerate}

%At the time of writing of this manual, these programs have been successfully compiled on the Mac OS X platform using the commercial \textit{Absoft} compiler,
as well as the public domain gfortran compiler.  There is no basic reason why this code should not work on Windows and Linux systems, and efforts are underway
to make sure that the code will run on all platforms. It would be interesting to see some of this code
implemented in a super computer setting, since many of the routines should be quite parallellizable.  Where possible, some of the computations
are carried out using the OpenMP directives, so that multiple cores can be used.  This aspect of the code is still under further development,
and we anticipate that GPUs will become useful as well, using the OpenCL approach.

The \ctp\ package is entirely written in f90 and does not use any of the newer commands available in the 95 and 2013 versions.
The source code is extensively commented, using regular comment lines, but also using DOxygen documentation generation commands.  Hence
there exists an extensive on-line documentation of all variables, variable types, modules, subroutines, functions, etc. for the latest version 
of the code.  For selected programs, more extensive manual pages are available.  If interested, please contact the author for further information.

The visualization part of the code consists of a series of  IDL routines that are available as source code or in the form of a Virtual Machine application. 
If you have an IDL license, then you can compile and run the IDL source code; alternatively, if you do not have a license,
then you can use one of the VM apps to perform the same task.  Note that in the VM environment, you will not be able to alter/compile the 
source code.

\newpage
\section{Electron channeling patterns: the underlying theory\label{sec:theory}}
The computation of an SEM-generated ECP can be carried out by means of a Bloch wave approach. The probability
of a beam electron being back-scattered by an atom in the bulk of the sample depends on the probability that the electron
comes sufficiently close to that atom.  Since a back-scatter event can take place across a range of depths in the sample, one 
integrates over the thickness, within the Bloch wave formalism, the probability of scattering from each subset $\mathcal{S}$ of atomic sites within the 
unit cell for a given incident beam direction $\mathbf{k}_0$; this results in \parencite{winkelmann2003a,winkelmann2008a}:
\begin{equation}
	\mathcal{P}(\mathbf{k}_0) = \sum_{\mathbf{g}} 
    \sum_{\mathbf{h}} S_{\mathbf{g}\mathbf{h}}L_{\mathbf{g}\mathbf{h}},
    \label{eq:prob}
\end{equation}
with
\begin{subequations}
\begin{align}
    S_{\mathbf{g}\mathbf{h}} &\equiv \sum_{n}\sum_{i\in\mathcal{S}_n} Z^2_n\,e^{-M^{(n)}_{\mathbf{h}-\mathbf{g}}}\,e^{2\pi\mathrm{i} 
    (\mathbf{h}-\mathbf{g})\cdot\mathbf{r}_{i}};\label{eq:defa}\\
    L_{\mathbf{g}\mathbf{h}} &\equiv \sum_{j}\sum_{k} 
    C^{(j)\ast}_{\mathbf{g}}\alpha^{(j)\ast}\mathcal{I}_{jk}\alpha^{(k)}
    C^{(k)}_{\mathbf{h}}.\label{eq:defb}
\end{align}
\end{subequations}
The first summation in (\ref{eq:defa}) runs over all the positions in the asymmetric unit of the unit cell; the second
sum runs over all equivalent positions in each subset $\mathcal{S}_n$.
The parameters $\alpha^{(j)}$ in (\ref{eq:defb}) are the Bloch wave excitation amplitudes (see \parencite{humphreys1979a,degraef2003b} for 
details on the Bloch wave approach), $C_{\mathbf{g}}^{(j)}$ are the 
Bloch wave coefficients, and the matrix $\mathcal{I}_{jk}$ is defined by the integral
\begin{equation}
	\mathcal{I}_{jk}\equiv \frac{1}{z_{0}}\int\limits_{0}^{z_{0}} 
    e^{-2\pi(\alpha_{jk}+\mathrm{i}\beta_{jk})z}\,\mathrm{d}z,
\end{equation}
where $z_0$ is a pre-determined thickness.  The ECP will change a lot for small values of $z_0$ but tends to converge towards 
a stable pattern for $z_0$ larger than a few tens of nanometers.  The program described in the next section can compute the ECP
as a function of $z_0$.  Carrying out the integration we find:
\begin{subequations}
\begin{align}
    \mathcal{I}_{jj} &= \frac{1-e^{-4\pi q^{(j)}z_{0}}}{4\pi 
    q^{(j)}z_{0}};\label{eq:a2a}\\
    \mathcal{I}_{jk} &= 
    \frac{1-e^{-2\pi(\alpha_{jk}+\mathrm{i}\beta_{jk})z_{0}}} 
    {2\pi(\alpha_{jk}+\mathrm{i}\beta_{jk})z_{0}}\qquad (j\ne k);
    \label{eq:a2b}
\end{align}
\end{subequations}
where
\begin{subequations}
\begin{align}
    \alpha_{jk} &= q^{(j)}+q^{(k)};\\
    \beta_{jk} &= \gamma^{(j)}-\gamma^{(k)}.
\end{align}
\end{subequations}
The complex numbers $\lambda^{(j)}\equiv\gamma^{(j)}+\mathrm{i}q^{(j)}$ are the eigenvalues of the dynamical scattering matrix in the 
presence of absorption.  The absorptive form factors of Weickenmeier and Kohl \parencite{weickenmeier1991a} can be used to estimate 
both the normal and anomalous absorption lengths (i.e., the imaginary part of the electrostatic lattice potential).   The backscatter probability goes as the square
of the atomic number, which accounts for the factor $Z_n^2$ in eq.~(\ref{eq:defa}). The thermal movements of 
atoms are described by $M^{(n)}_{\mathbf{g}}\equiv B_{n}\vert\mathbf{g}\vert^2/4$, with $B_n$ the standard (room temperature)
Debye-Waller factor for element $n$.  The thickness $z_0$ is usually taken as the Bethe parameter \parencite{reimer1985a}, which is typically a few tens of nanometers.  Alternatively,
one can take the limit for $z_0\rightarrow\infty$, in which case the matrix $\mathcal{I}_{jk}$ becomes diagonal.

The ECP problem can also be solved by means of the scattering matrix approach, which is described in more
detail in the manual for the ECCI program.  In the present program, the user can select the computational 
approach by means of a switch in the input file (see next section).

In a more complete simulation, one would use a Monte Carlo approach to compute the actual number of electrons that leave
the sample as BSEs after undergoing no other inelastic events inside the sample, since those are the electrons that make 
up the ECP signal.  This requires some additional modeling that is currently (end of 2013) in the works.

In terms of the \textsf{EMECP} program itself, the major program portions are:
\begin{itemize}
	\item read crystal structure and general parameter files;
	\item determine list of contributing reflections and incident wave vector range;
	\item for each incident beam direction, compute $\mathcal{P}(\mathbf{k}_0)$ for 
	different thickness values.
	\item store all computed patterns in a file that can be read by the IDL \textsf{ECPDisplay} program.
\end{itemize}

\section{The \protect\textsf{EMECP.f90} program\label{sec:f90ecp}}

\subsection{Program overview\label{sec:f90overviewecp}}
The \textsf{EMECP} program is a relatively basic program and is in its structure rather similar to the \textsf{EMlacbed} program.
The only major difference lies in the computation of the thickness-integrated intensity rather than individual transmitted and 
diffracted beam intensities.  The program input uses the namelist file approach described in the next section, and the output is a 
binary file, for now with a special format.\footnote{At some point in the future, all output files from the \ctp\ programs will use
the public domain HDF5 file format.}  

The program reads the input and crystal structure files, determines which reflections will contribute to the ECP for the specified zone
axis and angular range, and then performs
a Bloch wave computation for each incident beam direction.  The resulting ECP intensities are stored as a function of the sample depth.
The output file can be visualized using the \textsf{ECPDisplay} IDL interface.

\subsection{Namelist input files\label{sec:f90inputecp}}
The main (and only) input file for this program, \textsf{EMECP.nml}, is formatted as follows:
\fvset{frame=lines,formatcom=\color{blue},fontsize=\footnotesize}
\VerbatimInput{../templatefolder/EMECP.template}
Most of the parameters are straightforward.  The crystal structure file can be created with 
the \textsf{EMmkxtal} program.  The foil normal should be taken equal to the incident beam direction; having non-parallel 
directions has not been thoroughly tested.  The program allows for a distorted unit cell; set \textsf{distort} to \textsf{.TRUE.}
and list all six new lattice parameters in the variables \textsf{abcdist} and \textsf{albegadist}, and then the computation will
be carried out for the distorted cell.

The \textsf{compmode} parameter has a default value of \textsf{`Blochwv'}, but can be set to \textsf{`ScatMat'} instead, to use the 
scattering matrix approach rather than the Bloch wave approach.  Preliminary tests show that the scattering matrix and Bloch wave approaches 
have approximately equal execution times.  When the scattering matrix approach is selected, one can also define the \textsf{zintstep}
parameter, which defines the integration step size along the beam direction; the default value is $1.0$ \nano\meter, but the user 
may want to select a smaller value (e.g. $0.5$) for more complex unit cells.  Note that the ratio of the output thickness value(s)
to the integration step size must be an integer.  Smaller values of \textsf{zintstep} will lead to longer execution times compared to the
Bloch wave approach, which is not affected by the integration step size.

The beam convergence angle is determined by stating the maximum tangential wave vector component in units of the length of $\mathbf{g}_a$,
which is the reciprocal lattice vector corresponding to the horizontal direction in the ECP.  At the start of the program, the unit
cell information will be shown, followed by wave length and diffraction symmetry information; following this, there will be a line of 
the form:
\begin{verbatim}
	Reciprocal lattice vectors : (  0 -1  1) and (  1  0 -1)
\end{verbatim}
The first vector listed is the vector $\mathbf{g}_a$. Note that this vector need not correspond to an allowed reflection; it is
the shortest $\mathbf{g}$ vector in the zero order Laue zone, without taking into account any systematic absences due to lattice 
centering or lattice symmetry.  The largest tangential component \textsf{ktmax} is defined in multiples of $\vert\mathbf{g}_a\vert$;
so, setting \textsf{ktmax} equal to $5$ means that $\vert\mathbf{k}_{t,\text{max}}\vert=5\vert\mathbf{g}_a\vert$.  Since $\mathbf{k}_t=-\mathbf{g}/2$
in order to have $\mathbf{g}$ in Bragg orientation, a value of of \textsf{ktmax}$=5$ means that $10\mathbf{g}_a$ will be in Bragg orientation
for the largest beam tilt.

The range of reciprocal space that will be considered by the program is determined by the parameter \textsf{dmin}.  The default value is
equal to $0.025$ \nano\meter, which has proven to be a reasonable compromise between not including all the necessary HOLZ reflections (when
\textsf{dmin} is taken too large) and a rapidly increasing computation time (if \textsf{dmin} is taken too small).  Note that this value is
a little larger than what is typically used for TEM simulations, due to the much smaller Ewald sphere radius for typical SEM voltages.  This means
that for SEM simulations, we do not need to go as far out from the origin as we do for TEM simulations.

The size of the ECP is determined by the value of \textsf{npix}; the beam tilt range will be divided into \textsf{npix} equal intervals, including 
the zone axis orientation, so that the output ECP has dimensions $(2$\textsf{npix}$+1)^2$.  For the current program version, the ECP will be 
limited by a circular aperture; in a future version, this limitation will become optional.

Finally, the output file will contain ECPs for a range of thickness values; these are the integration ranges $z_0$
introduced in section~\ref{sec:theory}.  The thickness values are defined by setting the starting thickness \textsf{startthick} and
the increment value \textsf{thickinc} (in \nano\meter), and the number of increments \textsf{numthick}.  Usually, the ECP converges 
to a stable pattern for $z_0$ a few tens of nanometers, but that depends on the crystal structure and atomic numbers.


\subsection{Bethe parameters namelist file\label{sec:f90BetheParameters}}
The Bethe potential approximation used in this program can be adjusted by the
user via the \textsf{BetheParameters.nml} namelist file, which must be present 
in the folder in which the program is executed; if this file is not present, then the
program will assume default values for all three parameters.  The values shown below
are the current program defaults.
\fvset{frame=lines,formatcom=\color{blue},fontsize=\footnotesize}
\VerbatimInput{../templatefolder/BetheParametersSEM.template}
The last parameter sets the excitation error cutoff level for double diffraction reflections.  The issue with
such reflections is that they have a zero structure factor (and hence $U_{\mathbf{g}} = 0$), so that 
the standard Bethe potential threshold approach can not be used at all.  The current implementation
allows the user to set a threshold for the excitation error (in nm$^{-1}$) above which the double diffraction reflection
will not contribute at all.  Recall that double diffraction reflections can only occur in so-called non-symmorphic 
space groups, i.e., in space groups that have at least one screw axis or glide plane symbol in their space 
group symbol; in other words, in the space group \textbf{Fm$\bar{\mathbf{3}}$m}, there are no glides or screws in
the symbol, so double diffraction reflections are not possible for \textit{any} zone axis orientation.  For the space group \textbf{P6$_{\mathbf{3}}$/mmm},
on the other hand, there is a $6_3$ screw axis in the space group symbol, therefore it is possible for certain
zone axes to display double diffraction reflections.  The CBED programs automatically take these reflections into account, but the user must
set the excitation error threshold value, or accept the program default.

It is also important to note that the value of the \textsf{weakcutoff} parameter should always be less that
that of the \textsf{cutoff} parameter if one wants to make use of the Bethe approximation.  Testing has shown
that a \textsf{weakcutoff} value slightly larger than half the \textsf{cutoff} value produces reasonable 
results.  To avoid using Bethe potentials, the two parameters should be put equal to each other, but then the 
program will take potentially much longer to run.

Finally, it is likely that the Bethe approximation criteria will be modified in the future to provide a more
efficient algorithm.  In addition, second order perturbation corrections will be included, which should allow
for an even smaller dynamical matrix in most cases.


\section{The \protect\textsf{ECPDisplay.pro} program\label{sec:idl}}

\subsection{Program overview\label{sec:idloverview}}
The data file generated by the \textsf{EMECP} program consists of geometrical information and arrays of ECPs 
as a function of thickness.  This file can be visualized using the IDL program
\textsf{ECPDisplay.pro}, described in this section.

The \textsf{ECPDisplay.pro} routine requires an IDL program license or can be executed using the Virtual Machine, without the need for a license. 
If you do have a license, then make sure that the folder containing the routines is part of your IDL pathname,
and that IDL is properly installed for the UNIX shell that you will be using (csh, bsh, etc...).
To execute the program, first start an IDL session in a terminal window (using the /Appplications/Utilities/Terminal program), 
start IDL, compile the \textsf{ECPDdisplay.pro} routine by typing:
\begin{verbatim}
	IDL> .r ECPDisplay <return>
\end{verbatim}
at the IDL prompt and hitting return, followed by typing 
\begin{verbatim}
	IDL> ECPDisplay <return>
\end{verbatim}
to start the display program.  The main program widget window will appear as well as a file selector window.  
Details of all the windows are described in the following subsections.

In addition to requiring an IDL license, the program expects the X-windows environment to be installed.  On the Mac, this corresponds
to the \textsf{XQuartz} program which can be downloaded from the open source site \textsf{http://xquartz.macosforge.org/}.  This program
must be installed in the system /Applications/Utilities folder and requires OS X 10.6 or later.

Useful thing to know: if the \textsf{ECPDisplay.pro}  program hangs for some reason, you can reset your IDL session by typing
\begin{verbatim}
	IDL> .reset <return>
\end{verbatim}
This will destroy all program widgets and reset IDL to its original state.  If that does not work, then you may have to ``force quit'' the Terminal program
using the standard Esc-Option-Command key stroke.

There is also a Virtual Machine version of this program, which can be executed without an IDL license.

\subsection{Main window\label{sec:idlmain}}
This display program is really very simple, and only allos for the display of ECP patterns, and 
the superposition of a grid on the pattern.
The main window contains no user adjustable parameters, and only displays a few parameters from the computation.
The main display widget when the program is started is shown in Fig.~\ref{fig:widget1}.  

\begin{figure}[h]
\leavevmode\centering
\epsffile{figs/ECPwidget1}
\caption{\label{fig:widget1}Main \textsf{ECPDisplay.pro} program widget.}
\end{figure}

The only user options are listed at the bottom of the widget: the \button{ECP File} and the \button{Quit}, the latter's function being 
rather clear.  When the \button{ECP File} is selected, a file selection interface appears, and the user can select an ECP data file. 
Several parameters will be read from the file, and their values displayed in the main widget.  Note that both the main widget and 
the pattern display widget can be repositioned anywhere on the screen, and that the window locations will be stored in the preferences
file described in the next section.  

The pattern display widget is shown in Fig.~\ref{fig:widget2}.  The user selections are as follows:
\begin{itemize}
	\item \textbf{Integration Depth}: the \textsf{EMECP.f90} program generated a data file with ECPs for a number of different
	integration depths $z_0$; the user can select any of these values and the corresponding pattern will appear in the graphics area.
	Note that the ECP converges to a stable pattern after some integration depth.
	\item \textbf{Coordinate Grid}: this toggle will display a square coordinate grid on top of the ECP.  The spacing between the 
	grid lines is determined by the Bragg angle of the primary reflection $\mathbf{g}_a$ (see section~\ref{sec:f90inputecp}).  The grid
	can be used to determine an incident beam direction for the \textsf{EMECCI.f90} program, which is described in a separate manual.
	Note that the number of vertical or horizontal lines between the center line and one of the pattern edges is equal to the 
	integer part of the parameter \textsf{ktmax} that is used in the main program's namelist file.
	\item \textbf{Blur radius}: radius of a Gaussian blurring filter that can be applied to the ECP; a value of $0.0$ implies no
	filter.  Useful radius range is $[0.0,10.0]$.  This filter can be used to blur the patterns so that they look a little more 
	like the experimental ones.
	\item \textbf{File Format}: check boxes to determine the file format to be used when the \button{Save}\ at the bottom of the widget
	is selected.  Note that the default location for storage of the image files is the same folder in which the data file resides. 
	\item \textbf{Selected Position}: when the user clicks anywhere on the ECP, these text fields will display the coordinates of the point 
	that was selected.  The coordinates are measured with respect to the square coordinate grid described above.  This provides an easy
	way to determine which Laue center coordinates should be used for the main input file of the \textsf{EMECCI} program, when this
	program is used in single image mode.
\end{itemize}

\begin{figure}[h]
\leavevmode\centering
\epsfxsize=2.75in\epsffile{figs/ECPwidget2}
\caption{\label{fig:widget2}ECP pattern display widget.}
\end{figure}

\subsection{Preferences file\label{sec:idlpref}}
Upon the first execution of the \textsf{ECPDisplay.pro} routine, a preferences file will be created in the user's home folder.  
The file is called \textsf{.ECPgui.prefs}; the starting period
means that the file will not show up in a Finder window or on the UNIX command line when a simple \textsf{ls} command is issued.  This is a 
regular editable text file consisting of name::value pairs, although there is no real
need for the user to ever edit this file.  The first line shows the number of entries in the file, and then each entry is listed on a 
separate line.  A commented version of the preferences file is as follows:
\fvset{frame=lines,formatcom=\color{blue},fontsize=\footnotesize,numbers=left}
\VerbatimInput{ECPgui-commented.prefs}
Note that the comments are not part of the actual preferences file.  All lines must be present in the file or the program will exit with an error message; the order 
of the lines is not important, but the file will always be written in the same order.
The values shown above are not default values, but represent a random snapshot of the program status after it has been used for a while.

When the program starts, it will first internally initialize all variables to default values, and then read the preferences
file, if it exists.  Then the widgets will be created using the preference values.  When the program is ended normally (by
pressing the \button{QUIT}), all current values, including the widget locations, are written to the preferences
file.

\section{A worked example\label{sec:examples}}
In this final section, we show a simple worked example to illustrate how to use the combination of f90 and IDL programs 
to obtain simulated ECPs.  

We'll use the GaP Zinc-Blende structure for this example.  This structure has space group $\mathbf{F\bar{4}3m}$ with lattice parameter
$a=0.54505$ \nano\meter; Ga is located in the origin $(0,0,0)$, and P at $(1/4,1/4,1/4)$; both atoms have full occupancy $1.0$
and Debye-Waller factor can be set to the default value of $0.004$ \nano\meter$^{-2}$.  This value produces decent results, but 
if more accurate Debye-Waller factors for Ga and P are known, then those should be used instead.  The crystallographic parameters
can be entered into a data file using the \textsf{EMmkxtal} program; it would make sense to name this file \textsf{GaP.xtal}.

The input parameters for the ECP computation are shown here:
\fvset{frame=lines,formatcom=\color{blue},fontsize=\footnotesize,numbers=left}
\VerbatimInput{ECPGaP100.nml}
If multiple different runs are going to be carried out, then it would make sense to store this namelist in a 
file with an appropriate name, say \textsf{ECPGa100.nml}.  Note that most of the parameters have their default values.  The
zone axis \textsf{k} and the foil normal \textsf{fn} should be taken to be equal for the current program version.
Remember that you can obtain a template namelist file by executing the \textsf{EMECP} program with the $-t$ option; 
this will produce a \textsf{EMECP.template} file in the current folder, which can then be edited to set the correct
parameters. Finally, before we run the program, we must verify that the Bethe potential parameters have been set to reasonable values:
\textsf{weakcutoff}$= 60$, \textsf{cutoff}$= 120$, \textsf{sgcutoff}$=0.05$ nm$^{-1}$.

Next, we can execute the program by entering
\begin{verbatim}
	../exe/EMECP ECPGa100.nml
\end{verbatim}
where we assume that the program is being run from a folder at the same level as the exe folder.  Several pieces of
information will be printed on the screen, and then the program will update its progress every $5\%$ of the total number
of incident beam directions to be considered; this number is determined by the \textsf{npix} parameter, and for 
\textsf{npix}$=256$, we have a total of $205,861$ incident beam directions.
On an older generation Mac Pro ($2\times 3.06$ GHz $6$-core intel Xeon) the main loop of the program took about $148$ minutes
to complete, or about $23$ beam directions per second.  Reducing the Bethe cutoff parameters will decrease the computation time,
at the expense of accuracy of the resulting patterns.

Once the data file has been created, then it is a simple matter to use the IDL Virtual Machine application to 
visualize the result.   The IDL app is located in the VMapps folder of the distribution; simply double clicking the \textsf{ECPDisplay}
application will start up the Virtual Machine; click on the \textsf{Click to Continue} option in the IDL start up screen to 
initialize the app; then follow the guidelines from section~\ref{sec:idl}.

\begin{refsection}[papers]
\renewbibmacro{in:}{}
\nocite{degraef2003b,humphreys1979a,winkelmann2003a,winkelmann2008a,weickenmeier1991a,reimer1985a}%
  \printbibliography[title={References}]
\end{refsection}



\end{document}



