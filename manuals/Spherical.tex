\documentclass[11pt]{amsart}
\usepackage{amsmath}

\textwidth=6.5in
\textheight=9.0in
\oddsidemargin=0.0in
\evensidemargin=0.0in
\topmargin=-0.5in


\title{Thoughts about spherical wave diffraction}
\author{S. Singh, M. De Graef}
%\date{}                                           % Activate to display a given date or no date

\begin{document}
\maketitle

\section{Problem description}
Several of the computational approaches for CBEd, ECP, and EBSD require sampling of 
the diffracted wave on a spherical surface, representing the independent incident 
beam directions.  Each of these beam directions requires the solution of a complete $N$-beam diffraction 
problem, so the computations can conceivably take a rather long time, even when symmetry is 
taken into account.

The question addressed in this document is the following: \textit{is it possible to consider a 
completely different diffraction geometry, namely a spherical source inside a crystal 
represented by a spherical shell?}


\subsection{Representation of the electrostatic lattice potential}
Consider an infinite crystal lattice with potential $V(\mathbf{r})$; assume that this includes absorption.
We know that this potential can be written as a Fourier series:
\begin{equation}
	V(\mathbf{r}) = V_{\mathbf{0}}+\sum_{\mathbf{g}} V_{\mathbf{g}}\,\mathrm{e}^{2\pi\mathrm{i}\mathbf{g}\cdot\mathbf{r}}.
\end{equation}
If we want to represent this series in a spherical coordinate system, then we need to replace the 
plane waves of the Fourier series by their spherical expansions (with $\mathbf{g}=(g,\theta'_g,\phi'_g)$
and $\mathbf{r}=(r,\theta,\phi)$):
\begin{equation}
	\mathrm{e}^{2\pi\mathrm{i}\mathbf{g}\cdot\mathbf{r}} = 
	4\pi \sum_{\ell=0}^{\infty} \mathrm{i}^{\ell}j_{\ell}(2\pi gr)\sum_{m=-\ell}^{+\ell} Y^{\ast}_{lm}(\theta,\phi)Y_{lm}(\theta'_g,\phi'_g).
\end{equation}
All special functions have their usual meanings.

The spherical potential of the infinite crystal can then be written as:
\begin{align}
	V(r,\theta,\phi) - V_{\mathbf{0}} &= 
4\pi \sum_{\mathbf{g}} V_{\mathbf{g}}\sum_{\ell=0}^{\infty} \mathrm{i}^{\ell}j_{\ell}(2\pi gr)\sum_{m=-\ell}^{+\ell} Y^{\ast}_{lm}(\theta,\phi)Y_{lm}(\theta'_g,\phi'_g);\\
&= 4\pi \sum_{\ell=0}^{\infty} \mathrm{i}^{\ell} \sum_{m=-\ell}^{+\ell} Y^{\ast}_{lm}(\theta,\phi) \sum_{\mathbf{g}} V_{\mathbf{g}}\,j_{\ell}(2\pi gr)Y_{lm}(\theta'_g,\phi'_g);\\
&= 4\pi \sum_{\ell=0}^{\infty} \mathrm{i}^{\ell} \sum_{m=-\ell}^{+\ell} Y^{\ast}_{lm}(\theta,\phi) \sum_{s\in\mathsf{S}}\,j_{\ell}(2\pi g_{s}r)\sum_{i=1}^{N_s} V_{s,i}Y_{lm}(\theta'_{s,i},\phi'_{s,i}),
\end{align}
where $\mathsf{S}$ is the set of all possible stars of $\mathbf{g}$-vectors, and $N_s$ the number of elements (multiplicity) in star $s$.
If we introduce the notation 
\begin{equation}
	V_{s;lm} \equiv \sum_{i=1}^{N_s} V_{s,i}Y_{lm}(\theta'_{s,i},\phi'_{s,i})
\end{equation}
then the potential is simplified to:
\begin{equation}
	V(r,\theta,\phi) - V_{\mathbf{0}} = 
4\pi \sum_{\ell=0}^{\infty} \mathrm{i}^{\ell} \sum_{m=-\ell}^{+\ell} Y^{\ast}_{lm}(\theta,\phi) \sum_{s\in\mathsf{S}}\,V_{s;lm} j_{\ell}(2\pi g_{s}r)
\end{equation}

\subsection{Example: fcc}
For fcc, the first star $\{111\}$ has eight members, with potential coefficient $V_{1,i} = V_{111}$, and the spherical angles of the $\mathbf{g}_{\{111\}}$ vectors are 
given by:




\end{document}  