
\documentclass[DIV=calc, paper=letter, fontsize=11pt]{scrartcl}	 % A4 paper and 11pt font size

\usepackage[body={6.5in,9.0in},
  top=1.0in, left=1.0in]{geometry}
  
\usepackage[english]{babel} % English language/hyphenation
\usepackage[protrusion=true,expansion=true]{microtype} % Better typography
\usepackage{amsmath,amsfonts,amsthm} % Math packages
\usepackage[svgnames]{xcolor} % Enabling colors by their 'svgnames'
\usepackage[hang, small,labelfont=bf,up,textfont=it,up]{caption} % Custom captions under/above floats in tables or figures
\usepackage{booktabs} % Horizontal rules in tables
\usepackage{fix-cm}	 % Custom font sizes - used for the initial letter in the document

\usepackage{sectsty} % Enables custom section titles
\allsectionsfont{\usefont{OT1}{phv}{b}{n}} % Change the font of all section commands

\usepackage{fancyhdr} % Needed to define custom headers/footers
\pagestyle{fancy} % Enables the custom headers/footers
\usepackage{lastpage} % Used to determine the number of pages in the document (for "Page X of Total")

\usepackage{fancyvrb}% used to include files verbatim
%\usepackage{chemsym}

\usepackage{hyperref}

\usepackage[backend=bibtex,style=numeric,sorting=ydnt,maxnames=15]{biblatex}
\renewbibmacro{in:}{}

% Count total number of entries in each refsection
\AtDataInput{%
  \csnumgdef{entrycount:\therefsection}{%
    \csuse{entrycount:\therefsection}+1}}

% Print the labelnumber as the total number of entries in the
% current refsection, minus the actual labelnumber, plus one
\DeclareFieldFormat{labelnumber}{\mkbibdesc{#1}}    
\newrobustcmd*{\mkbibdesc}[1]{%
  \number\numexpr\csuse{entrycount:\therefsection}+1-#1\relax}


%\addbibresource[label=papers]{mypubs.bib}
%\addbibresource[label=books]{mypubs.bib}
%\addbibresource[label=edited]{mypubs.bib}
%\addbibresource[label=chapters]{mypubs.bib}


% Headers - all currently empty
\lhead{}
\chead{}
\rhead{}

% Footers
\lfoot{\textsf{EMsoft--2015} manual, Release 3.0, \today}
\cfoot{}
\rfoot{\footnotesize Page \thepage\ of \pageref{LastPage}} % "Page 1 of 2"

\renewcommand{\headrulewidth}{0.0pt} % No header rule
\renewcommand{\footrulewidth}{0.4pt} % Thin footer rule

\usepackage{lettrine} % Package to accentuate the first letter of the text
\newcommand{\initial}[1]{ % Defines the command and style for the first letter
\lettrine[lines=3,lhang=0.3,nindent=0em]{
\color{DarkGoldenrod}
{\textsf{#1}}}{}}

\usepackage{titling} % Allows custom title configuration

\newcommand{\HorRule}{\color{DarkGoldenrod} \rule{\linewidth}{1pt}} % Defines the gold horizontal rule around the title

\pretitle{\vspace{-1.5in} \begin{center} \HorRule \fontsize{25}{25} \usefont{OT1}{phv}{b}{n} \color{DarkRed} \selectfont} % Horizontal rule before the title

\title{EMsoft\\ General User Manual} % Your article title

\posttitle{\par\end{center}\vskip 0.5em} % Whitespace under the title

\preauthor{\begin{center}\large \lineskip 0.5em \usefont{OT1}{phv}{b}{sl} \color{DarkRed}} % Author font configuration

\author{\vspace*{-0.7in}} % Your name

\postauthor{\footnotesize \usefont{OT1}{phv}{m}{sl} \color{Black} % Configuration for the institution name

\par\end{center}\HorRule} % Horizontal rule after the title
\date{Program Manual, Release 3.0, \today}

\newcommand{\ctp}{\textsf{EMsoft}}
%
\begin{document}
\maketitle

\renewcommand{\contentsname}{Table of Contents}
{\small\tableofcontents}

\newpage
\section{Introduction}
This manual describes those aspects of the \ctp\ package that are common to almost all of the programs; it also describes the functionality of 
the library source code files.  The 2013 version of the source code is based on the original CTEMsoft code from the late 90s and the years 
leading up to the publication of the CTEM textbook (2003, Cambridge University Press).  While some of the code was described in the 
book, not everything was subsequently released in the public domain.  In fact, only the code corresponding to Chapters 1 through 7 was 
released; the remainder exists, but due to time limitations it was never released.  The present version (Release 3.0) aims to correct this issue and also
provides a significantly upgraded and updated code base.  The name of the package has also been changed from CTEMsoft to \ctp, to reflect the 
fact that both SEM and TEM codes are part of the library.

At the time of writing of this manual, these programs have been successfully compiled on the Mac OS X platform using the public domain gfortran compiler.  
There is no basic reason why this code should not work on Windows and Linux systems, and efforts are underway
to make sure that the code will run on all platforms. It would be interesting to see some of this code
implemented in a super computer setting, since many of the routines should be quite parallellizable.  Where possible, some of the computations
are carried out using the OpenMP directives, so that multiple cores can be used.  This aspect of the code is still under further development,
and we anticipate that GPUs will become useful as well, using the OpenCL approach.

The \ctp\ package is entirely written in f90 and, since March 2015, also includes commands from the fortran-2003 implementation, in 
particular ISO\_C\_BINDINGS for the HDF modules.

The source code is extensively commented, using regular comment lines, but also using DOxygen documentation generation commands.  Hence
there exists an extensive on-line documentation of all variables, variable types, modules, subroutines, functions, etc. for the latest version 
of the code.  For selected programs, more extensive manual pages are available.  If interested, please contact the author for further information.

\newpage
\section{The EMsoft library and third party libraries\label{sec:library}}
The core of the \ctp\ package is the static library \textsf{EMsoftlib.a}.  This library contains all the code segments and variable/type definitions
that are used by the programs.  The library must be compiled first, before any program can be compiled.  The source code is organized 
in a series of modules, starting with \textsf{src/local.f90} which is/should be the only routine that might need some editing before the library is 
compiled.  

\ctp\ also makes use of a number of third party libraries and source files, all of them in the public domain.  The most important ones are 
the LAPACK and associated BLAS libraries, as well as the HDF5 library.  With the help of Mike Jackson and Megna Shah (BlueQuartz.net), the 
compilation files from version 1.0 have been reworked from old-fashioned
makefiles to use the more flexible CMake environment; to compile the \ctp\ you must have the following installed on your system:




\newpage
\section{Things to do before running any of the programs\label{sec:todo}}
The programs require that three environment variables be set: \textsf{EMsofttemplates},
\textsf{EMsoftresources} and \textsf{EMsoftopencl}.  Depending on the UNIX shell that one is using, they must be set in
the appropriate shell start-up files.

\begin{itemize}
\item \textit{csh}: edit the \textsf{.cshrc} file in your home folder and add the following lines:
\begin{verbatim}
  setenv EMsofttemplates /some/absolute/path/EMsoft/templatefolder
  setenv EMsoftresources /some/absolute/path/EMsoft/resources
  setenv EMsoftopencl /som/absolute/path/EMsoft/opencl
\end{verbatim}
\item \textit{bash}: edit the \textsf{.bash\_profile} file in your home folder and add the following lines:
\begin{verbatim}
  EMsofttemplates=/some/absolute/path/EMsoft/templatefolder
  EMsoftresources=/some/absolute/path/EMsoft/resources
  EMsoftopencl=/some/absolute/path/EMsoft/opencl
  export EMsofttemplates EMsoftresources EMsoftopencl
\end{verbatim}
\end{itemize}
Make sure that you replace the string \textsf{/some/absolute/path} by the correct path for your system.
For other shells, please use the appropriate approach to defining shell variables.

{\color{blue}How is this done on the PC platform?}

\newpage
\section{The general structure of EMsoft programs\label{sec:structure}}
Each \ctp\ program has the same high level source code structure, and is called in the same way.  If \textsf{EMprogram} represents 
a program name (for instance \textsf{EMECP} or \textsf{EMlacbed}, then the following command line options can be used (all of them optional):
\begin{verbatim}
	EMprogram [-h] [-t] [file.nml]
\end{verbatim}
The arguments  between square brackets are optional, and are defined as follows:
\begin{itemize}
	\item \textsf{-h}: (h=help) this argument causes the program to display a list of all command line arguments and their meaning.  The program will quit after printing 
	the help message.
	\item \textsf{-t}: (t=template) this argument causes the program to create template files for all the input files used by the program.  For each
	name-value entry in the template file (see section~\ref{sec:f90input} for more details), a comment line is inserted with a brief explanation 
	of the variable and its units, if appropriate.  The user can then copy the template file to a new file with the namelist (.nml) extension and edit the 
	file with a regular text editor.  The program will quit after generating the template files.  Each template file contains the default values for each parameter.
	If this option produces an error, then this is likely due to an incorrect setting of the shell variables in the previous section.
	\item \textsf{file.nml}: the main namelist file to be used by the program.  If no name is present, then the default filename CTEMprogram.nml will be used.  If the 
	provided namelist file does not exist, the program will report an error and abort.
\end{itemize}

The programs in the top half of the table on the last page of this manual function in the way described above; all other programs
have a simple command line interface, with user prompts for all relevant variables.  Eventually, in the next Release, all programs
will make use of namelist files and will likely be callable from a single user interface.


\newpage
\section{Namelist input files\label{sec:f90input}}
All user input for the \ctp\ programs is performed via namelist files; this is a special f90 IO format that allows one to list 
\textit{name-value} pairs in an ordinary text file, and then this file will be read and interpreted by the 
program.  The basic format of a namelist file is as follows:
\begin{verbatim}
&nmlname
var1 = val1
var2 = val2
...
/
\end{verbatim}
The first line must begin with the ampersand character ``\&'' and is followed (without spaces) by the internal name used by the program 
to identify the set of variables (this is similar to a \textit{common block} in the older f77 standard).  After the mandatory first line, which should not be altered, a 
list of name-value pairs follows.  These can appear in any order, and variables may be omitted from the file, in which case the program
will use a default value (this value can be found in the template file; see section~\ref{sec:structure}).  The final line of the file must contain a single forward slash
character, ``/'', to end the namelist.

As a concrete example, consider the namelist file used to define a dislocation:
\begin{Verbatim}%
	[formatcom=\color{blue},frame=lines]
&dislocationdata
id = 0.501
jd = 0.501
u = 1.0,0.0,1.0
bv = 0.5,0.0,0.5
/
\end{Verbatim}
Internally, the program identifies this set of variables with the namelist identifier ``dislocationdata,'' as shown on the first line of the file.
Then there are four name-value pairs: \textsf{id} and \textsf{jd} are real numbers in the range $[-1,1]$ and define the position
of the defect; \textsf{u} represents the line direction, which is declared as a triplet of 
floating point numbers; and the burgers vector is represented by another triplet of floating point numbers and the variable name \textsf{bv}.
The forward slash closes the file.  Note that these entries can appear in any order. Comment lines are allowed in the namelist
file; simply start the line with the f90 comment character (the exclamation mark ``!'') and then type your comment.  For variables that 
have components, such as the two vectors above, one may also define each component separately, in which case the file would look
as follows:
\begin{verbatim}
&dislocationdata
id = 0.501
jd = 0.501
! these vectors are defined one component at a time
u(1) = 1.0 
u(2) = 0.0 
u(3) = 1.0
bv(1) = 0.5
bv(2) = 0.0
bv(3) = 0.5
/
\end{verbatim}
It is not necessary to leave spaces before and after the ``='' symbol, but it does improve the readability of the namelist file to do so. Since 
one can always recover the template file(s) by using the [-t] option when calling the program, there is no need to list all the name-value pairs
in the namelist file; pairs which are not listed will take on their default value(s), which are the ones listed in the template file.

One important thing to note: when entering file names and other strings in name list files, one must use single quotes both at the start and at 
the end of the string.  Some word processing programs have so-called ``smart quotes'' settings, which will introduce a different kind of quote,
and the fortran programs will not recognize smart quotes as being the same as single quotes.  Therefore, turn of the ``smart quotes'' option
in your editor before you start editing namelist files.


\section{Generating a crystal structure file\label{sec:f90input5}}

With very few exceptions, all programs in the \ctp\ suite of programs require a crystallographic input file that defines the 
crystal system, crystal lattice parameters, space group number, and all unique positions in the 
asymmetric unit along with site occupations and Debye-Waller parameters.  To keep things simple and 
portable across platforms, structure description files (*.xtal) are small binary files that contain the minimum
amount of information needed to unambiguously define the complete unit cell.  In a later version of the package, this
will be replaced by a more standard file format, perhaps one that can be created by the CrystalMaker program.

To create a crystal structure description file, one uses the \textsf{EMmkxtal} program, which will prompt
the user for the crystal system number, the lattice parameters (in nm and degrees), the space group number,
and the atom positions in the asymmetric unit.  Each atom entry is defined by five numbers: the fractional
coordinates (which can be entered as real numbers or as fraction, e.g., $0.5$ or $1/2$), the site 
occupation parameter (a real number between $0$ and $1$), and the isotropic Debye-Waller factor in nm$^{2}$.

Once all atoms in the asymmetric unit have been entered, the program will ask for a structure file name; we
recommend that all such files have the extension \textsf{.xtal}, to make them easily recognizable.  You can also
create a folder that holds all the structure files, and then use an absolute or relative path name to store 
the file in that folder.

%
%With very few exceptions, all programs in the \ctp\ suite of programs require a crystallographic input file that defines the 
%crystal system, crystal lattice parameters, space group number, and all unique positions in the 
%asymmetric unit along with site occupations and Debye-Waller parameters.  To keep things simple and 
%portable across platforms, structure description files (*.ctem) are simple editable text files that contain the minimum
%amount of information needed to unambiguously define the complete unit cell.
%
%The crystal structure definition file format is best illustrated with a few examples.  Consider first a simple face-centered 
%cubic unit cell for copper:
%\fvset{frame=lines,formatcom=\color{blue},fontsize=\footnotesize,numbers=left}
%%\VerbatimInput{../resources/Cu.xtal}
%\begin{itemize}
%\item The first line contains the crystal system number (1 integer), using the following values:
%cubic=1, tetragonal=2, orthorhombic=3, hexagonal=4, trigonal=5, monoclinic=6, and triclinic=7.
%\item The second line contains the space group number (1 integer), potentially followed by a second integer (see later).
%\item The third line lists the lattice parameters $\{a,b,c,\alpha,\beta,\gamma\}$ (6 reals; lengths in  nm; angles in degrees);
%\item The fourth line contains the number of atoms in the asymmetric unit (1 integer);
%\item The remaining lines (as many as indicated on line 4) each have six numbers, the first one an integer, the others reals.
%The first number is the atomic number, followed by three fractional coordinates, the site occupation number (in the interval $[0,1]$),
%and the isotropic Debye-Waller factor (in nm$^2$).  The fractional coordinates may be entered either as reals (e.g., $0.23$) or as fractions
%(e.g., $1/3$, $7/12$, etc.).
%\end{itemize}
%The structure file can be generated manually, or it can be generated by means of the \textsf{CTEMxtal} program, which prompts the 
%user for all relevant parameters and then creates the file in the appropriate format.
%
%A more complex example is the structure of tetragonal BaTiO$_3$:
%\fvset{frame=lines,formatcom=\color{blue},fontsize=\footnotesize,numbers=left}
%\VerbatimInput{../xtal/BaTiO3.xtal}
%Note that all lines after the last atom position line are ignored by the \ctp\ programs, so that one can add any relevant comment lines. 
%
%For space groups with two origin settings, special care is needed when defining the space group number.
%
%Finally, for trigonal space groups, one can define the lattice parameters in terms of the rhombohedral unit cell or a hexagonal unit 
%cell.  The \ctp\ programs will check the value of the $\gamma$ unit cell angle; if $\gamma=120^{\circ}$, then the hexagonal setting
%will be taken, otherwise the rhombohedral unit cell will be assumed.
%

\section{List of programs}
The list below shows all the available programs in the current Release (2.0) of the \ctp\ suite, along with whether or not there is a 
separate manual for each program.  Programs for which no manual is available are generally relatively easy to figure out.
In the current release, many programs still have a command-line interaction with the user; this will be modified
in the next release so that all programs will eventually make use of namelist files.  The top half of the table
lists those programs for which an extensive manual is available, as well as IDL visualization routines. The \textsf{EMmkxtal}
program is explained in the present document.  The other programs do not yet have manual pages, but are generally 
simple to figure out.  Many of the programs in the bottom half of the table are simple illustrations of how 
to perform certain computations.

%All it takes is to run the program with the \textit{-t} option, to create a template namelist file in the current folder;
%you can then edit this file to set the appropriate parameters, and 

\begin{table*}[h]
\centering\leavevmode
\begin{tabular}{|l|l|c|}
\hline
Program & Short description & Manual? \\
\hline\hline
  \textsf{EMlacbed} & large angle CBED pattern computation (TEM)& $\checkmark$ \\
  \textsf{EMmbcbed} & regular CBED pattern (TEM)& $\checkmark$\\
  \textsf{EMSRdefect} & (S)TEM systematic row defect images (TEM) & $\checkmark$\\
  \textsf{EMZAdefect} & (S)TEM zone axis defect images (TEM) & $\checkmark$\\
  \textsf{EMECP} & electron channeling pattern (SEM) & $\checkmark$\\
  \textsf{EMECCI} & ECCI defect images (SEM) & $\checkmark$\\
  \textsf{EMKossel} & electron Kossel patterns (SEM) & $\checkmark$\\
  \textsf{EMMC} & Monte Carlo trajectory simulations (SEM) & $\checkmark$\\
  \textsf{EMEBSDmaster} & EBSD master pattern simulation (SEM) & $\checkmark$\\
  \textsf{EMEBSD} & EBSD pattern simulation (SEM) & $\checkmark$\\
\hline
  \textsf{EMmkxtal} & make a crystal structure input file& $\checkmark$\\
  \textsf{EMlistSG} & list the equivalent positions of any space group & \\
  \textsf{EMqg} & list Fourier coefficient information & \\
  \textsf{EMfamily} & draw a family of planes (stereographic) & \\
  \textsf{EMstar} & list the star of any reciprocal lattice point & \\
  \textsf{EMorbit} & list the orbit of a point & \\
  \textsf{EMZAgeom} & print zone axis geometry and symmetry information & \\
  \textsf{EMlatgeom} & perform some simple lattice geometry calculations & \\
  \textsf{EMstereo} & basic stereographic prpjection tool & \\
  \textsf{EMorient} & stereographic projection for orientation relation & \\
  \textsf{EMxtalinfo} & makes postscript file with lots of information & \\
  \textsf{EMzap} & create postscript file with zone axis patterns & \\
  \textsf{EMdrawcell} & draw a unit cell (very primitive) & \\
\hline
\end{tabular}
\end{table*}





\end{document}



