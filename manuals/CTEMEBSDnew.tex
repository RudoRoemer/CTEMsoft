
\documentclass[DIV=calc, paper=letter, fontsize=11pt]{scrartcl}	 % A4 paper and 11pt font size

\usepackage[body={6.5in,9.0in},
  top=1.0in, left=1.0in]{geometry}
  
\usepackage[english]{babel} % English language/hyphenation
\usepackage[protrusion=true,expansion=true]{microtype} % Better typography
\usepackage{amsmath,amsfonts,amsthm} % Math packages
\usepackage[svgnames]{xcolor} % Enabling colors by their 'svgnames'
\usepackage[hang, small,labelfont=bf,up,textfont=it,up]{caption} % Custom captions under/above floats in tables or figures
\usepackage{booktabs} % Horizontal rules in tables
\usepackage{fix-cm}	 % Custom font sizes - used for the initial letter in the document
\usepackage{epsfig}
\usepackage{sectsty} % Enables custom section titles
\allsectionsfont{\usefont{OT1}{phv}{b}{n}} % Change the font of all section commands

\usepackage{fancyhdr} % Needed to define custom headers/footers
\pagestyle{fancy} % Enables the custom headers/footers
\usepackage{lastpage} % Used to determine the number of pages in the document (for "Page X of Total")
\usepackage{color}

\usepackage{fancyvrb}% used to include files verbatim
%\usepackage{chemsym}

\usepackage{hyperref}

\usepackage[backend=bibtex,style=numeric,sorting=ydnt,maxnames=15]{biblatex}
\renewbibmacro{in:}{}

% Count total number of entries in each refsection
\AtDataInput{%
  \csnumgdef{entrycount:\therefsection}{%
    \csuse{entrycount:\therefsection}+1}}

% Print the labelnumber as the total number of entries in the
% current refsection, minus the actual labelnumber, plus one
\DeclareFieldFormat{labelnumber}{\mkbibdesc{#1}}    
\newrobustcmd*{\mkbibdesc}[1]{%
  \number\numexpr\csuse{entrycount:\therefsection}+1-#1\relax}


%\addbibresource[label=papers]{mypubs.bib}
%\addbibresource[label=books]{mypubs.bib}
%\addbibresource[label=edited]{mypubs.bib}
%\addbibresource[label=chapters]{mypubs.bib}


% Headers - all currently empty
\lhead{}
\chead{}
\rhead{}

% Footers
\lfoot{\textsf{CTEMMC, CTEMEBSDmaster, and CTEMEBSD} manual, v2.0, \today}
\cfoot{}
\rfoot{\footnotesize Page \thepage\ of \pageref{LastPage}} % "Page 1 of 2"

\renewcommand{\headrulewidth}{0.0pt} % No header rule
\renewcommand{\footrulewidth}{0.4pt} % Thin footer rule

\usepackage{lettrine} % Package to accentuate the first letter of the text
\newcommand{\initial}[1]{ % Defines the command and style for the first letter
\lettrine[lines=3,lhang=0.3,nindent=0em]{
\color{DarkGoldenrod}
{\textsf{#1}}}{}}

\usepackage{titling} % Allows custom title configuration

\newcommand{\HorRule}{\color{DarkGoldenrod} \rule{\linewidth}{1pt}} % Defines the gold horizontal rule around the title

\pretitle{\vspace{-1.5in} \begin{center} \HorRule \fontsize{25}{25} \usefont{OT1}{phv}{b}{n} \color{DarkRed} \selectfont} % Horizontal rule before the title

\title{Electron Back-Scatter Diffraction Pattern Simulations} % Your article title

\posttitle{\par\end{center}\vskip 0.5em} % Whitespace under the title

\preauthor{\begin{center}\large \lineskip 0.5em \usefont{OT1}{phv}{b}{sl} \color{DarkRed}} % Author font configuration

\author{\vspace*{-0.7in}} % Your name

\postauthor{\footnotesize \usefont{OT1}{phv}{m}{sl} \color{Black} % Configuration for the institution name

\par\end{center}\HorRule} % Horizontal rule after the title
\date{Program Manual, v2.0, \today\protect\footnote{This set of programs was developed with financial support from two agencies. 
The Office of Naval Research sponsored the development of the f90 source code for computation of EBSD patterns in research 
grant N00014-12-1-0075.  The IDL visualization interface, and the complete version 2.0 and beyond were developed with financial 
support from an AFOSR/MURI grant FA9550-12-1-0458.}}

\newcommand{\ctp}{\textsf{CTEMsoft-2013}}
%
\newcommand{\upg}[1]{\mathrm{i}U_{\mathbf{#1}}^{\prime}}
\newcommand{\combo}[1]{U_{\mathbf{#1}}+\upg{#1}}
\newcommand{\upgcombo}[2]{2k_{0}s_{\mathbf{#1}}+\upg{#2}}
\newcommand{\ugh}[2]{U_{\mathbf{#1}-\mathbf{#2}}}
\newcommand{\ughp}[2]{U_{\mathbf{#1}'-\mathbf{#2}}}
\newcommand{\ughpp}[2]{U_{\mathbf{#1}-\mathbf{#2}'}}
\newcommand{\kkg}[1]{k_{0}^{2}-(\mathbf{k}+\mathbf{#1})^{2}}
\newcommand{\Cg}[1]{C_{\mathbf{#1}}}
\newcommand{\Cgj}[2]{C_{\mathbf{#1}}^{(#2)}}
\newcommand{\Cgjp}[2]{C_{\mathbf{#1}'}^{(#2)}}
\newcommand{\Cgja}[2]{C_{\mathbf{#1}}^{(#2)\ast}}
\newcommand{\button}[1]{\colorbox{green}{\textsf{#1}} button}


\begin{document}
\maketitle

\begin{figure*}[h]
\leavevmode\centering
\epsffile{figs/SEMONRlogo}
\end{figure*}

\renewcommand{\contentsname}{Table of Contents}
{\small\tableofcontents}

\newpage
\section{Introduction}
This manual describes a series of four programs, 
three written in Fortran-90,\footnote{f90 is a much richer language than the original fortran-f77, and is
used for all programs in the \ctp\ package.} 
the other in IDL,\footnote{The \textit{Interactive Data Language} is an interpreted scripting language with extensive graphics capabilities.} 
that can be used for the simulation of electron back-scatter diffraction (EBSD) patterns (EBSPs).  The main f90 programs
are named \textsf{CTEMMC} (note that all programs in the \ctp\ package start with the letters ``CTEM''), \textsf{CTEMEBSDmaster},
and \textsf{CTEMEBSD}.
The output generated by these programs can then be visualized by the IDL routine \textsf{EBSDDisplay.pro}.  

On the following pages we will try to accomplish four tasks:
\begin{enumerate}
	\item Explain briefly the underlying pattern formation theory and the numerical approach followed by the f90 programs (section~\ref{sec:theory});
	\item Document the input files for the three f90 programs (sections~\ref{sec:f90MC}, \ref{sec:f90EBSDmaster}, and \ref{sec:f90EBSD});
	\item Document the IDL interface (section~\ref{sec:idl});
	\item Explain the use of these programs by means of a few basic examples (section~\ref{sec:examples}).
\end{enumerate}

At the time of writing of this manual, these programs have been successfully compiled on the Mac OS X platform using the commercial \textit{Absoft} compiler,
as well as the public domain gfortran compiler.  There is no basic reason why this code should not work on Windows and Linux systems, and efforts are underway
to make sure that the code will run on all platforms. It would be interesting to see some of this code
implemented in a super computer setting, since many of the routines should be quite parallellizable.  Where possible, some of the computations
are carried out using the OpenMP directives, so that multiple cores can be used.  This aspect of the code is still under further development,
and we anticipate that GPUs will become useful as well, using the OpenCL approach.

The \ctp\ package is entirely written in f90 and does not use any of the newer commands available in the 95 and 2013 versions.
The source code is extensively commented, using regular comment lines, but also using DOxygen documentation generation commands.  Hence
there exists an extensive on-line documentation of all variables, variable types, modules, subroutines, functions, etc. for the latest version 
of the code.  For selected programs, more extensive manual pages are available.  If interested, please contact the author for further information.

The visualization part of the code consists of a series of  IDL routines that are available as source code or in the form of a Virtual Machine application. 
If you have an IDL license, then you can compile and run the IDL source code; alternatively, if you do not have a license,
then you can use one of the VM apps to perform the same task.  Note that in the VM environment, you will not be able to alter/compile the 
source code.


\newpage
\section{Version 2.0 features}
In version 2.x, we have made the following changes:
\begin{itemize}
	\item \textsf{CTEMMC} now uses an OpenCL kernel routine to perform the Monte Carlo simulation on a GPU, if one is available.  This results in significant acceleration compared to the single CPU or multi-core CPU routines from version 1.0.  Speed up factors of between $200\times$ and $300\times$ have been measured with respect to the single CPU code.
	\item \textsf{CTEMEBSDmaster} has been rewritten using the scattering matrix approach instead of the more traditional Bloch wave approach.  As a result, the bulk of the computation can once again be shifted onto a GPU; the GPU computes the exponential of the dynamical matrix using the modified/optimized Taylor expansion approach created by C. Koch and co-workers.
	\item \textsf{CTEMEBSD}, for now, remains unchanged, except for the symmetry handling which now includes full support for the hexagonal Lambert grid.
	\item \textsf{EBSDDisplay.pro}: this IDL program remains unchanged for the most, but will be replaced in the future with a \textsf{wx}-Python interface.
\end{itemize}

\newpage
\section{Electron back-scatter diffraction patterns: the underlying theory\label{sec:theory}}
As described in detail in a recent paper,\footnote{P.G. Callahan and M. De Graef, \textit{Dynamical Electron Backscatter Diffraction Patterns. Part I: Pattern Simulations}
Microsc.\ Microanal.\ \textbf{19}, 1255--1265, 2013.} there are three steps to the computation of a realistic (dynamical) EBSP:
\begin{itemize}
	\item Monte Carlo simulation of the energy, depth, and directional distributions of back-scattered electrons;
	\item Dynamical simulation of the EBSD master pattern (covering all possible directions);
	\item Simulation of an EBSP for a given detector geometry and sample (grain) orientation.
\end{itemize}

Before we describe the details of these three steps, we need to introduce a mapping and interpolation technique that is used extensively in all
three programs as well as in the visualization program: \textit{modified Lambert projections}.  We also need to discuss the 
symmetry of the scattering problem in terms of the Laue symmetry groups. For projections from hexagonal or trigonal symmetries, we 
also need to introduce a special version of barycentric coordinates.

\subsection{Modified Lambert Projections \label{sec:Lambert}}
The BSEs that leave a sample that is illuminated by a high energy electron beam typically travel in all possible directions in the hemisphere
on the vacuum side of the sample.  If we wish to compute and accurately represent this spatial distribution, we must define a sampling grid 
on the sphere surface for which all bins have very nearly the same shape and area, i.e., a uniform distribution of sampling points 
on the hemisphere surface.  This is not an easy problem, and there is quite a bit of literature on spherical sampling grids.  In addition 
to selecting an efficient sampling scheme, we must also make sure that the scheme is compatible with crystallographic symmetries; if a crystal
has hexagonal symmetry, then we should only compute quantities for a $60^{\circ}$ wedge of orientations, and copy them into the other symmetrically
equivalent wedges.  Such a copy action will require interpolations if the numerical grid is based on a square sampling, since it is not possible
to accurately represent six-fold symmetry on a square grid.  

To accommodate all possible crystal symmetries, we have opted for a mapping scheme between a square grid and a hemisphere derived by D.\ 
Ro\c{s}ca.\footnote{Daniela Ro\c{s}ca, \textit{New uniform grids on the sphere}, Astronomy \& Astrophysics, \textbf{520}, 
id.\ A63 (DOI: 10.1051/0004-6361/201015278).}
To also accommodate hexagonal symmetry, we have derived a new mapping scheme that was published recently.\footnote{D. Ro\c{s}ca and M. De Graef, 
\textit{Area-preserving projections from hexagonal and triangular domains to the sphere and applications to electron back-scatter diffraction pattern simulations}, 
Modeling and Simulations in Materials Science and Engineering, \textbf{21}, 055021 (2013).}  In this section we briefly discuss both mappings, as well 
as their implementation, taking into account the Laue crystal symmetries.

The main goal of the modified Lambert mappings is to create a bi-directional mapping that preserves the area between either a square grid or a hexagonal
grid in the plane, and uniform grids on the surface of a sphere.  
We start from a square grid with semi edge length $L=\sqrt{\pi/2}$ and grid coordinates $(a,b)$.  This point is mapped onto a point $(x,y,z)$ 
in the Northern hemisphere (i.e., with $z\ge 0$) of a sphere with unit radius by the following relations:
\begin{equation}
	(x,y,z) = \left\{\begin{array}{lcl}
	\left(\frac{2a}{\pi}\sqrt{\pi-a^2}\cos\frac{b\pi}{4a},\frac{2a}{\pi}\sqrt{\pi-a^2}\sin\frac{b\pi}{4a},1-\frac{2a^2}{\pi}\right) & \quad & 0 < \vert b\vert \le\vert a\vert\le L;\\
	\left(\frac{2b}{\pi}\sqrt{\pi-b^2}\sin\frac{a\pi}{4b},\frac{2b}{\pi}\sqrt{\pi-b^2}\cos\frac{a\pi}{4b},1-\frac{2b^2}{\pi}\right) & \quad & 0 < \vert a\vert \le\vert b\vert\le L.
	\end{array}\right.
\end{equation}
The inverse relation is given by:
\begin{equation}
	(a,b) = \left\{\begin{array}{lcl}
	\text{sign}(x) \sqrt{2(1-z)}\left(\frac{\sqrt{\pi}}{2},\frac{2}{\sqrt{\pi}}\arctan\frac{y}{x}\right) & \quad & 0\le\vert y\vert\le\vert x\vert;\\
	\text{sign}(y) \sqrt{2(1-z)}\left(\frac{2}{\sqrt{\pi}}\arctan\frac{x}{y},\frac{\sqrt{\pi}}{2}\right) & \quad & 0<\vert x\vert\le\vert y\vert.
	\end{array}\right.
\end{equation}
These mappings are readily implemented and provide a bi-directional transformation between a 2-D square grid and a uniform grid on the surface of a sphere.
The triplets $(x,y,z)$ can be regarded as direction cosines (since $x^2+y^2+z^2=1$), so that we can represent a uniform sampling of beam directions by
a sampling on a square grid.  An additional advantage of the uniform character of the mapping is that we can replace interpolation on the 
surface of a sphere by simple bi-linear interpolation on a square grid, which is easily implemented numerically.
For numerical purposes, we will subdivide the square of edge length $2L=\sqrt{2\pi}$ into $(2N+1)\times (2N+1)$ grid points with a step size
$\Delta a=\Delta b=\sqrt{\pi/2}/N$; note that the point $(a,b)=(0,0)$ maps onto the North pole $(0,0,1)$.  The origin of the square grid lies
at the center of the square; an arbitrary grid point then has coordinates $(i\Delta a, j\Delta b)$, with $-N\le i,j \le +N$ ($i,j\in\mathbb{Z}$).  Points along 
the outer perimeter of the square are mapped onto the equatorial circle of the Northern hemisphere.

For the hexagonal grid, we consider the sextants $I_k$, $k=0,\ldots, 5$, which are defined
as follows (see Figure \ref{fig:hextest}): 
\begin{eqnarray}
    I_0 &=& \left\{ (x,y)\in \mathbb{R}^2, 0\le x,\ -x/\sqrt{3}\le y\le x/\sqrt{3}\right\},\\
    I_1 &=& \left\{ (x,y)\in \mathbb{R}^2, 0\le x,\  x/\sqrt{3}\le y \right\},\\
    I_2 &=& \left\{ (x,y)\in \mathbb{R}^2, x\le 0,-x/\sqrt{3}\le y \right\},\\
    I_3 &=& \left\{ (x,y)\in \mathbb{R}^2, x\le 0,\ x/\sqrt{3}\le y\le -x\sqrt{3}\right\},\\
    I_4 &=& \left\{ (x,y)\in \mathbb{R}^2, x\le 0,\ y\le x/\sqrt{3} \right\},\\
    I_5 &=& \left\{ (x,y)\in \mathbb{R}^2, 0\le x,\ y\le -x/\sqrt{3} \right\}.
\end{eqnarray}
For $(x,y)\in H_a \cap I_k$ (where $H_a$ is the hexagon) one has $x\leq \alpha $ if $k=0,1,5$
and $x\geq -\alpha$ if $k=2,3,4$, where $\alpha=a\sqrt 3/2.$

\begin{figure}[h]
\centering
\includegraphics[width=5in]{figs/hextest}
\caption{(a) Subdivision of the hexagon into sextants; the circle has the same area as the hexagon; (b) 
shows a uniform grid in the hexagon, and (c) shows the same grid after equal-area mapping onto the circle.} \label{fig:hextest}
\end{figure}

The point $(x,y)\neq(0,0)$ in the circle is mapped into $(X,Y)_H$ in the hexagon with coordinates given by
\begin{itemize}
\item For $(x,y)\in I_0 \cup I_3$,
\begin{equation}\label{a03}
 (X,Y)_H=3^{\frac 14}{\sqrt{\frac 2\pi}}\, x \left(
\cos\frac{y\pi}{2\sqrt{3}x},\sin\frac{y\pi}{2\sqrt{3}x}\right);
\end{equation}
\item For $(x,y)\in I_1 \cup I_4$,
\begin{equation}\label{a14}
 (X,Y)_H = \frac{3^{\frac 14}}{\sqrt{2\pi}} (x+\sqrt 3y) \left(
\sin\frac{2\pi x}{3(x+\sqrt{3}y)},\ \cos\frac{2\pi
x}{3(x+\sqrt{3}y)}\right);
\end{equation}
\item For $(x,y)\in I_2 \cup I_5$,
\begin{equation}\label{a25}
 (X,Y)_H = \frac{3^{\frac 14}}{\sqrt{2\pi}} (x-\sqrt 3y) \left(
\sin\frac{2\pi x}{3(x-\sqrt{3}y)},-\cos\frac{2\pi
x}{3(x-\sqrt{3}y)}\right).
\end{equation}
\end{itemize}
For the origin we define $\mathcal T_H(0,0)=(0,0)$.  The inverse transformation is given by:
\begin{itemize}
\item For $(X,Y)\in I_0\cup I_3$
\begin{equation}\label{f03}
 (x,y)=3^{-\frac 14}\sqrt {X^2+Y^2}\,\mbox{sign}(X)\left(
\sqrt{\frac \pi 2},\,\sqrt{\frac 6\pi} \arctan \frac YX \right);
\end{equation}
\item For $(X,Y)\in I_1\cup I_4$,
\begin{equation}\label{f14}
 (x,y)=\frac{3^\frac 14\sqrt {X^2+Y^2}}{\sqrt
{2\pi}}\,\mbox{sign}(X)\left(
\sqrt3\Big(\frac \pi 6-\arctan \frac{Y-\sqrt 3 X}{X+\sqrt 3 Y}\,\Big),
\frac \pi 2+\arctan \frac{Y-\sqrt 3 X}{X+\sqrt 3 Y} \right);
\end{equation}
\item For $(X,Y)\in I_2\cup I_5$,
\begin{equation}\label{f25}
 (x,y)=\frac{3^\frac 14\sqrt {X^2+Y^2}}{\sqrt
{2\pi}}\,\mbox{sign}(X)\left(
\sqrt3\Big(\frac \pi 6+\arctan \frac{Y+\sqrt 3 X}{X-\sqrt 3 Y}\,\Big),
-\frac \pi 2+\arctan \frac{Y+\sqrt 3 X}{X-\sqrt 3Y} \right).
\end{equation}
\end{itemize}
The coordinates in the circle can then be transported onto the Northern unit hemisphere by the 
following inverse Lambert projection:
\begin{equation}
 (\widetilde{x},\widetilde{y},\widetilde{z}) =
\left(\frac{x}{2}\sqrt{4-(x^2+y^2)} ,\frac{y}{2}\sqrt{4-(x^2+y^2)}, 1-\frac{1}{2}(x^2+y^2)\right).\label{eq:lama}
\end{equation}
This relation assumes that the radius of the circle is defined as $R=\sqrt{2}$.
The Lambert projection from the hemisphere to the circle is then given by:
\begin{equation}
    (x,y) = \sqrt{\frac{2}{1+\widetilde{z}}} \left(\widetilde{x},\widetilde{y} \right);\label{eq:lamb}
\end{equation}
For all these relations to work together, we must also have $a=2\sqrt \pi 3^{-3/4}$ for the hexagon edge
length, and therefore $\alpha=3^{1/4}\sqrt{\pi}$ (see Fig.~\ref{fig:hextest}).  This completes the discussion 
of the equal area projections used in the EBSD software.

\subsection{Laue Symmetry \label{sec:Laue}}

There are $11$ Laue point group symmetries, and each of them will require a dedicated sampling scheme, using the approach described above.
Fig.~\ref{fig:Laue-hemispheres} shows the nine unique asymmetric units for the $11$ Laue groups; six of them will be sampled on a square grid 
(orange units), the other $3$ on a hexagonal grid.  The symbols labeling each stereographic circle represent the Laue point group,
followed by the point group order.  

\begin{figure}[t]
\centering\leavevmode
\epsffile{figs/Laue-hemispheres}
\caption{\label{fig:Laue-hemispheres}Asymmetric units for the $11$ Laue groups, drawn on stereographic projections.}
\end{figure}

The real space direction $\mathbf{a}_1$ is parellel to the $X$ axis of the sampling grid, and the reciprocal $\mathbf{a}_3^{\ast}$ direction
is normal to the sampling grid.  All other directions are then uniquely fixed.  The square grid has its own planar symmetry group $4mm$,
which is compatible with all of the Laue groups on the top two rows.  If we consider the Laue group $m\bar{3}$, then 
we will run into a problem with the implementation of the three-fold rotation axis.  The points that are equivalent with the simulated points
inside the shaded area will not, in general, fall on grid points.  Because of this, we must perform the EBSD simulations for all grid points
corresponding to an octant of the projection sphere for all Laue symmetries of $mmm$, $4/m$ and $m\bar{3}$, resulting in a symmetry speed-up by a factor $4$.  For Laue groups 
$4/mmm$ and $m\bar{3}m$, we use the $45^{\circ}$ wedge shape, resulting in a speed-up by a factor of $8$.  For the triclinic and monoclinic cases we use the entire hemisphere
or half a hemisphere, respectively.
For the hexagonal sampling grid, the proper choices are obvious from the drawing, and make use of the hexagonal modified Lambert mapping described
in the previous section.

\subsection{Barycentric Coordinates \label{sec:barycentric}}
For the hexagonal and trigonal Laue symmetries, the \textsf{CTEMEBSDmaster} program samples the electron beam directions on 
a hexagonal grid, so that the symmetry elements can be implemented properly.  Subsequently, this grid is transformed to a standard
square grid for further processing in the \textsf{CTEMEBSD} program.  The transformation is performed using barycentric coordinates\footnote{See, for
instance, \textit{http://en.wikipedia.org/wiki/Barycentric\_coordinate\_system} or \textit{http://mathworld.wolfram.com/ BarycentricCoordinates.html} for details.}
for the interpolation.  Referring to the hexagonal and square grids in Fig.~\ref{fig:bary}, 
the interpolated intensity $I_p(h)$ in a point that lies a 
distance $h$ above the base $1$--$2$ of a triangle with top $3$ is given by:
\[
	I_p(h) = (I_1+I_2)\lambda+(1-2\lambda) I_3\quad\text{with}\quad \lambda=\frac{1}{2} - \frac{h}{\delta\sqrt{3}},
\]
where $\delta$ is the step size along the horizontal direction.  The hexagonal grid points $(i,j)$ then 
lie on cartesian positions with coordinates $(x,y)=(i_h\delta-j_h\delta/2,j_h\delta\sqrt{3}/2)$.  With the exception of 
the grid points with $j_h=0$, the value of $h$ is always non-zero.  It is not too difficult to distinguish between
triangles with apex $3$ pointing upward or downward.

\begin{figure}[h]
\centering
\includegraphics[width=4in]{figs/bary}
\caption{Correspondence between hexagonal and square sampling grids used for barycentric interpolation.} \label{fig:bary}
\end{figure}

Summarizing, the master pattern for structures with hexagonal or trigonal symmetry is computed by sampling incident beam directions
on a hexagonal grid (limited to the corresponding asymmetric unit); these grid points correspond to incident beam directions according 
to the modified Lambert mapping introduced in section~\ref{sec:Lambert}.  Once all EBSD intensities have been computed, the symmetry
elements are applied, and the hexagonal grid is interpolated onto a standard square grid using the barycentric approach described above.
From then on, there is no further need for a distinction between hexagonal/trigonal and other symmetries (i.e., only the master pattern
computation in \textsf{CTEMEBSDmaster} needs to distinguish between the different Laue symmetries).

{\color{blue}Note that at the time of writing of this manual, the hexagonal/trigonal case is not yet completely implemented; this is ongoing work
and the source files will be updated when the modifications are completed.}

\newpage
\subsection{Monte Carlo Simulations \label{sec:MC}}
The current implementation of the Monte Carlo code makes use of the Continuous Slowing Down Approximation (CSDA),
which assumes that the incident electron loses energy at a constant rate as it travels inside the sample.  This is obviously a
rather coarse approximation, since core losses, outer shell excitations, plasmons, phonons, etc are not taken into account
explicitly, but replaced by a constant energy loss rate.  From a modeling point of view, on the other hand, it is relatively easy
to implement, so we have selected this as a first approximate approach over a more involved Monte Carlo model that has been
under development as well.  

In the Monte Carlo approach, we consider each beam electron separately and follow it on a stochastic trajectory 
through the sample.  The electron travels a random distance scaled by the mean free path length, and then undergoes a 
scattering event which changes the direction cosines of the trajectory; this process is repeated until either the electron has
lost more than a preset amount of energy, or the electron leaves the sample, at which point the direction cosines, energy,
and depth of the last scattering event are noted.  From this information, the program then generates two different data sets:
\begin{itemize}
	\item the spatial distribution of electrons, represented on a modified Lambert projection, for each of a series of 
	energy bins;
	\item and the depth distribution vs.\ energy and orientation.
\end{itemize}
This data is then used by the Master EBSD simulation program to determine the depth integration limits and the probability 
that an electron with a given energy was backscattered at a given depth (note that the depth is the effective distance traveled 
inside the sample after the last scattering event).  For more details on the Monte Carlo approach we refer the interested reader to
the literature.\footnote{Heinrich, K.H.J., Newbury, D.E. \& Yakowitz, H. (Eds.) (1975).
\textit{Use of Monte Carlo Calculations in Electron Probe Microanalysis and Scanning Electron Microscopy} Gaithersburg, MD: National Bureau of Standards;
Joy, D.C. (1995). \textit{Monte Carlo Modeling for Electron Microscopy and Microanalysis} New York: Oxford University Press.}

In version 2.x, the core of this Monte Carlo simulation is performed on a GPU platform, if one is available, using an OpenCL kernel.  The kernel source code 
must be available in the proper folder, which must be defined by means of an environment variable.

\subsection{Master EBSD Pattern Simulations \label{sec:Master}}
For the master pattern we employ the Bloch wave approach combined with a depth integration.  The probability 
of back-scattered electrons originating from the sample along a specific direction must be integrated over the depth
along that direction, which involves an integral over depth.  In our approach, this integral is replaced by a summation
over the depth bins generated in the \textsf{CTEMMC} program.  

The probability of scattering from each subset $\mathcal{S}$ of atomic sites within the 
unit cell for a given incident beam direction $\mathbf{k}_0$ and an exit energy $E$ can be written as:
\begin{equation}
\mathcal{P}(\mathbf{k}_0, E) = \sum_{\mathbf{g}}\sum_{\mathbf{h}} S_{\mathbf{g}\mathbf{h}}L_{\mathbf{g}\mathbf{h}}(E),
    \label{eq:prob}
\end{equation}
with
\begin{subequations}
\begin{align}
    S_{\mathbf{g}\mathbf{h}} &\equiv \sum_{n}\sum_{i\in\mathcal{S}_n} Z^2_n\,e^{-M^{(n)}_{\mathbf{h}-\mathbf{g}}}\,e^{2\pi\mathrm{i} 
    (\mathbf{h}-\mathbf{g})\cdot\mathbf{r}_{i}};\label{eq:defa}\\
    L_{\mathbf{g}\mathbf{h}}(E) &\equiv \sum_{j}\sum_{k} 
    C^{(j)\ast}_{\mathbf{g}}\alpha^{(j)\ast}\mathcal{I}_{jk}(E)\alpha^{(k)}
    C^{(k)}_{\mathbf{h}}.\label{eq:defb}
\end{align}
\end{subequations}
The first summation in (\ref{eq:defa}) runs over all the positions in the asymmetric unit of the unit cell; the second
sum runs over all equivalent positions in each subset $\mathcal{S}_n$.
The parameters $\alpha^{(j)}$ in (\ref{eq:defb}) are the Bloch wave excitation amplitudes for 
details on the Bloch wave approach), $C_{\mathbf{g}}^{(j)}$ are the 
Bloch wave coefficients, and the matrix $\mathcal{I}_{jk}$ is defined by the integral
\begin{equation}
	\mathcal{I}_{jk}(E)\equiv \frac{1}{z(E)}\int\limits_{0}^{z(E)} 
    \lambda(E,z) e^{-2\pi(\alpha_{jk}+\mathrm{i}\beta_{jk})z}\,\mathrm{d}z,
\end{equation}
where $z(E)$ and $\lambda(E,z)$ are, respectively, the integration depth and the depth- and 
energy-dependent scattering probability derived from the Monte Carlo program output,
and
\begin{subequations}
\begin{align}
    \alpha_{jk} &= q^{(j)}+q^{(k)};\\
    \beta_{jk} &= \gamma^{(j)}-\gamma^{(k)}.
\end{align}
\end{subequations}
The complex numbers $\lambda^{(j)}\equiv\gamma^{(j)}+\mathrm{i}q^{(j)}$ are the eigenvalues of the dynamical scattering matrix in the 
presence of absorption.  The integration is performed numerically by simply summing the result for a discrete series of depths.
The resulting data is stored as a modified Lambert projection grid (see previous sections).  

\begin{figure}[t]
\centering
\includegraphics[width=3in]{figs/geometry}
\caption{Typical geometry of sample and scintillator for an EBSD experiment.} \label{fig:geometry}
\end{figure}

In version 2.x, we replaced the Bloch wave approach by an equivalent scattering matrix approach, so that we can use the GPU platform
to compute the necessary matrix exponentials.  The Bloch wave approach requires an OpenCL version of the Lapack routines
which we have not been able to get to work thus far.  The derivation proceeds as follows.

Application of the scattering matrix approach to the EBSD modality requires a different ``Ansatz'' for the wave 
function $\Psi(\mathbf{r})$; we write:
\begin{equation}
	\Psi(\mathbf{r}) = \sum_{\mathbf{g}} \psi_{\mathbf{g}}(\mathbf{r}) 
	\mathrm{e}^{2\pi\mathrm{i}(\mathbf{k}_0+\mathbf{g})\cdot\mathbf{r}};\label{eq:planewaves}
\end{equation}
this expression implicitly assumes that the scattered electron can only travel along the directions predicted by the 
Bragg equation, $\mathbf{k}'=\mathbf{k}_0+\mathbf{g}$.  After substitution in the perfect crystal Schr\"odinger equation,
application of the high energy approximation, and the substitution:
\begin{equation}
	\psi_\mathbf{g}(\mathbf{r}) = S_{\mathbf{g}}(\mathbf{r}) \mathrm{e}^{\mathrm{i}\theta_{\mathbf{g}}},\label{eq:psig}
\end{equation}
where $\theta_{\mathbf{g}}$ is the phase of the Fourier coefficient $U_{\mathbf{g}}$, we obtain the following system of 
coupled first order differential equations:
\begin{equation}
    \frac{\mathrm{d} S_{\mathbf{g}}(z)}{\mathrm{d}z} =
    2\pi\mathrm{i}s_{\mathbf{g}}S_{\mathbf{g}}(z) + \mathrm{i}\pi {\sum_{\mathbf{g}'}}
    \frac{1}
    {q_{\mathbf{g}-\mathbf{g}'}}S_{\mathbf{g}'}(z),\label{eq:defectequation}
\end{equation}
where
\begin{equation}
	\frac{1}{q_{\mathbf{g}}} \equiv \frac{1}{\xi_{\mathbf{g}}} + \mathrm{i}
	\frac{\mathrm{e}^{\mathrm{i} (\theta^{\prime}_{\mathbf{g}}-\theta_{\mathbf{g}})}}{\xi^{\prime}_{\mathbf{g}}};
	\label{eq:defineq}
\end{equation}
the extinction distance $\xi_{\mathbf{g}}$ and anomalous absorption length $\xi'_{\mathbf{g}}$ are defined as:
\begin{equation}
	\frac{1}{\xi_{\mathbf{g}}}\equiv \frac{\vert U_{\mathbf{g}}\vert}{\vert\mathbf{k}_0+\mathbf{g}\vert\cos\alpha};\qquad
	\frac{1}{\xi'_{\mathbf{g}}}\equiv \frac{\vert U'_{\mathbf{g}}\vert}{\vert\mathbf{k}_0+\mathbf{g}\vert\cos\alpha};
\end{equation}
$U_{\mathbf{g}} = (2me/h^2) V_{\mathbf{g}} $, $\alpha$ is the angle between the beam direction and $\mathbf{k}_0+\mathbf{g}$, and $\mathbf{k}_0$ is 
the incident wave vector corrected for refraction.  The absorption potential Fourier coefficients are represented by $U'_{\mathbf{g}} = (2me/h^2) V'_{\mathbf{g}}$.

This set of equations can be represented in matrix form as:
\begin{equation}
	\frac{\mathrm{d}\mathbf{S}(z)}{\mathrm{d}z} = \mathrm{i}\mathcal{A}(\mathbf{r})\mathbf{S}(z),\label{eq:matrix}
\end{equation}
where the structure matrix $\mathcal{A}$ contains the excitation errors and the normal absorption coefficient $1/q_{\mathbf{0}}$ 
along its diagonal, and the interaction parameters $1/q_{\mathbf{g}}$ on the off-diagonal positions:
\begin{align*}
    \mathcal{A}_{\mathbf{g},\mathbf{g}} & = 2\pi s_{\mathbf{g}} + \frac{\pi}{q_{\mathbf{0}}};\\
    \mathcal{A}_{\mathbf{g},\mathbf{g}'} & = \frac{\pi}{q_{\mathbf{g}-\mathbf{g}'}}.
\end{align*}
The formal solution for a crystal of thickness $\epsilon$ is then given by:
\begin{equation}
	\mathbf{S}(\epsilon) = e^{\mathrm{i}\mathcal{A}\epsilon}\mathbf{S}(0) = \mathcal{S}(\epsilon)\mathbf{S}(0);
\end{equation}
the matrix exponential $\mathcal{S}(z)$ is commonly known as the \textit{scattering matrix}, and can be computed numerically
by means of the Pad\'e approximation \cite{moler2003a}.

Substitution of eq.~(\ref{eq:planewaves}) into the BSE yield equation results in the following expression 
for the BSE yield:
\begin{equation}
	\mathcal{P}(\mathbf{k}_0) = \frac{1}{z_0}\sum_{n}\sum_{i\in\mathcal{S}_n}\sum_{\mathbf{g}}\sum_{\mathbf{h}} \int\limits_{0}^{z_{0}} \mathrm{d}z\,
	 Z^2_n\,\mathrm{e}^{-M^{(n)}_{\mathbf{h}-\mathbf{g}}}\,\mathrm{e}^{2\pi\mathrm{i} (\mathbf{h}-\mathbf{g})\cdot\mathbf{r}_{i}} 
%	\mathrm{e}^{\mathrm{i}(\theta_{\mathbf{h}}-\theta_{\mathbf{g}})} 
	S^{\ast}_{\mathbf{g}}(z) S_{\mathbf{h}}(z).
    \label{eq:prob2}
\end{equation}
This expression can be simplified as for the Boch wave case to read:
\begin{equation}
	\mathcal{P}(\mathbf{k}_0) = \sum_{\mathbf{g}}\sum_{\mathbf{h}} S_{\mathbf{g}\mathbf{h}} L^{S}_{\mathbf{g}\mathbf{h}},
    \label{eq:prob2}
\end{equation}
where the matrices are defined as
\begin{subequations}
\begin{align}
    S_{\mathbf{g}\mathbf{h}} &\equiv \sum_{n}\sum_{i\in\mathcal{S}_n} Z^2_n\,\mathrm{e}^{-M^{(n)}_{\mathbf{h}-\mathbf{g}}}\,\mathrm{e}^{2\pi\mathrm{i} 
    (\mathbf{h}-\mathbf{g})\cdot\mathbf{r}_{i}};\label{eq:defnewa}\\
    L^{S}_{\mathbf{g}\mathbf{h}} &\equiv \frac{1}
    {z_{0}}\int\limits_{0}^{z_{0}} \mathrm{d}z\,  S^{\ast}_{\mathbf{g}}(z) S_{\mathbf{h}}(z).
    \label{eq:defnewb}
\end{align}
\end{subequations}
The superscript $S$ in $L^{S}_{\mathbf{g}\mathbf{h}}$ refers to the scattering matrix approach.  Note that the matrix 
$S_{\mathbf{g}\mathbf{h}}$ is identical to that for the Bloch wave approach in the previous section (eq.~(\ref{eq:defa})).  
As in the Bloch wave case, we must replace the integral above by an energy-dependent depth-weighted integration, as follows:
\begin{equation}
	L^{S}_{\mathbf{g}\mathbf{h}}(E) \equiv  \frac{1}{z(E)}\int\limits_{0}^{z(E)} \mathrm{d}z\,  
    \lambda(E,z) S^{\ast}_{\mathbf{g}}(z) S_{\mathbf{h}}(z).
\end{equation}
The entire computation of the matrix $L$ can be performed on the GPU, resulting in a tremendous acceleration compared to the 
multi-core CPU version.  Each OpenCL work-item performs an entire matrix computation, with the limitation that all 
matrices must have the same size (same number of beams); this requires a rather comprehensive rewrite of the fortran-90 
code in the main \textsf{CTEMEBSDmaster} program.


\subsection{Final EBSD Pattern Simulations \label{sec:EBSP}}
Consider the geometry in Fig.~\ref{fig:geometry}, which shows the sample-scintillator configuration 
for a typical EBSD experiment.  The diffraction geometry is fully determined if we can express the direction cosines of the line connecting
an arbitrary point on the scintillator screen to the interaction point $P$ on the sample surface, expressed with 
respect to the $(RD,TD,ND)$  triad.  We measure the scintillator coordinates with the $x_s$ axis in the direction 
opposite to TD, and the $y_s$ axis pointing up towards the objective lens.   The $z_s$ axis is then 
normal to the screen and forms a second right-handed unit triad.  These coordinates are measured with 
respect to the center of the screen and can be positive or negative (except for $z_s$ which is always positive).
In the scintillator reference frame, a single point on the screen has coordinates $(x_s,y_s,0)$ and the 
point $Q$ has coordinates $(0,0,\lambda)$.  The interaction point $P$ has coordinates $(x_{pc},y_{pc},L)$.
For simplicity, we will measure all coordinates in units of micrometers ($\mu$m).

The arbitrary point $(x_s,y_s,0)$ on the scintillator screen corresponds to a set of three direction cosines 
with respect to the sample reference frame.   The angle between ND and the scintillator normal is $\alpha=\frac{\pi}{2}-\sigma+\theta_c$.
If the pattern center were located on the ND axis, by rotating the detector assembly by a clockwise angle $\alpha$ around
the TD direction, then its $(RD,TD,ND)$ coordinates (keeping track of the definition of this reference frame) 
would be $(y_{pc}-y_s,x_{pc}-x_s,L)$.  After counterclockwise rotation to the actual camera orientation we 
find for the screen coordinates in the sample reference frame:
\[
	\mathbf{r}_g = \left[ (y_{pc}-y_s)\cos\alpha+L\sin\alpha, x_{pc}-x_s, -(y_{pc}-y_ss)\sin\alpha+L\cos\alpha\right].
\]
The length of this vector is simply:
\[
	\vert \mathbf{r}_g\vert \equiv \rho_s = \sqrt{L^2+(y_{pc}-y_s)^2 + (x_{pc}-x_s)^2}
\]
so that the direction cosines of a screen pixel $(x_s,y_s,0)$ in the (RD,TD,ND) reference frame are given by:
\begin{equation}
	\hat{\mathbf{r}}_g(x_s,y_s) = \frac{\mathbf{r}_g}{\rho_s}.\label{eq:dc}
\end{equation}
For a given crystallographic grain orientation, it is straightforward to convert these
direction cosines to an electron channeling direction (all rotations are implemented by means of quaternion arithmetic in our
implementation).  This transformed direction is then further converted
to coordinates $(X,Y)$ on the square grid of the master EBSD pattern, which is 
used as a look-up table (with appropriate  bi-linear interpolation) for the EBSD intensity at the scintillator
position $(x_s,y_s,0)$.   Repeating this for all scintillator pixels then provides the complete EBSD signal for a given grain orientation.

We represent the energy-dependent master EBSD pattern by the symbol $\mathcal{M}(\kappa,X,Y)$,  where the first component labels the energy bins, 
and $(X,Y)$ indicates the position on the square sampling grid.   The energy histogram from the Monte Carlo simulation is represented by the symbol
$\mathcal{E}(\kappa,i,j)$, where $(i,j)$ references the scintillator pixel $(x_s,y_s)=(i,j)\delta$.  The number of BSE electrons incident on that pixel can then be written as
a weighted sum over all $n_E$ energy bins:
\begin{equation}
	I_{\text{BSE}}(i,j) = \eta\sum_{\kappa=1}^{n_E} s(\kappa)\mathcal{E}(\kappa,i,j) \mathcal{M}\left(\kappa,X(x_s,y_s,\tilde{q}),Y(x_s,y_s,\tilde{q})\right),\label{eq:intensity}
\end{equation}
where $\tilde{q}$ is a unit quaternion that represents the grain orientation (derived from a triplet of Euler angles), and the coordinates 
$(X,Y)$ are computed from $(x_s,y_s)$ and $\tilde{q}$ by the sequential application of equation~(\ref{eq:dc}), the quaternion rotation of the resulting
direction cosines, and the inverse Lambert projection equations.  The prefactor $\eta$ in (\ref{eq:intensity}) is defined as:
\begin{equation}
	\eta = \frac{n_A I_0 \Delta t}{N_{MC}}
\end{equation}
where $I_0$ is the incident probe current in Amperes, $\Delta t$ the dwell time in seconds, $N_{MC}$ is the total number of incident electrons
used for the Monte Carlo simulation, and $n_A=6.241\times 10^{18}$ electrons per Coulomb.  The resulting units for $I_{\text{BSE}}$ are then number of  
electrons incident on the scintillator.  The factor $s(\kappa)$ represents the energy conversion efficiency of the scintillator and is currently set equal to unity
for all energies.

This is not the final EBSD pattern as acquired by the CCD camera.  One must still take into account the point spread function of the 
optics that projects the photons onto the CCD chip, Poisson noise, as well as any binning and contrast/brightness scaling that can 
be applied to the pattern. These final steps are currently only available in a preliminary form via the IDL visualization interface.


\section{The \protect\textsf{CTEMMC.f90} program\label{sec:f90MC}}
The Monte Carlo program must be executed first, and takes the following name list as input:
\fvset{frame=lines,formatcom=\color{blue},fontsize=\footnotesize}
\VerbatimInput{../templatefolder/CTEMMC.template}
The program uses OpenMP coding to carry out the simulations on multiple cores.  The input parameters
include the crystal file name (to compute the theoretical density and average atomic number), the sample
tilt angle with respect to the horizontal direction, a potential tilt around the \textsf{RD} axis, the number of pixels
along the modified Lambert projection edge (should be ``a multiple of $10$ plus one''), the number of incident electrons per computational thread along
with the number of threads, the incident beam energy, the minimum energy to be considered, and the energy interval,
as well as the maximum penetration depth and depth step size to be considered.  The data is then stored in a 
binary file (will be replaced by HDF-5 formatted output in the next version of the program).  It is noteworthy that 
a GPU-based version of this program is currently being developed.   It should be noted that the execution time of the 
following program, \textsf{CTEMEBSDmaster}, is partially determined by how many depth steps are selected in the
Monte Carlo computation; the more depth steps, the longer the dynamical simulation will take.

Since each OpenMP thread carries out the computation for the same number of incident electrons, the \textsf{CTEMMC} program takes 
about the  same time for a single thread as for multiple threads.  The output file contains a number of program parameters
that need to be passed on to the next program, as well as two large arrays that contain the electron distribution in the hemisphere on
the vacuum side of the sample surface, formatted as a modified Lambert projection.  Each energy bin has an individual Lambert 
projection.  The second large output array represents the distribution of electron exit depth as a function of energy bin and
orientation, but sampled on a smaller Lambert projection (with one-tenth the number of points along each axis).  This second output
array is used in the \textsf{CTEMEBSDmaster} program, whereas the first one is primarily used by the \textsf{CTEMEBSD} program
and the IDL display program.


\section{The \protect\textsf{CTEMEBSDmaster.f90} program\label{sec:f90EBSDmaster}}
The second program performs the computation of the EBSD master pattern on the
square or hexagonal modified Lambert projection, with all output on a square modified Lambert projection.
The input name list file is formatted as follows:
\fvset{frame=lines,formatcom=\color{blue},fontsize=\footnotesize}
\VerbatimInput{../templatefolder/CTEMEBSDmaster.template}
This short input file only requires the definition of the smallest $d$-spacing to be taken into account 
in the dynamical simulation, the number of pixels along the output Lambert projection, the name of the 
output file from the \textsf{CTEMMC} program, and the name of an output file.  The data file from
the Monte Carlo program already contains a lot of other information, so there is no need to duplicate 
this here.  This does imply that the Monte Carlo program must be completed before the 
master pattern can be computed.

This program can take a very long time to run; the current version runs on a single core but a multi-threaded version 
is being developed.  For each energy bin, a complete EBSD master pattern is computed, using the appropriate asymmetric unit according to
the crystal symmetry.  The output file contains a stack of modified Lambert projections, one for each energy bin, along with 
some other parameters.  The size of the Lambert projections does not need to be the same as the size of the Monte Carlo
Lambert projections.  The projections can be visualized using the IDL program.  Note that for complex crystal structures,
with multiple atom sites in the asymmetric unit, the program will store the master patterns separately for each entry in the asymmetric unit.
This means that one can, in principle, study the effect of elemental substitutions on EBSD patterns.

\section{The \protect\textsf{CTEMEBSD.f90} program\label{sec:f90EBSD}}
The final program takes as input a series of detector characteristics as well as orientations, and uses the output from both \textsf{CTEMMC} and
\textsf{CTEMEBSDmaster} to compute one or more actual EBSD patterns.  The easiest way to execute this program is 
via the IDL interface, but command line execution is also possible.  The input name list file contains the following entries:
\fvset{frame=lines,formatcom=\color{blue},fontsize=\footnotesize}
\VerbatimInput{../templatefolder/CTEMEBSD.template}

The program reads the name list file, then Monte Carlo output and master patterns;  then the detector geometry is used to
compute the energy distribution with respect to the scintillator.  This is then used along with the master pattern and the 
orientation information to compute the final EBSD patterns.  Note that these patterns may contain Poisson noise; this can
be turned on/off using the PoissonNoise switch in the name list file.  The output file contains all EBSPs in full resolution,
and can be read with the IDL visualization program; at that point, the camera point spread function can be included, as well
as binning and brightness/contrast scaling. See the next section for details.

Note that this program can also be used to create so-called EBSD dictionaries.  This is a much more advanced functionality 
that we will not fully detail in this manual, since it is still under development.

\newpage
\section{The IDL \protect\textsf{EBSDDisplay.pro} program\label{sec:idl}}
The \textsf{EBSDDisplay.pro} program is a complex interface that provides the user with the ability to handle the 
output from all three fortran-90 programs, \textsf{CTEMMC}, \textsf{CTEMEBSDmaster}, and \textsf{CTEMEBSD}.  The main
user window is shown in Fig.~\ref{fig:EBSDmain}, and consists of the following regions:
\begin{itemize}
	\item Left pane: Monte Carlo run parameters for a given input file (file can be selected using the \button{Load MC file}\ at the lower right
	of the main window).  These are simply displayed as information and can not be changed by the user.
	\item Right pane: The top half displays a few parameters for an EBSD master pattern file (which can be loaded with the \button{Load master file}).
	\item Information window: this shows program messages about file loading, changed parameters, etc.
	\item Button row: allows the user to load either an MC file or an EBSD master file (which will also automatically load the 
	corresponding MC file).  The \button{Define detector}\ will be described in more detail below.  The user also has the option to 
	create a log-file that will display all the entries shown also in the information window.
\end{itemize}

\begin{figure}[t]
\leavevmode\centering
\epsfxsize=6in\epsffile{figs/EBSDmain.eps}
\caption{\label{fig:EBSDmain}Main user interface for the EBSD suite of programs.}
\end{figure}

\subsection{Monte Carlo visualization interface\label{sec:idlMC}}
In each of the two panes of the main program widget, there is a \button{Display}, which brings up a graphical interface.  If the user has loaded a Monte Carlo data file, 
then the display window will look similar to the one shown in Fig.~\ref{fig:MCdisplay}.  There are two items on the top menu line, and a few additional widgets both 
above and below the graphics display.  Here are all the options:
\begin{itemize}
	\item \textsf{Projection Mode} menu:  this menu has three items, \textsf{Lambert [square]}, \textsf{Lambert [circle]}, and \textsf{Stereographic P.}, corresponding to the 
	Lambert projection explained in section~\ref{sec:Lambert}, the regular equal-area Lambert projection, and the equal-angle stereographic projection, respectively.  The data in the Monte Carlo
	file is stored in the square Lambert projection format.
	\item \textsf{Display Mode} menu: there are three entries in this menu, \textsf{Individual Energy Bin}, \textsf{Simple Energy Sum}, and \textsf{Simple Energy Sum RGB}.
	\begin{itemize}
		\item \textsf{Individual Energy Bin}: in this display mode, the graphics window will show the spatial distribution of BSEs for a single energy value.  The energy
		can be selected using the slides above the graphics window.  Note that the label above the slider indicates the sequential number of the energy in the 
		data file, not the actual energy which is shown in the small text box to the right once the slider is released.\footnote{This is not an optimal arrangement and 
		it will be changed in the next version.}
		\item \textsf{Simple Energy Sum}: in this mode, the slider bar is grayed out, and the graphics window will show the sum pattern over all energy values.  
		\item \textsf{Simple Energy Sum RGB}: this is similar to the previous mode, except that a color code (from Green to Blue to Red) 
		is used to indicate the various energy levels.
	\end{itemize}
\end{itemize}
Note that all three \textsf{Display Mode}s work for either of the \textsf{Projection Modes}; in a later version of the program, stereographic projection mode and 
3D sphere mode will be added to the \textsf{Projection Mode} menu.  The pattern that is displayed in the graphics window can be saved by pressing the 
\button{Save}; the file format must be set first using the File Format selector (jpeg, tiff, or bmp).  The minimum and maximum value of the displayed array
are also shown below the image region.



\begin{figure}[t]
\leavevmode\centering
\epsfxsize=3in\epsffile{figs/EBSDMC.eps}
\caption{\label{fig:MCdisplay}Monte Carlo interface.}
\end{figure}

\newpage
\subsection{Monte Carlo and Master Pattern visualization interface\label{sec:idlMCMP}}
When the user selects the \button{Display}\ in the Master Pattern pane, a similar widget will appear, but this time there 
will be two graphics windows, one for the Monte Carlo results (left) and one for the master pattern (right), as shown in 
Fig.~\ref{fig:MCMPdisplay}.  Each of the graphics areas has a separate set of min/max indicators as well as a \button{Save}.
Note that this widget has an additional pull-down menu next to the energy slider; initially, this menu will display the selection
\textsf{SUM all sites}.  For a crystal structure with more than one atom in the asymmetric unit, the pull-down menu will
display a list of all atoms along with the square of the atomic number (recall that the back-scattering cross section varies
approximately as the square of the atomic number).  When the user selects one of the atoms from the asymmetric unit, 
the graphics window on the right will display only the contribution from those sites (the one in the asymmetric unit
and all equivalent positions).  The \textsf{SUM all sites} option displays the sum over all atomic sites in the unit cell.
Separating the BSE contributions by element (asymmetric unit site) is useful, because it reveals which pattern is the dominant one.  
This will obviously depend on the atomic number for each site; for crystal structures with low elemental contrast\footnote{We refer
to the difference between the atomic numbers as ``elemental contrast;''  in BaTiO$_3$, for instance, we have $Z^2_{\text{Ba}}=3,136$,
$Z^2_{\text{Ti}}=484$, and $Z^2_{\text{O}}=3\times 64=192$, so that the EBSD pattern is dominated by the contributions from Ba.  In CaTiO$_3$, on the
other hand, we have $Z^2_{\text{Ca}}=400$, so the contrast between Ca and Ti is low, and the pattern will have contributions from
both elements.  In both cases, the O contribution would be rather small.}, the net master pattern may have approximately equal contributions 
from two or more elements.  For structures with high elemental contrast, usually one element (the one with the largest atomic number) will
dominate the EBSD pattern.

\begin{figure}[t]
\leavevmode\centering
\epsfxsize=6in\epsffile{figs/EBSDMCMP.eps}
\caption{\label{fig:MCMPdisplay}Monte Carlo and Master Pattern interface.}
\end{figure}

\newpage
\subsection{Detector interface\label{sec:idldetector}}
Clicking on the \button{Define detector}\ in the main program widget generates a new widget with several groups of user-definable parameters, as
shown in Fig.~\ref{fig:detector}.
On the left, the user can define the detector geometry (scintillator-sample distance, detector tilt, scintillator pixel size, number of pixels, and pattern
center location), the incident beam current, and the beam dwell time.  The Euler angle convention can also be set to TSL or HKL.  The energy range 
over which the EBSD patterns should be integrated can be set by means of the Min and Max droplists.  Finally, the output file name must be selected by 
pressing the \button{Set Output File Name}.

\begin{figure}[t]
\leavevmode\centering
\epsfxsize=4in\epsffile{figs/EBSDdetector.eps}
\caption{\label{fig:detector}Detector interface.}
\end{figure}

\begin{figure}[t]
\leavevmode\centering
\epsfxsize=3in\epsffile{figs/EBSDEBSP.eps}
\caption{\label{fig:EBSP}EBSP display interface (single pattern mode).}
\end{figure}

On the right side of the interface there are several boxes:
\begin{itemize}
	\item \textsf{Pattern Mode}: the display program has three display modes for EBSD patterns: \textit{Single Pattern}, \textit{Angle File},
	and \textit{Dictionary}.  They are explained in the following bullets.
	\item \textsf{Single Patterns Parameters}: After the user sets the output file name, the \button{Display Pattern} in this section 
	becomes active.  The user can set an Euler angle triplet for which the EBSP should be displayed (using all the detector and microscope
	parameters set in other parts of the interface).  Clicking on the \button{Display Pattern} will then bring up another interface that 
	displays the actual EBSP (Fig.~\ref{fig:EBSP}).  In addition to the Euler angles, the user can also specify an axis-angle pair which will
	be applied to the crystal orientation \textit{in addition to (and following) the Euler angles};  leaving the Euler angle set to $0.0$ allows for application 
	of only the axis-angle pair.
	\item \textsf{Angle File Parameters}: In this mode, the user selects an existing angle file (format described below); once the file has been 
	selected, the display will list the angle type (Euler or quaternion) and the number of angles in the file.  The \button{Go} next to the 
	title of this region will then be highlighted.  Clicking on this button will execute the \textsf{CTEMEBSD} program using all the parameters 
	that are currently set, and an EBSP will be computed for each entry in the angle file.  Once the computation is complete, an interface similar
	to that for the individual pattern will pop up (see Fig.~\ref{fig:EBSP2}); the main difference is that in this interface, there is no possibility of modifying the 
	imaging parameters; the program will use the parameters determined by the user in the \textit{Single Pattern} mode.  The display interface
	does offer the possibility of stepping through the entries in the angle file and displaying individual EBSPs along with their Euler angle
	triplets or quaternions.  There is also an option to generate separate output files for individual patterns or for the whole set.  
	\item \textsf{Dictionary Parameters}: this option is still under development and is disabled in the current Release of the \ctp.  In the next version,
	this option will allow the user to first generate a uniform sampling of the Rodrigues Fundamental Zone for the current crystal symmetry, and then
	generate a dictionary file with an EBSP for each sampling point.  % Such files can become very large, depending on the crystal symmetry.
\end{itemize}


\begin{figure}[t]
\leavevmode\centering
\epsfxsize=3in\epsffile{figs/EBSDEBSP2.eps}
\caption{\label{fig:EBSP2}EBSP display interface (angle file mode).}
\end{figure}

The format for the angle file is a simple text file, structured as follows:
\begin{verbatim}
	eu
	15
	34.5 25.9 183.0
	22.9 65.3 112.0
	...
\end{verbatim}
On the first line we have a two-character angle type identifier ('eu' for Euler angles, 'qu' for quaternions); the second line 
lists the number of angle entries in the file, and each subsequent line lists either the three Euler angles (in degrees, without commas)
or the four quaternion components.

A future version of the display program will make it possible for the user to refine detector and microscope parameters, given a number of experimental
patterns and orientations.  In addition, the next version will include the ability to apply a detector point spread function.  We'll also add
an option to save the parameters for a particular microscope setup, to allow for easy switching between instrument parameters.


\section{A worked example\label{sec:examples}}
In this final section, we describe a simple example of an EBSD pattern simulation for pure aluminum.  We begin by using the 
\textsf{CTEMmkxtal} program to enter the following data:
\begin{verbatim}
	crystal system ---> 1
	a [nm] = 0.405
	Enter space group number : 225
	Atomic number : 13
	Fractional coordinates etc. : 0.0, 0.0, 0.0, 1.0, 0.00746
	Another atom ? (y/n)  n
	Enter output file name : Al.xtal
\end{verbatim}
The room temperature Debye-Waller factor above comes from a paper by Peng et al.\footnote{L.-M. Peng, G. Ren, S.L. Dudarev, and M.J. Whelan,
``Debye-Waller factors and absorptive scattering factors of elemental crystals,'' Acta Cryst. (1996) A52, 456-470.}

The EBSD pattern simulation then consists of three consecutive program runs, described in the following subsections.

\subsection{Monte Carlo simulation}
In the \textsf{Examples/EBSDPatterns} folder you will find the following namelist file (CTEMMCexample.nml):
\fvset{frame=lines,formatcom=\color{blue},fontsize=\footnotesize}
\VerbatimInput{../Examples/EBSDPatterns/CTEMMCexample.nml}
Important note: make sure that the output file name includes the complete (absolute) path!!!
 
This will produce an output file with energy statistics for the range $[10,25]$ keV in $0.5$ keV steps
on an $801\times 801$ square Lambert grid.  The program can be executed by the following command line
(assuming the program runs in the example folder):
\begin{verbatim}
	../../exe/CTEMMC CTEMMCexample.nml
\end{verbatim}
After about 40 minutes,\footnote{iMac 3.4 GHz Intel Core i7} 
this run produced a file \textsf{Al\_MCoutput.data} of size $161,728,920$ bytes, along with the following output:
\begin{verbatim}
 Program name : CTEMMC.f90
 Monte Carlo Electron Trajectory Simulation for EBSD+Lambert

May 28 2014   8:46:47.535 AM

  File Al.xtal open for input

-->Crystal Structure Information<--
   a [nm]             :   0.40500
   b [nm]             :   0.40500
   c [nm]             :   0.40500
   alpha [deg]        :  90.00000
   beta  [deg]        :  90.00000
   gamma [deg]        :  90.00000
   Volume [nm^3]      :   0.06643013
   Space group #      :  225
   Space group symbol :  F m 3 m
   Generator String   : 16aODDaDODbOOOcOOOdOOOeOOO0
   Structure is centrosymmetric

   Number of asymmetric atom positions    1
   General position / atomic number / multiplicity :    1/13/  4 (Al)
   Equivalent positions  (x y z  occ  DWF)
          >     0.00000,  0.00000,  0.00000,  1.00000,  0.00746
          >     0.00000,  0.50000,  0.50000,  1.00000,  0.00746
          >     0.50000,  0.00000,  0.50000,  1.00000,  0.00746
          >     0.50000,  0.50000,  0.00000,  1.00000,  0.00746

 Mean inner potential [V]  17.77
 Wavelength corrected for refraction
 Relativistic correction factor [gamma] 1.048924
 Relativistic Accelerating Potential [V]     25629.32
 Electron Wavelength [nm]   0.76608E-02
 Interaction constant [V nm]^(-1)   0.16784E-01
 Density, avZ, avA =    2.69781  13.00000,  26.98154
 setting number of threads to   4
  Completed electron #   1000000 ; bse hits =   466324
  Completed electron #   2000000 ; bse hits =   931979
  ...
  Completed electron #   24000000 ; bse hits =   11175556
  Completed electron #   25000000 ; bse hits =   11641955
  
  All threads complete; saving data to file Al_MCoutput.data
  Total number of electrons generated =   100000000
  Number of electrons on detector       =     46574757
\end{verbatim}
The last lines of output indicate that the overall BSE yield is $46.6\%$; this is not the 
yield on the detector, but the total yield in the Northern hemisphere with respect to the 
sample surface (i.e., all BSEs are counted, regardless of the direction in which they
escape the surface).

\subsection{EBSD master pattern simulation}
The next step is the computation of the EBSD master pattern, using the \textsf{CTEMEBSDmaster} program. 
The example input file \textsf{CTEMEBSDMexample.nml} contains the following entries:
\fvset{frame=lines,formatcom=\color{blue},fontsize=\footnotesize}
\VerbatimInput{../Examples/EBSDPatterns/CTEMEBSDMexample.nml}
Important note: make sure that the input and output file names include the complete (absolute) path!!!

One should also set the Bethe potential parameters; in this case, since the computation is not very long anyway, 
we set both cutoff parameters equal to each other in the \textsf{BetheParameters.nml} file:
\fvset{frame=lines,formatcom=\color{blue},fontsize=\footnotesize}
\VerbatimInput{../Examples/EBSDPatterns/BetheParameters.nml}
The program is executed as follows:
\begin{verbatim}
	../../exe/CTEMEBSDmaster CTEMEBSDMexample.nml
\end{verbatim}
and, after about $100$ minutes, produces a file \textsf{Al\_EBSDmaster.data} of size $79,559,150$ bytes.  A partial program output
is as follows:
\begin{verbatim}
 Program name : CTEMEBSDmaster.f90
 EBSD Energy-dependent Master Pattern Simulation

May 28 2014   9:40:29.585 AM

opening Al_MCoutput.data
  NumEbins, numzbins, nsx, nsy, num_el, MCnthreads           31         101         400         400   100000000,           0
  EkeV, Ehistmin, Ebinsize, depthmax, depthstep   25.00000  10.00000   0.50000 100.00000,   1.00000
 -> completed reading Al_MCoutput.data
  File Al.xtal open for input

-->Crystal Structure Information<--
   a [nm]             :   0.40500
   b [nm]             :   0.40500
   c [nm]             :   0.40500
   alpha [deg]        :  90.00000
   beta  [deg]        :  90.00000
   gamma [deg]        :  90.00000
   Volume [nm^3]      :   0.06643013
   Space group #      :  225
   Space group symbol :  F m -3 m
   Generator String   : 16aODDaDODbOOOcOOOdOOOeOOO0
   Structure is centrosymmetric

   Number of asymmetric atom positions    1
   General position / atomic number / multiplicity :    1/13/  4 (Al)
   Equivalent positions  (x y z  occ  DWF)
          >     0.00000,  0.00000,  0.00000,  1.00000,  0.00746
          >     0.00000,  0.50000,  0.50000,  1.00000,  0.00746
          >     0.50000,  0.00000,  0.50000,  1.00000,  0.00746
          >     0.50000,  0.50000,  0.00000,  1.00000,  0.00746

  Laue group #   11  m-3m
Read Bethe parameters from BetheParameters.nml
 &BETHELIST
 WEAKCUTOFF =   40.0000,
 CUTOFF =   40.0000,
 SGCUTOFF =   5.000000E-02,
 /

Starting computation for energy bin    31 ; energy [keV] =  25.00

 Mean inner potential [V]  17.77
 Wavelength corrected for refraction
 Relativistic correction factor [gamma] 1.048924
 Relativistic Accelerating Potential [V]     25629.32
 Electron Wavelength [nm]   0.76608E-02
 Interaction constant [V nm]^(-1)   0.16784E-01
  Range of reflections along a*, b* and c* =    9  9  9
  Normal absorption length =    155.647
  Length of the master list of reflections :      537
 # independent beam directions to be considered =    80601
 completed beam direction   2500
 completed beam direction   5000
  ... many lines removed ...
 completed beam direction   77500
 completed beam direction   80000
Final data stored in file Al_EBSDmaster.data
\end{verbatim}
Note that this program computes a master pattern for each energy bin produced by the Monte Carlo program; in the
present case, there are $31$ energy bins, so the program executes the full dynamical computation that many times.
Output is stored on the square Lambert projection, and can be read by the \textsf{CTEMEBSD} program as well
as the IDL visualization interface \textsf{EBSDDisplay.pro}.

\subsection{EBSD pattern simulation}
When the Monte Carlo and master pattern files have been created, then one can start up the 
\textsf{EBSDDisplay} visualization program, or, if all imaging parameters are known, one can
run the \textsf{CTEMEBSD} program directly with a namelist file.

We leave it up to the reader to experiment with the namelist file approach.  For the IDL visualization interface, 
start up the program, then click on the \button{Load master file} to load both the master file and the Monte Carlo
file; using the \button{Display} one can then examine both the electron distribution vs.\ energy and the master
patterns.  Use the \button{Define detector} to bring up the detector interface; select \textsf{single pattern} mode
and fill in the detector parameters.  Select an output file, then enter an Euler angle triplet, click on the
\button{Display pattern} and you will see the corresponding EBSP.  Set the intensity scaling to gamma, and 
use the slider to select an appropriate pattern contrast.

To get a feel for how the various detector parameters affect the EBSP, you should try changing the parameters 
one by one.  For instance, decreasing either the beam current or the dwell time, the patterns will become more
noisy.  Note that these patterns do not include the point spread function of the detector; that option will become 
available in the next release of the software.



\end{document}



