
\documentclass[DIV=calc, paper=letter, fontsize=11pt]{scrartcl}	 % A4 paper and 11pt font size

\usepackage[body={6.5in,9.0in},
  top=1.0in, left=1.0in]{geometry}
  
\usepackage[english]{babel} % English language/hyphenation
\usepackage[protrusion=true,expansion=true]{microtype} % Better typography
\usepackage{amsmath,amsfonts,amsthm} % Math packages
\usepackage[svgnames]{xcolor} % Enabling colors by their 'svgnames'
\usepackage[hang, small,labelfont=bf,up,textfont=it,up]{caption} % Custom captions under/above floats in tables or figures
\usepackage{booktabs} % Horizontal rules in tables
\usepackage{fix-cm}	 % Custom font sizes - used for the initial letter in the document
\usepackage{epsfig}
\usepackage{sectsty} % Enables custom section titles
\allsectionsfont{\usefont{OT1}{phv}{b}{n}} % Change the font of all section commands

\usepackage{fancyhdr} % Needed to define custom headers/footers
\pagestyle{fancy} % Enables the custom headers/footers
\usepackage{lastpage} % Used to determine the number of pages in the document (for "Page X of Total")
\usepackage{color}

\usepackage{fancyvrb}% used to include files verbatim
%\usepackage{chemsym}

\usepackage{hyperref}

\usepackage[backend=bibtex,style=numeric,sorting=ydnt,maxnames=15]{biblatex}
\renewbibmacro{in:}{}

% Count total number of entries in each refsection
\AtDataInput{%
  \csnumgdef{entrycount:\therefsection}{%
    \csuse{entrycount:\therefsection}+1}}

% Print the labelnumber as the total number of entries in the
% current refsection, minus the actual labelnumber, plus one
\DeclareFieldFormat{labelnumber}{\mkbibdesc{#1}}    
\newrobustcmd*{\mkbibdesc}[1]{%
  \number\numexpr\csuse{entrycount:\therefsection}+1-#1\relax}


%\addbibresource[label=papers]{mypubs.bib}
%\addbibresource[label=books]{mypubs.bib}
%\addbibresource[label=edited]{mypubs.bib}
%\addbibresource[label=chapters]{mypubs.bib}


% Headers - all currently empty
\lhead{}
\chead{}
\rhead{}

% Footers
\lfoot{\textsf{CTEM Tasks}, \today}
\cfoot{}
\rfoot{\footnotesize Page \thepage\ of \pageref{LastPage}} % "Page 1 of 2"

\renewcommand{\headrulewidth}{0.0pt} % No header rule
\renewcommand{\footrulewidth}{0.4pt} % Thin footer rule

\usepackage{lettrine} % Package to accentuate the first letter of the text
\newcommand{\initial}[1]{ % Defines the command and style for the first letter
\lettrine[lines=3,lhang=0.3,nindent=0em]{
\color{DarkGoldenrod}
{\textsf{#1}}}{}}

\usepackage{titling} % Allows custom title configuration

\newcommand{\HorRule}{\color{DarkGoldenrod} \rule{\linewidth}{1pt}} % Defines the gold horizontal rule around the title

\pretitle{\vspace{-1.5in} \begin{center} \HorRule \fontsize{25}{25} \usefont{OT1}{phv}{b}{n} \color{DarkRed} \selectfont} % Horizontal rule before the title

\title{CTEM Tasks} % Your article title

\posttitle{\par\end{center}\vskip 0.5em} % Whitespace under the title

\preauthor{\begin{center}\large \lineskip 0.5em \usefont{OT1}{phv}{b}{sl} \color{DarkRed}} % Author font configuration

\author{\vspace*{-0.7in}} % Your name

\postauthor{\footnotesize \usefont{OT1}{phv}{m}{sl} \color{Black} % Configuration for the institution name

\par\end{center}\HorRule} % Horizontal rule after the title
\date{\today\protect\footnote{This document will list thoughts, ideas, and descriptions of what still needs to be done in terms of improving the CTEMsoft package.}}

\newcommand{\ctp}{\textsf{CTEMsoft-2013}}
%

\begin{document}
\maketitle
\renewcommand{\contentsname}{Table of Contents}
{\small\tableofcontents}

\newpage
\section{Status on 10/4/14}
The Summer of 2014 saw a major rewrite of the \ctp\ package, mainly to get rid of \textit{all} global variable 
definitions, which would make it easier to integrate portions of the package with the DREAM.3D code.  All of the 
modules have been completely rewritten, including an extensive revision of the \textsf{rotations} module and 
associated \textsf{Lambert} and \textsf{quaternions} modules.  Most of the larger programs have also been converted;
the only major conversion still left is for the ECCI and STEM defect image programs, which will require a bit of 
work since they make extensive use of the rotations module; how the changes to this module affect the other 
programs has not yet been verified.  Another program that needs to be converted is the large angle CBED program CTEMlacbed.

The current more-or-less stable Release version is 2.0 and is on the github public repository; so far only three issues have
been reported.

In the longer term, it is possible that large chunks of the package will be rewritten in C++ by Stuart Wright and the BlueQuartz people.
For now, we're not going to worry about that and we'll pretend that everything will remain in f90 with occasional dedicated OpenCL kernels.

In March 2015, a new module was created for data output in the Hierarchical Data Format (HDF5).  This was first implemented
for f90, but then it was realized that things could be made substantially simpler and more versatile by switching to fortran-2003,
which provides C-bindings and such.  data exchange with C programs should now be come much easier.  A special handler was
also written to convert the fortran name list files into string arrays that can be included in the HDF file.

March 2015 also saw a renaming of the package to \textsf{EMsoft}. 


\section{To-do List}
Here we list a few items that definitely need a lot of work at the moment.

\subsection{Simplification and unification of the IDL GUI programs}
The current IDL GUIs are useful, but a bit cumbersome to write.  They are currently dedicated 
to a single modality, and there is a lot of overlap between them.  This must be modified. Since
we now have the ability to include namelist files in the HDF files, that means that we need IDL
routines to read the HDF files, and display the content of the files. 

We also need to reduce the number of GUIs to just two, one for TEM-related programs and 
one for SEM.  This could be done by means of a master GUI.


\subsection{Creation of wave vector lists}
Currently, there are several routines that generate linked lists of wave vectors.  We need to review what we
are currently doing, and consolidate those routines into a smaller but more versatile set.  As a first guess,
here are the different modes:
\begin{itemize}
	\item single wave vector [ZAP, DCI, ECCI];
	\item wave vectors inside an illumination cone or pyramid [CBED, LACBED, ECP, ECCI];
	\item wave vectors inside an asymmetric unit [ECP, EBSD, LACBED];
	\item wave vector along a linear trace (between two Laue center positions) [DCI, ECCI];
	\item wave vector along a conical surface [PED];
	\item \ldots
\end{itemize}

So far, we have always defined wave vectors based on some form of sampling on a unit sphere.  To make things 
more general, we should probably use true 3D rotations in all cases, since that will be required for all the dictionary style
calculations.  By the way, we still need individual high resolution ``single pattern'' programs for each of the dictionary 
areas, in the same way that CTEMmbcbed computes a single high resolution pattern based on an orientation 
derived from CTEMlacbed.

For the defect imaging programs, we are always close to a zone axis orientation and currently we actually 
make use of the shortest g-vectors normal to the zone to define a reference frame; perhaps that is not 
the most general way of doing things...and we may have to replace that by a general approach to be able
to do truly random beam orientations for defect image simulations.

We will also need to review the complete symmetry reduction to the asymmetric unit, in particular whether or not two opposite wave vectors 
will produce the sample diffracted amplitudes for a non-centrosymmetric crystal.  This might have consequences for the computation
of master patterns in that potentially two master patterns may be needed for non-centrosymmetric materials.  The current code
contains two different ways of dealing with diffraction symmetry, in addition to the CBED symmetry which is the 
diffraction group approach.  It might be useful to look at the other symmetries from the point of view of the diffraction group.

Hexagonal and rhombohedral structures will also need to be considered in more detail, in particular when it comes to 
implementation of master pattern calculations on a hexagonal Lambert projection.  This was partially completed 
in the past, but needs to be revisited and corrected if necessary.

\subsection{Additional modalities}
PED (for the AStar system) and TKD are two modalities that come to mind that will need an extensive effort to 
make any progress.  PED is the easier one and it will benefit from the redefined wave vector routines.  A kinematical
version of PED has been implemented and can be used in connection with the dictionary approach.

In conjunction with DREAM.3D, integration of some SE and BSE imaging modalities might be really useful.



\subsection{Stuff to be done on GPU}
We need to have a general GPU routine to compute dynamical amplitudes for a given scattering matrix;
this might include computation of the scattering matrix itself.

\subsection{More modern GUIs}
The current IDL GUIs are functional but they don't look all that nice and are rather primitive compared to modern GUI standards.
Redoing them is a low priority at the moment.   We need to do the important things first and then potentially revisit a single GUI for
the whole package.  Perhaps a conversion into C++ might make that simpler?










\end{document}



