
\documentclass[DIV=calc, paper=letter, fontsize=11pt]{scrartcl}	 % A4 paper and 11pt font size

\usepackage[body={6.5in,9.0in},
  top=1.0in, left=1.0in]{geometry}
  
\usepackage[english]{babel} % English language/hyphenation
\usepackage[protrusion=true,expansion=true]{microtype} % Better typography
\usepackage{amsmath,amsfonts,amsthm} % Math packages
\usepackage[svgnames]{xcolor} % Enabling colors by their 'svgnames'
\usepackage[hang, small,labelfont=bf,up,textfont=it,up]{caption} % Custom captions under/above floats in tables or figures
\usepackage{booktabs} % Horizontal rules in tables
\usepackage{fix-cm}	 % Custom font sizes - used for the initial letter in the document
\usepackage{epsfig}
\usepackage{sectsty} % Enables custom section titles
\allsectionsfont{\usefont{OT1}{phv}{b}{n}} % Change the font of all section commands

\usepackage{fancyhdr} % Needed to define custom headers/footers
\pagestyle{fancy} % Enables the custom headers/footers
\usepackage{lastpage} % Used to determine the number of pages in the document (for "Page X of Total")
\usepackage{color}

\usepackage{fancyvrb}% used to include files verbatim
%\usepackage{chemsym}

\usepackage{hyperref}

\usepackage[backend=bibtex,style=numeric,sorting=ydnt,maxnames=15]{biblatex}
\renewbibmacro{in:}{}

% Count total number of entries in each refsection
\AtDataInput{%
  \csnumgdef{entrycount:\therefsection}{%
    \csuse{entrycount:\therefsection}+1}}

% Print the labelnumber as the total number of entries in the
% current refsection, minus the actual labelnumber, plus one
\DeclareFieldFormat{labelnumber}{\mkbibdesc{#1}}    
\newrobustcmd*{\mkbibdesc}[1]{%
  \number\numexpr\csuse{entrycount:\therefsection}+1-#1\relax}


%\addbibresource[label=papers]{mypubs.bib}
%\addbibresource[label=books]{mypubs.bib}
%\addbibresource[label=edited]{mypubs.bib}
%\addbibresource[label=chapters]{mypubs.bib}


% Headers - all currently empty
\lhead{}
\chead{}
\rhead{}

% Footers
\lfoot{\textsf{EMsoft HDF Routines}, \today}
\cfoot{}
\rfoot{\footnotesize Page \thepage\ of \pageref{LastPage}} % "Page 1 of 2"

\renewcommand{\headrulewidth}{0.0pt} % No header rule
\renewcommand{\footrulewidth}{0.4pt} % Thin footer rule

\usepackage{lettrine} % Package to accentuate the first letter of the text
\newcommand{\initial}[1]{ % Defines the command and style for the first letter
\lettrine[lines=3,lhang=0.3,nindent=0em]{
\color{DarkGoldenrod}
{\textsf{#1}}}{}}

\usepackage{titling} % Allows custom title configuration

\newcommand{\HorRule}{\color{DarkGoldenrod} \rule{\linewidth}{1pt}} % Defines the gold horizontal rule around the title

\pretitle{\vspace{-1.5in} \begin{center} \HorRule \fontsize{25}{25} \usefont{OT1}{phv}{b}{n} \color{DarkRed} \selectfont} % Horizontal rule before the title

\title{EMsoft HDF Routines} % Your article title

\posttitle{\par\end{center}\vskip 0.5em} % Whitespace under the title

\preauthor{\begin{center}\large \lineskip 0.5em \usefont{OT1}{phv}{b}{sl} \color{DarkRed}} % Author font configuration

\author{\vspace*{-0.7in}} % Your name

\postauthor{\footnotesize \usefont{OT1}{phv}{m}{sl} \color{Black} % Configuration for the institution name

\par\end{center}\HorRule} % Horizontal rule after the title
\date{\today\protect\footnote{This document describes all the HDF routines available in the HDFsupport.f90 module.}}

\newcommand{\ctp}{\textsf{EMsoft} package}
%

\begin{document}
\maketitle
\renewcommand{\contentsname}{Table of Contents}
{\small\tableofcontents}

\newpage
\section{Introductory comments}
Starting with Release 3.0.0, the \ctp\ includes support for the Hierarchical Data Format (HDF5).  This support consists
mostly of a single module, \textsf{HDFsupport.f90} in the \textsf{src} folder, which defines all the routines for reading
and writing datasets, as well as creating groups and so on.  While the H5LT API provides many of the same routines, it was
deemed necessary to create a separate module to obtain complete control over all options.  This also forced us to enable
support for the Fortran-2003 version of the language, which is fully supported by the open source \textsf{gfortran} compiler. The
2003 version provides extensive support for C language communications, as well as access to variables and such across 
languages.  The HDF5 package (currently hdf5-1.8.14 as off \today) must be compiled with fortran2003 support enabled,
otherwise the \textsf{HDFsupport} module will not compile.
In the remainder of this document, we decribe all available functions and subroutines, and provide a few brief examples 
of writing to and reading from \textsf{EMsoft} HDF files.


\section{The HDFsupport.f90 Module}
This section enumerates all the available routines in the \textsf{HDFsupport} module.  To make use of the module, the following
commands must be present at the start of the routine that will make use of them:
\begin{verbatim}
	use typedefs
	use HDF5
	use HDFsupport
	use ISO\_C\_BINDINGS
\end{verbatim}
The latter use-statement enables access to C-type variables and types.

\subsection{Push-pop stack}
HDF routines typically have a lot of \textsf{open} and \textsf{close} statements for all the types 
of objects available (file, group, dataset, attribute, etc.).  To make things easier for the user, 
all handling of the associated ID variables is kept hidden from the user through the use of a push-pop
stack.  The stack is a linked list and items are added to the top of the list in a \textsf{last-in--first-out}
approach.  Each stack entry is defined by the type:
\begin{verbatim}
! type definition for HDF-based output
type HDFobjectStackType   ! this is a push-pop stack to keep track of the open objects
  character(LEN=1)						:: objectType
  character(fnlen)						:: objectName
  integer(HID_T) 						:: objectID
  type(HDFobjectStackType),pointer		:: next
end type HDFobjectStackType
\end{verbatim}
If a routine wants to make use of the \textsf{HDFsupport} module, it must first declare
and nullify two variables of this type:
\begin{verbatim}
  type(HDFobjectStackType),pointer  :: HDF_head
  type(HDFobjectStackType),pointer  :: HDF_tail

  nullify(HDF_head)
  nullify(HDF_tail)
\end{verbatim}
Failure to \textsf{nullify(\,)} the pointers may result in unpredictable program behavior.  The two 
pointers are arguments to every \textsf{HDFsupport} call and should never be modified by the user 
program.  Note that the \textsf{HEAD\_tail} variable may be removed in a later version.

The HDF stack must be used by the calling program to close HDF objects; when an object is opened 
by an textsf{HDFsupport} routine, its information will be automatically pushed onto the stack, and the 
calling program does not need to access the stack.  Closing an object, however, requires user intervention
by means of the \textsf{HDF\_pop} routine, which can be called in two ways:
\begin{verbatim}
	call HDF_pop(HDF_head)
\end{verbatim}
will close the single object that is currently on the top of the stack. If multiple objects need to
be closed, then this call must be made multiple times.  It is the responsibility of the programmer to 
keep track of the hierarchical structure of the HDF file.

When the final write or read has been done, all currently open objects, including the HDF file itself, can
be closed by calling:
\begin{verbatim}
	call HDF_pop(HDF_head,.TRUE.)
\end{verbatim}
There is hence no need for the programmer to explicitly close an HDF file; a simple \textsf{HDF\_pop} call
will suffice.

\subsection{Initializing the fortran interface}
The HDF5 library makes use of many ``behind-the-scenes'' variables that must be initialized before
any HDF operation can be performed. This is done by the following code:
\begin{verbatim}
!
! Initialize the FORTRAN interface.
!
CALL h5open_f(hdferr)
if (hdferr.ne.0) then
    ! do some error handling
end if
\end{verbatim}
where \textsf{hdferr} is an integer variable that will equal $0$ when the initialization was successfull.

At the end of the program, or when no further HDF operations are needed, the fortran interface must be 
closed to correctly release any HDF-allocated memory resources:
\begin{verbatim}
!
! Close the FORTRAN interface.
!
CALL h5close_f(hdferr)
if (hdferr.ne.0) then
    ! do some error handling
end if
\end{verbatim}
In the remainder of this document, we will suppress any further statements of error handling.


\subsection{File level operations}
The following two function calls can be used to create a new file, and to open an existing file in readonly mode (optional):
\begin{verbatim}
hdferr = HDF_createFile(HDFname, HDF_head, HDF_tail)
hdferr = HDF_openFile(HDFname, HDF_head, HDF_tail, readonly)
\end{verbatim}
where \textsf{HDFname} is of the type \textsf{character(fnlen)} and is the filename, either the file itself or the full path.
Note that the \textsf{fnlen} parameter is defined in the \textsf{local.f90.in} file in the \textsf{src} folder and is currently set
to $132$ characters, reflecting the limit of characters per line of fortran-90 source code.

The first function above creates the new file and opens it, pushing the file object onto the stack.  The second function
attempts to open an existing file and also pushes the file object on the stack.  At that point, the HDF file is opened into
the root \textsf{/} level.  Note that the last parameter of the \textsf{HDF\_openFile} call is optional, and is of the \textsf{logical} type.
Its value (.TRUE. or .FALSE.) is not important; when it is present, file access will be in read-only mode, when it is absent, access will be in read-write mode.

\subsection{Group level operations}
There are two group level operations, one to create a new group, the other to open an existing group. In both cases, the group object ID will
be pushed onto the stack.  The routines are:
\begin{verbatim}
hdferr = HDF_createGroup(groupname, HDF_head, HDF_tail)
hdferr = HDF_openGroup(groupname, HDF_head, HDF_tail)
\end{verbatim}
where \textsf{groupname} is of type \textsf{character(fnlen)}.

\subsection{Dataset level operations}

\subsubsection{Create/open a dataset}
There are two dataset level operations, one to create a new dataset, the other to open an existing dataset. In both cases, the dataset object ID will
be pushed onto the stack.  The routines are:
\begin{verbatim}
hdferr = HDF_createDataset(dataname, HDF_head, HDF_tail)
hdferr = HDF_openDataset(dataname, HDF_head, HDF_tail)
\end{verbatim}
where \textsf{dataname} is of type \textsf{character(fnlen)}.

\subsubsection{Writing to a dataset}
There are $21$ write routines, grouped by the type of the dataset that is being written.
For strings, we have the following general routine (for brevity, note that the ... are shorthand for \textsf{HDF\_head, HDF\_tail}):
\begin{verbatim}
hdferr = HDF_writeDatasetStringArray(dataname, stringarray, nlines, HDF_head, HDF_tail)
hdferr = HDF_writeDatasetCharArray1D(dataname, chararr, dim0, ...)
hdferr = HDF_writeDatasetCharArray2D(dataname, chararr, dim0, dim1,  ...)
hdferr = HDF_writeDatasetCharArray3D(dataname, chararr, dim0, dim1, dim2,  ...)
hdferr = HDF_writeDatasetCharArray4D(dataname, chararr, dim0, dim1, dim2, dim3,  ...)
\end{verbatim}
where \textsf{dataname} is the dataset name (used to create the dataset), \textsf{stringarray} 
is an array of type \textsf{character(len=fnlen, KIND=c\_char)} and has \textsf{nlines} elements; note
that to write a single string, this string must be copied into a one-element stringarray(1). The other
four routines are used to write an array of individual characters (basically unsigned integers of length
$1$ byte, representing the interval $[0\ldots 255]$) to the file.

The module also provides a routine to write a complete text file to the HDF file:
\begin{verbatim}
hdferr = HDF_writeDatasetTextFile(dataname, filename, HDF_head, HDF_tail)
\end{verbatim}
where \textsf{filename} is of type \textsf{character(fnlen)}.  This will copy the entire text file 
into a variable length string array dataset, with each string terminated by the \textsf{C\_NULL\_CHAR}
character.  Note that currently the maximum string length is \textsf{fnlen}.

There are three write routines for single variables of the \textsf{integer}, \textsf{float} (=real4),
and \textsf{double} (=real8) formats:
\begin{verbatim}
hdferr = HDF_writeDatasetInteger(dataname, intval, HDF_head, HDF_tail)
hdferr = HDF_writeDatasetFloat(dataname, fltval, HDF_head, HDF_tail)
hdferr = HDF_writeDatasetDouble(dataname, dblval, HDF_head, HDF_tail)
\end{verbatim}
where the second argument in each call is the corresponding variable.

Finally, there are routines to write arrays of integers, floats, and doubles to the file; the arrays
can be 1D, 2D, 3D, or 4D :
\begin{verbatim}
hdferr = HDF_writeDatasetIntegerArray1D(dataname, intarr, dim0, ...)
hdferr = HDF_writeDatasetIntegerArray2D(dataname, intarr, dim0, dim1,  ...)
hdferr = HDF_writeDatasetIntegerArray3D(dataname, intarr, dim0, dim1, dim2,  ...)
hdferr = HDF_writeDatasetIntegerArray4D(dataname, intarr, dim0, dim1, dim2, dim3,  ...)

hdferr = HDF_writeDatasetFloatArray1D(dataname, fltarr, dim0,  ...)
hdferr = HDF_writeDatasetFloatArray2D(dataname, fltarr, dim0, dim1,  ...)
hdferr = HDF_writeDatasetFloatArray3D(dataname, fltarr, dim0, dim1, dim2,  ...)
hdferr = HDF_writeDatasetFloatArray4D(dataname, fltarr, dim0, dim1, dim2, dim3,  ...)

hdferr = HDF_writeDatasetDoubleArray1D(dataname, dblarr, dim0,  ...)
hdferr = HDF_writeDatasetDoubleArray2D(dataname, dblarr, dim0, dim1,  ...)
hdferr = HDF_writeDatasetDoubleArray3D(dataname, dblarr, dim0, dim1, dim2,  ...)
hdferr = HDF_writeDatasetDoubleArray4D(dataname, dblarr, dim0, dim1, dim2, dim3,  ...)
\end{verbatim}

\subsubsection{Reading from a dataset}
Every write routine from the previous section has a corresponding read version.  In 
all cases, the result of calling these fuctions is directly written into an allocatable 
array of the proper type, except for the single variable reading functions.  
The main program must not allocate these variables, it should
only declare them as allocatable, and then the HDF functions will perform the appropriate
allocations with the correct dimensions. The available read routines are:
\begin{verbatim}
stringarray = HDF_readDatasetStringArray(dataname, nlines, ...)
hdferr = HDF_extractDatasetTextfile(dataname, textfile, ...) 

chararr1D = HDF_readDatasetCharArray1D(dataname, dims, ...) 
chararr2D = HDF_readDatasetCharArray2D(dataname, dims, ...) 
chararr3D = HDF_readDatasetCharArray3D(dataname, dims, ...) 
chararr4D = HDF_readDatasetCharArray4D(dataname, dims, ...) 

intdata = HDF_readDatasetInteger(dataname, ...)
fltdata = HDF_readDatasetFloat(dataname, ...)
dbldata = HDF_readDatasetDouble(dataname, ...) 

intarr1D = HDF_readDatasetIntegerArray1D(dataname, dims, ...) 
intarr2D = HDF_readDatasetIntegerArray2D(dataname, dims, ...) 
intarr3D = HDF_readDatasetIntegerArray3D(dataname, dims, ...) 
intarr4D = HDF_readDatasetIntegerArray4D(dataname, dims, ...) 

fltarr1D = HDF_readDatasetFloatArray1D(dataname, dims, ...) 
fltarr2D = HDF_readDatasetFloatArray2D(dataname, dims, ...) 
fltarr3D = HDF_readDatasetFloatArray3D(dataname, dims, ...) 
fltarr4D = HDF_readDatasetFloatArray4D(dataname, dims, ...) 

dblarr1D = HDF_readDatasetDoubleArray1D(dataname, dims, ...)
dblarr2D = HDF_readDatasetDoubleArray2D(dataname, dims, ...)
dblarr3D = HDF_readDatasetDoubleArray3D(dataname, dims, ...)
dblarr4D = HDF_readDatasetDoubleArray4D(dataname, dims, ...)
\end{verbatim}
The second routine specifically extracts an array of strings from the dataset, and
stores it as a textfile with the name \textsf{textfile}; no strings are returned to the 
calling program.

As an example of the use of these functions, consider a dataset named \textsf{MyArray}, which 
is of type \textsf{float} (=real4) and has dimensions $400\times 400 \times 5$.  This array can
then be read from the file (assuming that the dataset has been opened) into a variable named \textsf{vals}
as follows:
\begin{verbatim}
! in the variable declaration section
	real(kind=sgl),allocatable	:: vals(:,:,:)
	integer(kind=irg)			:: dims(3)
	
! after the dataset has been opened
	dataname = 'MyArray'
	vals = HDF_readDatasetFloatArray3D(dataname, dims, HDF_head, HDF_tail)
\end{verbatim}
The array \textsf{vals} then contains the values read from the file, and \textsf{dims} will
contain the dimensions of the array.

\subsubsection{Writing and reading hyperslabs}
In many cases, we do not need to write or read an entire large dataset, but we may want to 
write it in segments.  In HDF parlance, these segments are called ``hyperslabs''.  The $12$ routines 
below create a data space of dimensions \textsf{hdims} (which is an \textsf{HSIZE\_T}-type array
of $2$, $3$, or $4$ entries), and write the hyperslab \textsf{wdata} of dimensions \textsf{dim0, dim1, \ldots}
at an offset of \textsf{offset} (again an \textsf{HSIZE\_T}-type array
of $2$, $3$, or $4$ entries) into the larger dataset.  It is up to the user to make sure that 
the dimensions are consistent and will fit in the available space.  The final argument \textsf{insert} is
an optional logical argument; when present, the hyperslab will be written to an already existing dataset.
So, the first time these functions would be called without the \textsf{insert} argument, and for every
additional write to the dataset, the keyword must be present.

{\footnotesize
\begin{verbatim}
hdferr = HDF_writeHyperslabCharArray2D(dataname, wdata, hdims, offset, dim0, dim1, ..., [insert])
hdferr = HDF_writeHyperslabCharArray3D(dataname, wdata, hdims, offset, dim0, dim1, dim2, ..., [insert])
hdferr = HDF_writeHyperslabCharArray4D(dataname, wdata, hdims, offset, dim0, dim1, dim2, dim3, ..., [insert])
hdferr = HDF_writeHyperslabIntegerArray2D(dataname, wdata, hdims, offset, dim0, dim1, ..., [insert])
hdferr = HDF_writeHyperslabIntegerArray3D(dataname, wdata, hdims, offset, dim0, dim1, dim2, ..., [insert])
hdferr = HDF_writeHyperslabIntegerArray4D(dataname, wdata, hdims, offset, dim0, dim1, dim2, dim3, ..., [insert])
hdferr = HDF_writeHyperslabFloatArray2D(dataname, wdata, hdims, offset, dim0, dim1, ..., [insert])
hdferr = HDF_writeHyperslabFloatArray3D(dataname, wdata, hdims, offset, dim0, dim1, dim2, ..., [insert])
hdferr = HDF_writeHyperslabFloatArray4D(dataname, wdata, hdims, offset, dim0, dim1, dim2, dim3, ..., [insert])
hdferr = HDF_writeHyperslabDoubleArray2D(dataname, wdata, hdims, offset, dim0, dim1, ..., [insert])
hdferr = HDF_writeHyperslabDoubleArray3D(dataname, wdata, hdims, offset, dim0, dim1, dim2, ..., [insert])
hdferr = HDF_writeHyperslabDoubleArray4D(dataname, wdata, hdims, offset, dim0, dim1, dim2, dim3, ..., [insert])
\end{verbatim}
}

The hyperslabs can be read from the file using the following $12$ functions, with arguments already explained above; the
functions return a complete array, which must have the status \textsf{allocatable} in the calling program, but must
not already be allocated.
\begin{verbatim}
chararr2D = HDF_readHyperslabCharArray2D(dataname, offset, dims, ...)
chararr3D = HDF_readHyperslabCharArray3D(dataname, offset, dims, ...)
chararr4D = HDF_readHyperslabCharArray4D(dataname, offset, dims, ...)
intarr2D = HDF_readHyperslabIntegerArray2D(dataname, offset, dims, ...)
intarr3D = HDF_readHyperslabIntegerArray3D(dataname, offset, dims, ...)
intarr4D = HDF_readHyperslabIntegerArray4D(dataname, offset, dims, ...)
fltarr2D = HDF_readHyperslabFloatArray2D(dataname, offset, dims, ...)
fltarr3D = HDF_readHyperslabFloatArray3D(dataname, offset, dims, ...)
fltarr4D = HDF_readHyperslabFloatArray4D(dataname, offset, dims, ...)
dblarr2D = HDF_readHyperslabDoubleArray2D(dataname, offset, dims, ...)
dblarr3D = HDF_readHyperslabDoubleArray3D(dataname, offset, dims, ...)
dblarr4D = HDF_readHyperslabDoubleArray4D(dataname, offset, dims, ...) 
\end{verbatim}





\subsection{Example source code}
The following example contains code that reads a namelist file for a given program (in this case the \textsf{EMKossel} program
and creates an HDF file that contains the entire file in the \textsf{NMLFiles} group, as well as a parsed version of the 
file in the \textsf{NMLparameters} group.  Then the program writes a few data items to the file in the \textsf{EMData} group, 
and closes the file.

{\small\begin{verbatim}
program hdf_writetest

  use local
  use HDF5
  use typedefs
  use HDFsupport
  use NameListTypedefs
  use NameListHandlers
  use NameListHDFwriters
  use ISO_C_BINDING
  
  IMPLICIT NONE

  CHARACTER(fnlen)                  :: filename, groupname, dataset, nmlname, programname
  
  CHARACTER(len=fnlen, KIND=c_char), DIMENSION(1), TARGET  :: line 
  CHARACTER(len=fnlen, KIND=c_char), ALLOCATABLE, TARGET   :: lines(:) 

  character(len=1)                  :: chararr(256)
  character(len=1)                  :: chararr2(256,256)

  INTEGER                           :: i, j, length, nlines, hdferr
  integer(kind=irg)                 :: intarr(8), dim0, dim1, intarr2(3,3), dims(1), dims2(2), &
                                       intdata(100,100), sdata(10,10), offset(2), cnt(2)
  real(kind=sgl)                    :: fltarr(8), realdata(20, 20)
  real(kind=dbl)                    :: dblarr(8)

  type(HDFobjectStackType),pointer  :: HDF_head
  type(HDFobjectStackType),pointer  :: HDF_tail
  type(KosselNameListType)          :: knl

  character(11)                     :: dstr
  character(15)                     :: tstrb
  character(15)                     :: tstre

  nullify(HDF_head)
  nullify(HDF_tail)

! read the namelist file into the knl structure
nmlname = 'EMKossel.nml'
call GetKosselNameList(nmlname,knl)

! get date and time stamps
call timestamp(datestring=dstr, timestring=tstrb)

! since there is no actual computation involved, set end time equal to start time
tstre = tstrb

! Initialize FORTRAN interface.
CALL h5open_f(hdferr)

! Create a new HDF5 file
filename = 'HDFtest.h5'
hdferr =  HDF_createFile(filename, HDF_head, HDF_tail)

! write the EMheader to the file
programname = 'hdf_writetest.f90'
call HDF_writeEMheader(HDF_head, HDF_tail, dstr, tstrb, tstre, programname)

! create a namelist group to write all the namelist files into
groupname = "NMLfiles"
hdferr = HDF_createGroup(groupname, HDF_head, HDF_tail)

! read the text file and write the array to the file
  dataset = 'KosselNML'
  hdferr = HDF_writeDatasetTextFile(dataset, nmlname, HDF_head, HDF_tail)

! leave this group
call HDF_pop(HDF_head)

! create a NMLparameters group to write all the namelist entries into
groupname = "NMLparameters"
hdferr = HDF_createGroup(groupname, HDF_head, HDF_tail)

  call HDFwriteKosselNameList(HDF_head, HDF_tail, knl)

! and leave this group
call HDF_pop(HDF_head)

! then the remainder of the data in a EMData group
groupname = 'EMData'
hdferr = HDF_createGroup(groupname, HDF_head, HDF_tail)

allocate(lines(2))
lines(1) = 'This is line 1'
lines(2) = 'and this is line 2'
dataset = 'StringTest'
hdferr = HDF_writeDatasetStringArray(dataset, lines, 2, HDF_head, HDF_tail)
deallocate(lines)

intarr = (/ 1, 2, 3, 4, 5, 6, 7, 8 /)
dataset = 'intarr1D'
dims = shape(intarr)
dim0 = dims(1)
hdferr = HDF_writeDatasetIntegerArray1D(dataset, intarr, dim0, HDF_head, HDF_tail)

intarr2 = reshape( (/ 1,2,3,4,5,6,7,8,9 /), (/3,3/))
dataset = 'intarr2D'
dims2 = shape(intarr2)
dim0 = dims2(1)
dim1 = dims2(2)
hdferr = HDF_writeDatasetIntegerArray2D(dataset, intarr2, dim0, dim1, HDF_head, HDF_tail)

fltarr = (/ 1.0, 2.0, 3.0, 4.0, 5.0, 6.0, 7.0, 8.0 /)
dataset = 'fltarr1D'
dims = shape(fltarr)
dim0 = dims(1)
hdferr = HDF_writeDatasetFloatArray1D(dataset, fltarr, dim0, HDF_head, HDF_tail)

dblarr = (/ 1.D0, 2.D0, 3.D0, 4.D0, 5.D0, 6.D0, 7.D0, 8.D0 /)
dataset = 'dblarr1D'
dims = shape(dblarr)
dim0 = dims(1)
hdferr = HDF_writeDatasetDoubleArray1D(dataset, dblarr, dim0, HDF_head, HDF_tail)

! some character arrays
do i1=0,255
  chararr(i1+1) = char(i1)
end do

dataset = 'chararray1D'
dim0 = 256
hdferr = HDF_writeDatasetCharArray1D(dataset, chararr, dim0, HDF_head, HDF_tail)

do i1=0,255
 do i2=0,255
  chararr2(i1+1,i2+1) = char(mod(i1+i2,256))
 end do
end do

dataset = 'chararray2D'
dim0 = 256
hdferr = HDF_writeDatasetCharArray2D(dataset, chararr2, dim0, dim0, HDF_head, HDF_tail)

do i = 1, 100
  do j = 1, 100
    intdata(i,j) = (i-1) + (j-1);
  end do
end do

! and finally two hyperslab examples
dataset = 'hyperslab'
sdata = intdata(1:10,1:10)
dims2 = (/ 1000, 1000 /)
cnt = (/ 10, 10 /)
offset = (/ 5, 5 /)
hdferr = HDF_writeHyperslabIntegerArray2D(dataset, sdata, dims2, offset, cnt(1), cnt(2), HDF_head, HDF_tail)
insert = .TRUE.
offset = (/ 0, 0 /)  ! this will partially overwrite the previous hyperslab
hdferr = HDF_writeHyperslabIntegerArray2D(dataset, sdata, dims2, offset, cnt(1), cnt(2), HDF_head, HDF_tail, insert)

do i=1,20
  do j=1,20
    realdata(i,j) = float(i-1) + float(j+1)
  end do
end do

dataset = 'realhyperslab'
dims2 = (/ 100, 100 /)
cnt = (/ 20, 20 /)
offset = (/ 2, 2 /)
hdferr = HDF_writeHyperslabFloatArray2D(dataset, realdata, dims2, offset, cnt(1), cnt(2), HDF_head, HDF_tail)

! close all objects, including the file
call HDF_pop(HDF_head,.TRUE.)

! close the Fortran interface
call h5close_f(hdferr)

end program hdf_writetest
\end{verbatim}}

This creates a file \textsf{HDFtest.h5} than can be analyzed with the java-based \textsf{HDFView} program, available
from the HDF group web site.  Fig.~\ref{fig:HDFView} shows the internal layout of the file, with all the groups and
datasets.

\newpage
\begin{figure}[h]
\leavevmode\centering
\epsfysize=6in\epsffile{figs/HDFView}
\caption{\label{fig:HDFView}Internal layout of the \textsf{HDFtest.h5} file as represented by the Java \textsf{HDFView} program.}
\end{figure}


In the second example, this file is opened to read some of the data entries and write them to the console.
{\small\begin{verbatim}
program hdf_readtest

  use local
  use HDF5
  use typedefs
  use HDFsupport
  use NameListTypedefs
  use NameListHandlers
  use NameListHDFwriters
  use ISO_C_BINDING
  
  IMPLICIT NONE

  character(fnlen)                 :: filename, groupname, dataset, textfile
  
  character(len=fnlen, KIND=c_char), ALLOCATABLE, TARGET   :: lines(:) 

  integer(HSIZE_T)                  :: dims(1), dims2(2)

  integer                           :: i, j, length, nlines, hdferr, offset(2), cnt(2)
  integer(kind=irg),allocatable     :: rdintarr(:), rdintarr2(:,:), intarr10(:,:)
  real(kind=sgl),allocatable        :: rdfltarr(:), realarr10(:,:)
  real(kind=dbl),allocatable        :: rddblarr(:)
  character(len=1),allocatable      :: rdchararr(:)

  type(HDFobjectStackType),pointer  :: HDF_head
  type(HDFobjectStackType),pointer  :: HDF_tail
  type(KosselNameListType)          :: knl

  character(11)                     :: dstr
  character(15)                     :: tstrb
  character(15)                     :: tstre

  nullify(HDF_head)
  nullify(HDF_tail)

! Initialize FORTRAN interface.
CALL h5open_f(hdferr)

! Open the file using the default properties.
filename = 'HDFtest.h5'
hdferr =  HDF_openFile(filename, HDF_head, HDF_tail)

! open the NMLfiles group
groupname = 'NMLfiles'
hdferr = HDF_OpenGroup(groupname, HDF_head, HDF_tail)

! read a dataset
  dataset = 'KosselNML'
  lines = HDF_readDatasetStringArray(dataset, nlines, HDF_head, HDF_tail) 
  write (*,*) 'data set name : ',trim(dataset),':  number of lines read = ',nlines

  do i=1,nlines
    write (*,*) lines(i)
  end do
  deallocate(lines)

! extract and write the KosselNML array to a test file 
  textfile = 'test.nml'
  hdferr = HDF_extractDatasetTextfile(dataset, textfile, HDF_head, HDF_tail)

call HDF_pop(HDF_head)

! next, read one of the integer string arrays

! open the EMData group
groupname = 'EMData'
hdferr = HDF_OpenGroup(groupname, HDF_head, HDF_tail)

! remember that the read routines do the allocations !
dataset = 'StringTest'
lines = HDF_readDatasetStringArray(dataset, nlines, HDF_head, HDF_tail)
do i=1,nlines
  write (*,*) lines(i)
end do
deallocate(lines)

dataset = 'intarr1D'
rdintarr = HDF_readDatasetIntegerArray1D(dataset, dims, HDF_head, HDF_tail)

write (*,*) 'shape of read intarr = ',shape(rdintarr)
write (*,*) rdintarr
deallocate(rdintarr)

dataset = 'intarr2D'
rdintarr2 = HDF_readDatasetIntegerArray2D(dataset, dims2, HDF_head, HDF_tail)

write (*,*) 'shape of read intarr2 = ',shape(rdintarr2)
write (*,*) rdintarr2
deallocate(rdintarr2)

dataset = 'fltarr1D'
rdfltarr = HDF_readDatasetFloatArray1D(dataset, dims, HDF_head, HDF_tail)

write (*,*) 'shape of read fltarr = ',shape(rdfltarr)
write (*,*) rdfltarr
deallocate(rdfltarr)

dataset = 'dblarr1D'
rddblarr = HDF_readDatasetDoubleArray1D(dataset, dims, HDF_head, HDF_tail)

write (*,*) 'shape of read dblarr = ',shape(rddblarr)
write (*,*) rddblarr
deallocate(rddblarr)

dataname = 'chararray1D'
rdchararr = HDF_readDatasetCharArray1D(dataname, dims, HDF_head, HDF_tail)

write (*,*) 'shape of chararray = ',shape(rdchararr)
write (*,*) rdchararr

! and finally, read a hyperslab from the hyperslab dataset
dataname = 'hyperslab'
offset = (/ 3, 3 /)
cnt = (/ 10, 10 /)
intarr10 = HDF_readHyperslabIntegerArray2D(dataname, offset, cnt, HDF_head, HDF_tail)

write (*,*) 'hyperslab read : '
do i=1,10
  write (*,"(10I3)") (intarr10(i,j),j=1,10)
end do

dataname = 'realhyperslab'
offset = (/ 2, 2 /)
cnt = (/ 20, 20 /)
realarr10 = HDF_readHyperslabFloatArray2D(dataname, offset, cnt, HDF_head, HDF_tail)

write (*,*) 'real hyperslab read : '
do i=1,10
  write (*,"(20F6.1)") (realarr10(i,j),j=1,20)
end do

! close all objects, including the file
call HDF_pop(HDF_head,.TRUE.)

! close the Fortran interface
call h5close_f(hdferr)

end program hdf_readtest
\end{verbatim}}
Note that the \textsf{dims} and \textsf{dims2} arrays are of the HDF integer type \textsf{HSIZE\_T}, which
is an 8-byte integer.


\section{Crystal structure data in HDF5 format}
The crystal structure data has thus far been stored in binary form in files of the type \textsf{*.xtal}. 
A series of new routines was created to store crystal structure data in HDF5 format, which makes the 
files editable by the HDFView program.  These files will also be stored in a single folder, the location 
of which can be selected by the user by means of an environmental variable.  A mechanism is provided that 
will recognize data files in the old format, and automatically convert them to the new HDF5 format and 
store them in the dedicated folder; in the old approach, \textsf{*.xtal} files would be scattered all 
over the filesystem.  From a user point of view, everything looks the same, except for the dedicated folder.

The environmental variable should be set as follows:
\begin{verbatim}
% in .cshrc file  [folder name 'XtalFolder' can be anything]
    setenv EMsoftxtalpath /some/path/XtalFolder
% in .bash_profile
    EMsoftxtalpath=/some/path/XtalFolder
    export EMsoftxtalpath
\end{verbatim}

\begin{figure}[h]
\leavevmode\centering
\epsfysize=4in\epsffile{figs/XtalFile}
\caption{\label{fig:XtalFile}Internal layout of a typical \textsf{*.xtal} structure file as represented by the Java \textsf{HDFView} program.}
\end{figure}

New structure files are generated with the \textsf{EMmkxtal} program.  In contrast with earlier versions
of the programs, this is now the only way to create a structure input file.  The output of the \textsf{EMmkxtal}
program is an HDF file (suggested extension: \textsf{.xtal}) with the internal structure shown in Fig.~\ref{fig:XtalFile}.




\end{document}



