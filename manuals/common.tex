At the time of writing of this manual, these programs have been successfully compiled on the Mac OS X platform using the public domain gfortran compiler.  There is no basic reason why this code should not work on Windows and Linux systems, and efforts are underway
to make sure that the code will run on all platforms. It would be interesting to see some of this code
implemented in a super computer setting, since many of the routines should be quite parallellizable.  Where possible, some of the computations
are carried out using the OpenMP directives, so that multiple cores can be used.  This aspect of the code is still under further development,
and we anticipate that GPUs will become useful as well, using the OpenCL approach.

The \ctp\ package is entirely written in f90 and does not use any of the newer commands available in the 95 and 2013 versions.
The source code is extensively commented, using regular comment lines, but also using DOxygen documentation generation commands.  Hence
there exists an extensive on-line documentation of all variables, variable types, modules, subroutines, functions, etc. for the latest version 
of the code.  For selected programs, more extensive manual pages are available.  If interested, please contact the author for further information.

The visualization part of the code consists of a series of  IDL routines that are available as source code or in the form of a Virtual Machine application. 
If you have an IDL license, then you can compile and run the IDL source code; alternatively, if you do not have a license,
then you can use one of the VM apps to perform the same task.  Note that in the VM environment, you will not be able to alter/compile the 
source code.